\section{Svolgimento}

\subsection{Confronto con Imola Informatica sulla Specifica Architetturale} 
In seguito al collegamento di tutti i membri del team e dei rappresentati dell'azienda Imola Informatica, \textit{Qi Fan Andrea Pan} inizia l'esposizione relativa agli esempi di Specifica Architetturale a cui ha lavorato nei giorni precedenti all'incontro. In particolare le due soluzioni si differenziano nel fatto che una segue un approccio di tipo \glossario{Stateless} mentre l'altra segue un approccio di tipo \glossario{Stateful}. \newline
I rappresentanti di Imola Informatica riferiscono al team che entrambe le soluzioni proposte risultano corrette da un punto di vista logico. Successivamente fanno notare al gruppo i limiti e le criticità nell'utilizzo di un approccio di tipo \glossario{Stateful}, ovvero: 
\begin{itemize}
    \item Problema relativo alla scalabilità: essendo tutti i dati gestiti dal server con un aumento di utenti sarebbe necessario comprare nuovi server (\textit{scalabilità orizzontale}) oppure potenziare quelli già in possesso (\textit{scalabilità verticale});
    \item Gestione della sessione: l'apertura di una sessione risulta essere esclusiva di un determinato server. Per fare in modo che l'utente possa ritrovare la sua sessione, il team dovrebbe implementare tecniche di gestione della condivisione delle risorse tra istanze di server diverse oppure un approccio che utilizzi le \glossario{Sticky Session}.
\end{itemize}
Considerate tali criticità relative ad un approccio di tipo \glossario{Stateful}, il team e i rappresentanti di Imola Informatica concordano che la Specifica Architetturale basata sul principio \glossario{Stateless} sia la più adatta per il corretto proseguimento del progetto e per un'eventuale scalabilità futura. 

\subsection{Confronto con Imola Informatica sul Client}
Terminata la discussione in merito alla Specifica Architetturale, il team espone all'azienda le proprie riflessioni in merito all'utilizzo di \glossario{Telegram} come unico Client con il quale gli utenti potranno utilizzare l'applicazione. \newline
Il gruppo, in seguito al colloquio avvenuto con il Prof. Riccardo Cardin, durante la presentazione \glossario{Technology Baseline}, ha effettuato uno studio, dopo il quale è arrivato alla conclusione che, per quanto \glossario{Telegram} offra molti vantaggi - come il supporto nativo multidispositivo - sia più opportuno, per uno sviluppo corretto del progetto, avere un controllo completo anche del Client dell'applicazione, cosa che non sarebbe possibile appoggiandosi ad un servizio di messaggistica già esistente. \newline
Data questa premessa il team comunica all'azienda Imola Informatica, la volontà di realizzare una WebApp per la gestione del lato Client, il cui sviluppo, a differenza di quanto mostrato durante il \glossario{Proof of Concept}, sarà sopportato da un framework. 
L'azienda apprezza la scelta supportata da una fase di studio e approva questo tipo di soluzione, suggerendo \glossario{React JS} come possibile framework da utilizzare per lo sviluppo del lato Client. 

\newpage
\subsection{Riflessioni e decisioni in seguito all'incontro}
Terminato l'incontro con l'azienda Imola Informatica il team si ritrova per fare il punto della situazione, in seguito alle decisioni prese. In particolare si decide di suddividere il lavoro in vista del prossimo incontro come segue: 
\begin{itemize}
    \item \textit{Matteo Pillon} e \textit{Pietro Macrì} si occupano di studiare \glossario{React JS} per capire come sfruttarlo per realizzare il Client dell'applicazione. 
    \item \textit{Qi Fan Andrea Pan}, \textit{Mattia Episcopo} e \textit{Samuele Rizzato} si organizzano per sistemare alcuni dettagli relativi alla Specifica Architetturale.
    \item \textit{Irene Benetazzo} e \textit{Tommaso Berlaffa} proseguono con la sistemazione e la verifica dei documenti, modificati nel corso delle settimane precedenti.
\end{itemize}