\section{Svolgimento}
Il gruppo chiede se l’azienda può fornire qualche consiglio per l’inizio dell’attività progettuale e per lo studio delle tecnologie utilizzabili.
Vengono dati tre consigli principali:
	\begin{itemize}
		\item Iniziare a studiare Python individualmente
		\item Esercitarsi su API Rest fornite dall’azienda
		\item Progettare un semplice modello di chatbot funzionante
	\end{itemize}

\subsection{Studio di Python}
L’azienda Imola dà la libera scelta sul linguaggio utilizzato per la programmazione del chatbot, ma consiglia fortemente di utilizzare il linguaggio Python, in quanto esistono già molte librerie focalizzate per il sviluppo del chatbot (ad esempio ChatterBot); inoltre Phyton è un linguaggio molto semplice da usare ed imparare.
L'azienda dà anche alcuni consigli per lo studio del linguaggio, in quanto Python è molto simile agli altri linguaggi: è possibile infatti effettuare piccoli esercizi individualmente, trovare qualche tutorial online, e studiare dal sito HTML.it. In caso di problemi l'azienda ha garantito disponibilità in qualsiasi momento. 

\subsection{API Rest fornita da Imola}
L’azienda spiega che API Rest è un contratto che permette di utilizzare i servizi su internet come se fossero in remoto, tramite comandi base del web (GET, POST …).\newline
L’azienda fornisce un API Rest costruita da loro all'indirizzo https://apibot4me.imolinfo.it; è ancora incompleta, ma dispone delle funzioni basilari con cui si possono testare i programmi.\newline
Per la creazione dell’interfaccia della documentazione API, l’azienda sfrutta Swagger, in cui utilizza JSON come formato che viene inviato tramite protocollo http. \newline
L’azienda specifica che la documentazione non è ancora completa, e possono quindi esserci degli errori di descrizione che vengono man mano corretti.\newline
Per il test si può accedere direttamente all'indirizzo fornito, oppure tramite la riga di commando utilizzando il comando \textit{curl}. \newline
Riguardo al fattore autenticazione, al momento viene fornita per tutti un'unica chiave che rappresenta il login del sistema; in futuro verranno aggiungi diversi strumenti per il riconoscimento dei diversi utenti.

\subsection{Consiglio per il primo modello e altri suggerimenti}
Per iniziare l’azienda suggerisce di progettare un semplicissimo modello in cui il chatbot dispone solo di piccolie funzioni basilari; da esempio: ``Mi chiamo Andrea'' e il bot risponde ``Ciao Andrea''. Per far ciò spiega che serve collegare il programma con l'API fornita, testando che tutto funzioni correttamente e, solo dopo, sviluppando altri comportamenti.\newline
In più suggerisce di progettare i vari casi d’uso già con diverse tipologie di utenze, in cui l’utente può decidere con quali token effettuare il login; in caso di dubbio, l’azienda fornisce la disponibilità di confronto.
Come server l’azienda dà la disponibilità a fornire un server interno, ma suggerisce di utilizzare le varie risorse gratuite aperte agli studenti, come Heroku, il quale risulta limitato ma sufficiente per lo sviluppo del bot.
