\section{Svolgimento}
\subsection{UC1.1.1 e UC1.1.2}
Il gruppo Sweven Team espone la propria idea: l'utente richiede al bot la possibilità di accedere, via 
scrittura o via vocale, e il bot risponde con un link; una volta cliccato tale link, l'utente può inserire 
il proprio token (unicamente via scrittura). \newline
Dopo un'analisi complessiva, si è giunti alla conclusione che tale processo non è necessario, grazie ad alcune 
semplificazioni introdotte dall'azienda: ora sarà sufficiente che ogni utente faccia richiesta di autenticarsi, 
con a seguito il token di autenticazione. Per praticità, l'inserimento del codice viene lasciato solo tramite 
scrittura.

\subsection{UC2.1}
L'azienda si mostra allineata con l'idea del gruppo Sweven Team. Viene lasciata dunque a quest'ultimo la 
gestione dello Use Case, poiché gli studenti che lo compongono conoscono meglio le richieste del docente.

\subsection{UC5}
L'azienda sconsiglia l'uso della geolocalizzazione (uso già non previsto da parte del gruppo). Questo perché 
la geolocalizzazione dei Personal Computer avviene tramite analisi dell'indirizzo IP, il che può portare a 
incongruenze con l'effettiva posizione dell'utente. Alessandro Proscia mostra il suo esempio a sostegno di tale 
tesi.

\subsection{UC8.3 e UC8.4}
L'azienda spiega che i ticket vengono sempre generati nello stato ``sospeso'', in quanto la generazione di un 
ticket già terminato è priva di senso.

\subsection{Altre considerazioni}
L'azienda si sofferma particolarmente sull'importanza di concentrarsi in primis sullo sviluppo della parte 
obbligatoria del capitolato, e di soffermarsi solo in un secondo momento sulla parte facoltativa.

\subsection{Post incontro}
Il gruppo Sweven Team rimane al termine dell'incontro per effettuare una rapida riunione circa i temi 
analizzati e le soluzioni proposte.

\newpage