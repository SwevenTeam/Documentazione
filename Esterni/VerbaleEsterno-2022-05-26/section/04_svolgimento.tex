\section{Svolgimento}

\subsection{Discussione sul Materiale} 
  Il gruppo si ritrova poco prima del collegamento, per decidere quale versione del Proof of Concept mostrare al referente di Imola Informatica e, più nello specifico, 
  quali quesiti porre riguardanti lo sviluppo di quest'ultimo. \\ 
  Viene affidato a \textit{Mattia Episcopo} il ruolo di espositore del codice, poiché in buona parte scritto da lui.
\subsection{Confronto con Imola Informatica}
  Conseguentemente all'entrata in chiamata via Zoom del referente di Imola Informatica, \textit{Lorenzo Patera}, inizia l'esposizione del progetto, che comprende dimostrazione su
  Client e visualizzazione del codice della parte Server. \\
  Da questa prima dimostrazione sorge il primo quesito, ovvero il corretto funzionamento delle API proposte da Imola Informatica. Il referente conferma
  la corretta esecuzione di una buona parte delle API, mentre alcune risultano avere delle problematiche interne. \\ 
  Queste problematiche verranno riportate ai collaboratori di Imola Informatica che se ne occuperanno in futuro.\\
  \textit{Lorenzo Patera} spiega quindi come interagire al meglio con le API Rest e come risolvere il problema riscontrato precedentemente con l'utilizzo delle API. \\
  La discussione si sposta quindi su due aspetti: in primis, l'utilizzo degli adapter, che viene ritenuto elegante ed apprezzabile da parte del referente, 
  in un secondo momento, una riflessione su come far interagire il bot con discussioni molto lunghe. \\
  Infatti, uno degli aspetti più complicati nella creazione di bot è quello di riuscire a memorizzare gli ultimi messaggi, in modo tale da poter comporre vere e proprie 
  conversazioni, come nel caso in cui si voglia chiedere all'utente un'insieme di elementi, uno alla volta. Il referente spiega infatti come sia sconsigliabile constringere 
  l'utente ad inviare messaggi molto lunghi e contententi multiple informazioni. \\
  Terminata la dimostrazione, il referente si mostra soddisfatto della direzione che il gruppo sta prendendo nella creazione del progetto, dicendo che ciò che è stato mostrato
  corrisponde alle aspettative del proponente.
\subsection{Riflessioni in seguito dell'incontro}
  Finita la dimostrazione con il referente, il gruppo si ritrova brevemente per fare il punto della situazione. \\
  Il gruppo si ritiene soddisfatto della conversazione avvenuta con il referente di Imola Informatica, e decide di iniziare a studiare possibili soluzioni per riuscire a farsì
  che il bot riesca a memorizzare gli ultimi messaggi. \\
  Viene inoltre discussa brevemente la divisione oraria dei compiti, che verrà discussa più in dettaglio nella riunione di Lunedì 30 Maggio.