\section{Conclusioni}
In conclusione, dopo aver esposto lo stato di avanzamento dell'applicativo e dopo la discussione sui punti da migliorare, Imola Informatica ritiene il lavoro svolto fin'ora soddisfacente e l'applicazione nel corretto stato di avanzamento. Dopo le spiegazioni dettagliate fornite da Imola Informatica, il gruppo prende insieme ai proponenti le decisioni riguardo le migliorie da apportare. Per l'identificazione univoca di ogni client che si collega al server, il gruppo decide che andrà ad approfondire l'utilizzo di UUID, cercando di mantenere più client attivi in contemporanea; invece, per il salvataggio dell'api-key lato client il gruppo opta per approfondire l'utilizzo dei local storage del browser. 
\newpage

\section*{Tracciamento delle decisioni}
	\renewcommand{\arraystretch}{1.8} %aumento ampiezza righe
	\begin{tabular}{ |c|l| }
		\hline
		\textbf{Codice} & \textbf{Descrizione} \\
		\hline
		VE\_2022-08-17.2 & Scelto local storage per salvataggio api-key sul client\\
		\hline
		VE\_2022-08-17.1 & Scelto uuid per gestione client multipli \\
		\hline
	\end{tabular}