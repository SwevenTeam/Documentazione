\section{Svolgimento}
\subsection{Presentazione dello stato complessivo dell'applicativo a Imola Informatica}
Il gruppo inizia esponendo la versione attuale dell'applicativo. In particolare \textit{Tommaso Berlaffa} spiega come è stata realizzata la parte server e dà dimostrazione del suo funzionamento, in seguito \textit{Matteo Pillon} continua esponendo la parte client, sia nella realizzazione che nel funzionamento.
Durante le dimostrazioni, la discussione si incentra in particolare su alcuni punti cruciali che ancora sono pendenti per avere una versione stabile. In primis il gruppo chiede come gestire al meglio l'autenticazione dell'utente mediante api-key. Oltre a questo viene sollevato da Imola il problema della gestione di client multipli che potrebbero accedere all'applicativo e utilizzarlo contemporaneamente. Dopo di che viene fatto un check sulle stringhe che l'applicativo accetta, cioè con quali parole chiave si può accedere alle varie funzonalità del bot, da questo punto di vista i rappresentati di Imola Informatica si dicono soddisfatti di come è stato gestito questo punto. Infine vengono poste alcune domande dal gruppo, in merito ad alcune funzionalità che Imola si aspetta dal chatbot. In particolare viene chiesto se l'utente oltre a consuntivare, possa anche vedere quello che ha fin'ora consuntivato e quale valore debba aver un campo ("status") dell'api per la richiesta di apertura del cancello. Per la prima domanda Imola risponde che sarebbe una funzionalità di supporto alla consuntivazione gradita, per quanto riguarda la seconda, la risposta è che al momento viene data libertà di scelta al gruppo per il valore del campo siccome non ha una funzionalità ben precisa.
\subsection{Confronto con Imola Informatica sui punti da migliorare}
I punti su cui la presentazione si è soffermata e che vanno approfonditi sono come gestire la comunicazione del server con più client e come salvare l'api-key lato client. Per quanto riguarda la comunicazione con più client Imola Informatica propone varie soluzioni, tra le quali sticky session e uuid (Universally Unique Identifier). Per il salvataggio dell'api-key lato client invece Imola Informatica propone e spiega nel dettaglio varie tecniche, come cookies, local storage del browser e session storage del browser.