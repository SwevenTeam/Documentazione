\section{Tecnologie}
\subsection{Server}
\subsubsection{Python}
Linguaggio di programmazione ad alto livello, adatto alla programmazione orientata agli oggetti. E' stato utilizzato per sviluppare il back-end insieme alla libreria esterna Chatterbot.
\paragraph{Chatterbot}
Libreria esterna in \glossario{Python} che utilizza algoritmi di intelligenza artificiale per trovare la migliore risposta per emulare il comportamento di un \glossario{chatbot} nel server. \newline
Grazie alla sua flessibilità si sono implementati degli adapter che modellano e gestiscono le varie richieste dell'utente.
\subsubsection{API Rest Imola Informatica}
L'azienda ha fornito delle \glossario{API Rest} che permettono al \glossario{chatbot} di interagire con i loro sistemi aziendali.

\subsection{Client}
\subsubsection{React}
React è una libreria JavaScript per costruire l'interfaccia utente caratterizzata dal fatto che è dichiarativa, efficiente e flessibile.
\subsubsection{HTML}
Linguaggio di markup, in standard W3C, per documenti visualizzabili attraverso un web browser
\subsubsection{CSS}
Linguaggio di formattazione per i documenti HTML.
\subsubsection{Flask}
Framework Python per lo sviluppo di applicazioni web. Il modulo Python Flask contiene tutte le classi e le funzioni necessarie per la costruzione di una web app.
\subsubsection{API AssemblyAI}
L'API deve essere integrata con React e permette di tradurre automaticamente l'audio in testo.

\newpage