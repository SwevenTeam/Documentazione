\section{Tecnologie}
\subsection{API Rest}
Un'API REST è un'interfaccia di programmazione delle applicazioni conforme ai vincoli dello stile architetturale REST, che consente l'interazione con servizi web RESTful.\newline 
Il termine REST è l'acronimo di REpresentational State Transfer. REST è un insieme di vincoli architetturali, non un protocollo né uno standard.
Quando una richiesta client viene inviata tramite un'API RESTful, questa trasferisce al richiedente o all'endpoint uno stato rappresentativo della risorsa. L'informazione viene consegnata in HTTP in un formato JSON, HTML, Python o txt.

\subsection{Server}
\subsubsection{Python}
Linguaggio di programmazione ad alto livello, adatto alla programmazione orientata agli oggetti. E' stato utilizzato per sviluppare il back-end insieme alla libreria esterna Chatterbot.
\paragraph{Chatterbot}
Libreria esterna in \glossario{Python} che utilizza algoritmi di intelligenza artificiale per trovare la migliore risposta per emulare il comportamento di un \glossario{chatbot} nel server. \newline
Grazie alla sua flessibilità si sono implementati degli adapter che modellano e gestiscono le varie richieste dell'utente. \newline
Durante l'esecuzione Chatterbot crea un adapter che gli consente di connettersi ad un database \glossario{SQLite}.
\subsubsection{Flask}
Framework Python per lo sviluppo di applicazioni web. Flask contiene tutte le classi e le funzioni necessarie per la costruzione di una web app, e ha agevolato l'organizzazione e la gestione del \glossario{chatbot}.
\subsubsection{Cors}
Cross-Origin Resource Sharing regola la cooperazione tra sito web e il caricamento dati da un server esterno garantendone la sicurezza attraverso un'intestazione \glossario{HTTP}.
\subsubsection{UUID}
Universally Unique IDentifier, cioè identificativo univoco universale, è usato nelle infrastrutture software per avere l'unicità pratica senza garanzia in un sistema distribuito semplificando il mantenimento dell'identità degli oggetti in ambienti disconnessi.
\subsection{Client}
\subsubsection{React}
React è una libreria JavaScript per costruire l'interfaccia utente caratterizzata dal fatto che è dichiarativa, efficiente e flessibile. E' stato utilizzato per creare l'applicazione lato client.
\subsubsection{HTML}
Linguaggio di markup, in standard W3C, per documenti visualizzabili attraverso un web browser.
\subsubsection{CSS}
Linguaggio di formattazione per i documenti HTML.
\subsubsection{Regex}
Libreria esterna che controlla le espressioni regolari per la validazione dell'\glossario{api-key}. Viene infatti definito uno schema secondo il quale un'\glossario{api-key} per poter essere valida deve validare uno schema di cifre. 
\subsubsection{API AssemblyAI}
Servizio esterno che permette di tradurre automaticamente file audio in testo richiamando l'API Speech-to-text. 
\newpage