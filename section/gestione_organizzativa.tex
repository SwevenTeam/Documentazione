\subsection{Gestione Organizzativa}
\subsubsection{Scopo}
Lo scopo di questa parte è quello di fornire un insieme organizzato di attività al fine di:
\begin{itemize}
    \item fare proprio un modello organizzativo per il tracciamento dei rischi;
    \item stabilire ruoli per pianificare il lavoro e rispettare le scadenze;
    \item scegliere gli strumenti per le comunicazioni interne e esterne;
    \item decidere un modello di sviluppo.
\end{itemize}

\subsubsection{Descrizione}
Di seguito vengono riportati gli argomenti delle attività di organizzazione:
\begin{itemize}
    \item definizione dei ruoli e dei compiti assegnati ai componenti del gruppo;
    \item modalità di esecuzione delle attività;
    \item esame dei progressi delle attività.
\end{itemize}

\subsubsection{Aspettative}
Le aspettative per questa parte sono:
\begin{itemize}
    \item ottenere un piano di schemi da seguire;
    \item definire i ruoli all'interno del gruppo;
    \item agevolare le comunicazioni interne e esterne al gruppo;
    \item controllare il progetto e le attività del gruppo.
\end{itemize}

\subsubsection{Ruoli di Progetto}
I ruoli del progetto saranno ricoperti da ogni membro del gruppo in rotazione per permettere
una equa distribuzione delle mansioni da svolgere. L'organizzazione delle attività
è esposta nel \emph{Piano di Progetto} e deve essere seguita dalle varie figure progettuali, ovvero:

\paragraph{Responsabile} \hfill \break
Il responsabile è quella figura che si occupa della parte di coordinamento, di pianificazione del progetto 
e di mediazione con i soggetti esterni al gruppo. Si occupa di:
\begin{itemize}
    \item controllare le attività del team;
    \item pianficare le attività del team;
    \item coordinare i membri del gruppo;
    \item approvare i documenti;
    \item gestire le relazioni esterne.
\end{itemize}

\paragraph{Amministratore} \hfill \break
L'amministratore è colui che gestisce l'ambiente di lavoro all'interno del gruppo; le attività che svolge sono:
\begin{itemize}
    \item controllare le infrastrutture di supporto;
    \item documentare le regole e gli strumenti utilizzati;
    \item attuare le scelte tecnologiche fissate dal gruppo;
    \item controllare le configurazioni e le versioni.
\end{itemize}

\paragraph{Analista} \hfill \break
L'analista ha il dovere di identificare e comprendere il dominio del problema per consentire, successivamente,
una corretta progettazione. I suoi compiti sono:
\begin{itemize}
    \item analizzare il dominio del problema;
    \item scrivere l'\emph{Analisi dei Requisiti}.
\end{itemize}
L'analista partecipa al progetto fino a quando non si conclude l'analisi del problema.

\paragraph{Progettista} \hfill \break
Il progettista concorre alla ricerca di una soluzione per il prodotto e alle scelte tecniche e tecnologiche.
Partecipa allo sviluppo software ma non alla manutenzione. Le sue attività sono:
\begin{itemize}
    \item progettare l'architettura dell'applicativo in modo che sia mantenibile e affidabile;
    \item trovare soluzioni efficienti ai problemi tecnici e tecnologici del progetto;
    \item controllare la fase di sviluppo.
\end{itemize}

\paragraph{Programmatore} \hfill \break
Il programmatore copre la parte di codifica del progetto usando le soluzioni e tecnologie stabilite dal team, inoltre
spetta a questa figura la scrittura dei test per la validazione. Ricapitolando si occupa di:
\begin{itemize}
    \item scrivere il codice che implementi le soluzioni trovate dal progettista;
    \item realizzare i test per la verifica e validazione del software;
    \item stilare il \emph{Manuale Utente}.
\end{itemize}

\paragraph{Verificatore} \hfill \break
Il verificatore è tenuto a esaminare i progressi del lavoro compiuto dagli altri membri del gruppo.   
L'attività di verifica viene condotta sul codice e sui documenti col fine di far rispettare le \emph{Norme di Progetto}.

\subsubsection{Gestione delle Riunioni}
\paragraph{Riunioni Interne} \hfill \break
Le riunioni interne avvengono su \glossario{Zoom}: il link per collegarsi viene inviato, generalmente, 
qualche minuto prima dell'inizio. Ad ogni riunione i membri del gruppo si aggiornano sul lavoro svolto, discutono dei problemi e/o dubbi incontrati 
e, a seguire, stabiliscono le attività da svolgere per il prossimo incontro. Tramite un foglio Excel in cui segnare i propri impegni, 
i componenti del team hanno scelto un giorno fisso in cui effettuare la riunione settimanale.

\paragraph{Riunioni Esterne} \hfill \break
Le riunioni esterne avvengono tramite \glossario{Zoom} sia con il proponente che con il committente. Prima di ogni incontro con 
l'azienda Imola, vengono inviate delle mail per concordare la data e le coordinate della riunione. Generalmente si scelgono
degli argomenti su cui verterà la riunione e successivamente vengono anticipati all'azienda tramite \glossario{Telegram} o email.

\paragraph{Verbali} \hfill \break
In tutte le riunioni viene stabilito un Redattore, il quale si occupa di riassumere tutto ciò che
viene detto durante il meeting, e un Verificatore, il quale si assicura che non vi siano errori; 
il ruolo di Approvatore, cui spetta il compito di confermare o respingere - per ulteriori modifiche - il 
documento è assegnato in automatico all'attuale Responsabile della fase. La struttura del verbale viene 
largamente approfondita nei processi di supporto al punto \$3.1.5.2.

\subsubsection{Metriche}
\textbf{M15VC}:
\begin{itemize}
    \item Nome: Variazione di costo;
    \item Descrizione: numero che indica di quanto si è sopra o sotto al budget preventivato:
         \begin{itemize}
            \item se è > 0 indica che si è sotto al budget;
            \item se è = 0 indica che si sta spendendo esattamente il budget;
            \item se è < 0 indica che si è sopra al budget.
        \end{itemize}
    \item Scopo: controllare quanto si sta spendendo al fine di rispettare il preventivo;
    \item Formula: $ CP - CE $
        \begin{itemize}
            \item CP: indica il costo pianificato per svolgere le attività del progetto in un determinato 
            periodo;
            \item CE: indica il costo effettivo per svolgere le attività del progetto in un determinato 
            periodo.
        \end{itemize}
\end{itemize}

\textbf{M16VP}:
\begin{itemize}
    \item Nome: Variazione di piano;
    \item Descrizione: numero che indica di quanti giorni il lavoro è avanti o indietro rispetto alla 
    pianificazione:
         \begin{itemize}
            \item se è > 0 si è in anticipo rispetto alla schedule;
            \item se è = 0 si è rispettata la schedule;
            \item se è < 0 si è in ritardo rispetto alla schedule.
        \end{itemize}
    \item Scopo: controllare quanto tempo si sta impiegando al fine di rispettare la pianificazione;
    \item Formula: $ GP - GC $
        \begin{itemize}
            \item GP: indica i giorni pianificati per completare il lavoro;
            \item GC: indica i giorni consuntivati per completare il lavoro.
        \end{itemize}
\end{itemize}