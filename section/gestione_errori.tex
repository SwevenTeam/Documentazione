\subsection{Gestione degli Errori}
Questa sezione ha lo scopo di gestire le modalità di approccio e di soluzione ai problemi 
che possono verificarsi nell'arco dello svolgimento dell'intero progetto.
Per evitare perdite di tempo eccessive a causa di vari ed eventuali errori, serve limitare 
al massimo due fattori: il ripetersi del medesimo errore e il propagarsi dell'errore su altre 
entità. \newline
Il percorso per il controllo e l'eleminazione degli errori è il seguente:
\begin{enumerate}
    \item Individuazione;
    \item Analisi possibili conseguenze su entità esterne;
    \item Identificazione;
    \item Crezione possibili soluzioni;
    \item Scelta soluzione;
    \item Verifica soluzione scelta:
    \begin{itemize}
        \item In caso di verifica con esito positivo, procedere dal punto 7;
        \item In caso di verifica con esito negativo, riprendere dal punto 5.
    \end{itemize}
    \item Modifica del sistema di verifica per evitare che l'errore si presenti nuovamente.
\end{enumerate}
Seguono approfondimenti su alcune delle parti sopraelencate. 
\subsubsection{Individuazione}
Non appena viene individuato un problema, si contattano nell'immediato le persone coinvolte direttamente o 
indirettamente con esso. Si trovano quindi le possibili cause.
\subsubsection{Identificazione}
Al problema viene assegnato un codice univoco che viene successivamente memorizzato. \newline 
Il codice ha il seguente formato: 
\begin{center}
    \textbf{PROB [PRIORITY] - [TYPE] - [NUMBER]}
\end{center}
\renewcommand{\arraystretch}{1.8} %aumento ampiezza righe
    \begin{tabular}{ |m{7em}|m{30em}| }
        \hline
        \textbf{Nome} & \textbf{Descrizione} \\
        \hline
            PROB & Indica un problema \\
        \hline
            PRIORITY 	& 	Indica il tipo di priorità: \\
                        &	\textbf{MAX}: Priorità totale, problema urgente \\
                        &	\textbf{MED}: Priorità media, problema non urgente ma importante \\
                        &	\textbf{MIN}: Priorità bassa, problema non urgente e poco importante \\
        \hline
                    
            TYPE 	& 	Indica il tipo di problema: \\
                    & 	\textbf{G}: Problema di grammatica (ortografico, sintattico, etc.) \\
                    &	\textbf{C}: Problema di contenuto (dati mancanti o errati) \\
                    &	\textbf{F}: Problema di funzionalità (qualcosa non funziona) \\
        \hline
            NUMBER & Codice numerico identificativo \\
        \hline
    \end{tabular} \newline \newline
La risoluzione di una serie di problemi va eseguita prima secondo l'ordine di priorità e poi secondo l'ordine degli identificativi. \newline 
La priorità viene calcolata in base alla gestione dei tempi e in base all'impatto che il problema ha sul resto del progetto, il che comprende l'eventuale propagazione valutata nel punto 2. del processo.
\subsubsection{Modifica del Sistema di Verifica}
Il codice del problema viene aggiunto, insieme ad una breve descrizione e alla soluzione, nel sistema di gestione delle issues offerto da GitHub accennato in precedenza.
In questo modo si riesce a tenere meglio sotto controllo l'evolversi della situazione.