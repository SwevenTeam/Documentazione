\subsection{Client}
\subsubsection{Introduzione}
Il \glossario{Client} è rappresentato dall'interfaccia web, con la quale l'utente usufruisce dei servizi offerti dall'applicativo. Esso è \glossario{Stateless} cioè non contiene lo stato attuale della conversione, questa informazione viene gestita e salvata solamente lato \glossario{Server}. \newline
Il \glossario{Client} ad ogni avvio manda una richiesta \glossario{POST} all'indirizzo del server \textit{/getID} richiendo di ricevere un identificativo univoco. Una volta ricevuta la risposta dal server contentente l'ID, esso viene salvato all'interno del \glossario{LocalStorage} del browser, in questo modo ogni messaggio che viene mandato dal \glossario{Client} al \glossario{Server} avrà all'interno dei parametri della \glossario{POST} il \glossario{clientID} che permetterà al chatbot di ricollegare la conversazione allo specifico client. \newline
L'aggiornamento della pagina del browser comporta l'assegnazione di un nuovo \glossario{clientID} il che implica lo sviluppo di uno dei seguenti scenari:
\begin{itemize}
    \item utente precedentemente loggato: la sua \glossario{API-KEY} viene reperita dal \glossario{LocalStorage} del browser, dove era stata memorizzata. Verrà quindi chiesto all'utente se effetturae il login nuovamente con tale \glossario{API-KEY} o eventualmente inserirne una diversa.
    \item utente precedentemente non loggato: non avendo effettuato il login in precedenza, l'applicazione ripartirà chiedendo all'utente di inserire un \glossario{API-KEY} per poter utilizzare i servizi offerti. 
\end{itemize}

\subsubsection{Componenti}
Il \glossario{Client} è stato sviluppando utilizzando \glossario{React} che per struttura invita ad utilizzare dei componenti, i quali aiutano a suddividere e riutilizzare il codice. I componenti utilizzati dalla nostra applicazione sono:
\begin{itemize}
    \item \textbf{AudioRecorder}: componente dedicato alla registrazione e invio del file audio al servizio esterno. Quest'ultimo si occupa della conversione del file audio in un testo, che viene restituito come stringa pronta da inviare al server.
    \item \textbf{CustomButton}: per questioni di manutenibilità e futura espansione si è deciso di realizzare un componente dedicato ai bottoni presenti all'interno dell'applicativo. 
    \item \textbf{CustomIcon}:  per questioni di manutenibilità e futura espansione si è deciso di realizzare un componente dedicato alle icone presenti all'interno dell'applicativo. 
    \item \textbf{Home}: componente principale dell'applicativo lato \glossario{Client}, racchiude al suo interno gli altri componenti e rappresenta l'UI con la quale l'utente si interfaccia per utilizzare i servizi. 
    \item \textbf{LoadingSpinner}: componente dedicato che avvisa l'utente sullo stato di avanzamento del processo di conversione del file audio. 
\end{itemize}
\subsubsection{Dipendenze esterne}
Lato client possiamo trovare le seguenti dipendenze esterne:
\begin{itemize}
    \item \textbf{AssemblyAI}: servizio esterno che si occupa della conversione dei file audio registrati dall'utente, ritornando un testo sotto forma di stringa. 
\end{itemize}
\newpage

