\section{Architettura}
L'architettura del prodotto è suddivisa tra Client e Server, inoltre si utilizzano le \glossario{API Rest} messe a disposizione dall'azienda Imola Informatica.
---IMG unica, mentre se è separata Client e Server vanno messe dentro relativa sezione ---
\subsection{Client}
\subsection{Server}
\subsubsection{Chatterbot} classe della libreria esterna scritta in \glossario{Python}
\subsubsection{State} classe che è la generalizzazione di tutti i vari stati e 
come dato privato si salva l'attuale stato corrente e pubblicamente dispone anche di un metodo per aggiungere 
informazioni necessarie per completare la richiesta in corso.
\paragraph*{State\_Null} sottoclasse aggiuntiva di \textit{State} che comprende il caso di uno stato vuoto, una possibile causa può essere quando l'utente non è autenticato.
\subsubsection{Statement\_State} classe che fa da intermediario tra lo State e lo Statement. Come unico campo privato ha l'\glossario{Api-Key} che certifica se l'utente è autenticato e la rende visibile pubblicamente tramite una funzione get.
\subsubsection{Statement} classe che viene fornita come dipendenza della libreria \glossario{Chatterbot} e ha il compito di memorizzare i dati e viaggia tra il Client, \textit{LogicAdapter} e Server.
\subsubsection{LogicAdapter} classe astratta base per tutte le classi adapter derivate, una per ogni funzionalità, dispone dei due metodi base di cui verrà fatto l'\glossario{overriding}:
    \begin{itemize}
        \item can\_process: metodo booleano che controlla tutte le varie condizioni e se tutto okay fa procedere il metodo \textit{process}.
        \item process: controlla ed elabora tutti i dati forniti così da produrre una risposta.
    \end{itemize}
\subsubsection{Request} generalizzazione di tutte le varie \textit{request} distinte per funzionalità. Interfaccia che riceve i dati pronti verificandone la completezza e in base all'\textit{adapter} invia la richiesta \glossario{HTTP} alle \glossario{API Rest} di Imola per interagire con i loro servizi e soddisfare la richiesta dell'utente.
\subsubsection{Login}
\subsubsection{Logout}
\subsubsection{Activity} classi \textit{Adapter\_Activity}, \textit{State\_Activity} e \textit{Request\_Activity} per la funzionalità di consuntivare le ore dedicate ad un progetto compreso le eventuali ore di viaggio.
\subsubsection{Gate} classi \textit{Adapter\_Gate}, \textit{State\_Gate} e \textit{Request\_Gate} per la funzionalità di apertura cancello
\subsubsection{Project\_Creation} classi \textit{Adapter\_Project\_Creation}, \textit{State\_Project\_Creation} e \textit{Request\_Project\_Creation}
\subsubsection{Presence} classi \textit{Adapter\_Presence}, \textit{State\_Presence} e \textit{Request\_Presence} per la funzionalità di registrazione della presenza

\subsection{API Rest}
Un'API REST è un'interfaccia di programmazione delle applicazioni conforme ai vincoli dello stile architetturale REST, che consente l'interazione con servizi web RESTful.\newline 
Il termine REST è l'acronimo di REpresentational State Transfer. REST è un insieme di vincoli architetturali, non un protocollo né uno standard.
Quando una richiesta client viene inviata tramite un'API RESTful, questa trasferisce al richiedente o all'endpoint uno stato rappresentativo della risorsa. L'informazione viene consegnata in HTTP in un formato JSON, HTML, Python o txt.
\newpage