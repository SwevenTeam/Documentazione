\subsection{Validazione}
Il processo di validazione è il processo successivo alla verifica. Quando gli sviluppatori e i verificatori 
si ritengono soddisfatti del prodotto finale e si considera il prodotto pronto alla consegna, viene richiesta 
una riunione del gruppo, insieme al responsabile, nella quale vengono rivalutate tutte le aspettative e le 
richieste del proponente. Nel caso questa fase abbia esito positivo, il prodotto in questione viene approvato 
e rilasciato come prodotto finale.
\subsubsection{Attività}
La validazione è composta da due fasi:
\begin{itemize}
\item \textbf{Test di accettazione}: vengono rieffettuati tutti i test, assicurandosi che vengano soddisfatti 
i requisiti necessari;
\item \textbf{Collaudo}: insieme al proponente viene effettuata un riunione dove viene illustrato il corretto 
funzionamento dei requisiti richiesti. 
\end{itemize}

\subsubsection{Strumenti}
Non sono stati rilevati particolari strumenti per il processo di Validazione.
\subsubsection{Metriche}
\textbf{M19FLC}:
  \begin{itemize}
    \item Nome: Frontend Line Coverage;
    \item Descrizione: Linee di codice testate sulla parte di Frontend rispetto al totale;
    \item Scopo: confermare tramite valore l'avvenuto testing di una parte del progetto;
    \item Formula:
    \begin{center}
    $ \frac{\textit{\#Linee di Codice Eseguibile Testate}}{\textit{\#Linee di Codice Eseguibile}}$
    \end{center}
  \end{itemize}
\textbf{M20FBC}:
  \begin{itemize}
    \item Nome: Frontend Branch Coverage;
    \item Descrizione: Rami di codice testate sulla parte di Frontend rispetto al totale;
    \item Scopo: confermare tramite valore l'avvenuto percorrimento da parte del testing di tutti i rami del progetto;
    \item Formula:
    \begin{center}
    $ \frac{\textit{\#Rami possibili di Codice Testati}}{\textit{\#Rami possibili di Codice}}$
    \end{center}
  \end{itemize}
\textbf{M21BLC}:
  \begin{itemize}
    \item Nome: Backend Line Coverage;
    \item Descrizione: Linee di codice testate sulla parte di Backend rispetto al totale;
    \item Scopo: confermare tramite valore l'avvenuto testing di una parte del progetto;
    \item Formula:
    \begin{center}
    $ \frac{\textit{\#Linee di Codice Eseguibile Testate}}{\textit{\#Linee di Codice Eseguibile}}$
    \end{center}
  \end{itemize}
\textbf{M22BBC}:
  \begin{itemize}
    \item Nome: Backend Branch Coverage;
    \item Descrizione: Rami di codice testate sulla parte di Backend rispetto al totale;
    \item Scopo: confermare tramite valore l'avvenuto percorrimento da parte del testing di tutti i rami del progetto;
    \item Formula:
    \begin{center}
    $ \frac{\textit{\#Rami possibili di Codice Testati}}{\textit{\#Rami possibili di Codice}}$
    \end{center}
  \end{itemize}



