\subsection{Scopo del documento}
Lo scopo di questo documento è di redigere tutte le norme, regole, convenzioni, decisioni 
prese dai componenti del gruppo Sweven Team, al fine di migliorare la qualità e l'organizzazione
per il lavoro del gruppo e individuale. \newline
La stesura del documento è incrementale durante il progetto così da riportare le decisioni
prese durante le riunioni. Inoltre le norme possono subire modifiche e aggiornamenti,
se ciò avviene sarà premura del responsabile che ha dato l'approvazione alla modifica 
avvisare tutti i componenti di prenderne visione al più presto. \newline 
Tutti i membri del gruppo sono tenuti a rispettare le norme di progetto che costituiscono
il way of working del gruppo Sweven Team.

\subsection{Scopo del capitolato}
Lo scopo di tale progetto è quello di sviluppare un Chatbot che interfacciandosi con software aziendali spesso complessi e dispersivi, semplifichi i compiti che i dipendenti devono svolgere. In particolare vengono individuate le seguenti operazioni: 
\begin{itemize}
	\item Tracciamento della presenza in sede (\textbf{EMT}\textsubscript{G})
	\item Rendiconto attività svolte quotidianamente (\textbf{EMT}\textsubscript{G})
	\item Apertura del cancello aziendale (\textbf{MQTT}\textsubscript{G})
	\item Creazione di una riunione in un servizio esterno
	\item Servizio di ricerca documentale (\textbf{CMIS}\textsubscript{G})
	\item Creazione e tracciamento di bug (\textbf{Redmine}\textsubscript{G})
\end{itemize}

\subsection{Glossario}
Per assicurare la massima fruibilità e leggibilità del documento, il team SWEven ha deciso di creare un documento denominato \textit{Glossario} il cui scopo sarà quello di contenere le definizioni dei termini ambigui o specifici del progetto. Sarà possibile riconoscere i termini presenti al suo interno in quanto terminanti con la lettera \textit{G} posta come pedice della parola stessa. 
\subsection{Riferimenti}

\subsubsection{Normativi}
\begin{itemize}
	\item standard ISO/IEC 12207
\end{itemize}

\subsubsection{Informativi}
\begin{itemize}
	\item \href{https://www.math.unipd.it/~tullio/IS-1/2021/Progetto/C1.pdf}{\color{blue} Capitolato di appalto C1 - BOT4ME}   (non so se sia corretto)
\end{itemize}
\newpage