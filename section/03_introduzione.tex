\section{Introduzione}
    Il Piano di Qualifica permette di raggruppare e ordinare le diverse modalità tramite le quali vengono effettuate le operazioni di verifica e di validazione necessarie per lo svolgimento corretto del progetto.
	\subsection{Obiettivi del capitolato scelto}
	    Il chatbot da sviluppare deve essere in grado di facilitare l'utente in una serie di operazioni che egli deve effettuare. \newline
	    L'azienda di Imola Informatica ha fornito la lista delle operazioni richieste per Bot4Me. Dalle slide e dal video di presentazione si evincono infatti le seguenti funzioni:
	    \begin{itemize}
	        \item Consultivare le attività giornaliere\textsuperscript{1}
	        \item Tracciare le presenze in sede\textsuperscript{1}
	        \item Aprire il cancello (tramite protocollo MQTT)
	        \item Creare una nuova riunione (Teams / Zoom / Google meet)
	        \item Ricercare documenti (integrazione via CMIS)
	        \item Creare un ticket (su piattaforma Redmine)
	    \end{itemize}
	    Le attività contrassegnate con il numero 1 sono quelle obbligatorie \newline
	    Le attività contrassegnate con il numero 2 sono quelle scelte dal gruppo tra le facoltative
	\newpage