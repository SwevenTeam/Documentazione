\section{Introduzione}
\subsection{Scopo del Documento}
Il Manuale Utente ha lo scopo di descrivere il funzionamento del \glossario{Chatbot} e fornire le istruzioni e i consigli per l'utilizzo.

\subsection{Scopo del Capitolato}
Lo scopo di tale progetto è quello di sviluppare un Chatbot che interfacciandosi con software aziendali spesso complessi e dispersivi, semplifichi i compiti che i dipendenti devono svolgere. In particolare vengono individuate le seguenti operazioni: 
\begin{itemize}
	\item Tracciamento della presenza in sede (\textbf{EMT}\textsubscript{G})
	\item Rendiconto attività svolte quotidianamente (\textbf{EMT}\textsubscript{G})
	\item Apertura del cancello aziendale (\textbf{MQTT}\textsubscript{G})
	\item Creazione di una riunione in un servizio esterno
	\item Servizio di ricerca documentale (\textbf{CMIS}\textsubscript{G})
	\item Creazione e tracciamento di bug (\textbf{Redmine}\textsubscript{G})
\end{itemize}

\subsection{Glossario}
Per assicurare la massima fruibilità e leggibilità del documento, il team SWEven ha deciso di creare un documento denominato \textit{Glossario} il cui scopo sarà quello di contenere le definizioni dei termini ambigui o specifici del progetto. Sarà possibile riconoscere i termini presenti al suo interno in quanto terminanti con la lettera \textit{G} posta come pedice della parola stessa. 
\subsection{Riferimenti}
\subsubsection{Normativi}
\begin{itemize}
    \item Norme di Progetto \docVersionNdP
\end{itemize}
\subsubsection{Informativi}
\begin{itemize}
    \item \href{https://www.math.unipd.it/~tullio/IS-1/2021/Progetto/C1.pdf}{\color{blue} Capitolato di appalto C1 - BOT4ME}
    \item Specifica Architetturale \docVersionSA
\end{itemize}
\newpage