\section{Requisiti di Sistema}
\subsection{Requisiti hardware minimi}
Per l'applicazione è necessario avere un computer che soddisfi le seguenti specifiche minime:
\begin{itemize}
    \item \textbf{Sistema Operativo:} Windows 10 (64bit), Ubuntu 20.04, MacOs 10.13 High Sierra
    \item \textbf{memoria (RAM):} 2GB
    \item \textbf{connessione internet:} attiva
\end{itemize}
\subsection{Requisiti minimi Web Application}
L'applicazione è accessibile tramite i browser:
\begin{itemize}
    \item \textbf{Google Chrome} a partire dalla versione 75
    \item \textbf{Microsoft Edge} a partire dalla versione 42
    \item \textbf{Mozilla Firefox} a partire dalla versione 69
    \item \textbf{Safari} a partire dalla versione 13
\end{itemize}

\newpage
\section{Inizio}
All'avvio il chatbot vi da il benvenuto con un messaggio suggerendo anche delle possibili funzionalità.
---IMG--- \newline
Inizialmente non si è loggati e il bot risponde soltanto al saluto e per tutto il resto ricorda la necessità di autenticarsi.
---IMG--- \newline
\subsection{Presentazione grafica e pulsanti}
\begin{itemize}
    \item IMG \textbf{Login / Logout:} in alto a destra il pulsante blu attualmente è solo per il login, dopo aver effettuato il login diventerà rosso per indicare il logout;
    \item IMG \textbf{Annulla:} in basso a sinistra il pulsante arancione serve per annullare la richiesta dell'operazione in corso, ritornando all'inizio. Non è possibile annullare un'operazione conclusa;
    \item IMG \textbf{Elimina:} in basso a sinistra il pulsante rosso serve per cancellare ciò che si sta scrivendo, non è possibile premere il pulsante se la casella di testo è vuota;
    \item IMG \textbf{Invio:} in basso a destra il pulsante azzurro serve per inviare il messaggio scritto o alternativamente si può premere l'invio della tastiera;
    \item IMG \textbf{Vocale:} in basso a destra il pulsante verde serve per registrare vocalmente il messaggio che verrà automaticamente trascritto, con la possibilità di modificarlo manualmente nella casella di testo, e poi basterà cliccare sul pulsante azzurro o l'invio della tastiera; 
    \item IMG \textbf{Salva:} in basso a destra il pulsante azzurro con il simbolo di salvataggio, che prende il posto del pulsante d'invio e vocale, permette di salvare la propria \glossario{api-key} così da rimanere autenticati.
\end{itemize}
\subsection{Autenticazione}
Per autenticarsi si può scrivere il messaggio "login" oppure cliccare sul pulsante Login in alto a destra. Verrà chiesto di inserire la propria \glossario{API-KEY} di autenticazione, se errata verrà comunicato che è fallita perchè non valida e si può riprovare; se corretta comunica che l'autenticazione è avvenuta con successo. 
Dopo l'autenticazione il pulsante in alto a destra diventerà di colore rosso con il simbolo di logout.
Se si ricarica la pagina web sembra che si abbia fatto il logout, in realtà \glossario{api-key} è salvata e basta cliccare sul pulsante azzurro in basso a destra con il simbolo di salvataggio.
---IMG--- \newline

\newpage