\section{Requisiti di Sistema}
\subsection{Requisiti hardware minimi}
\subsection{Requisiti minimi Web Application}

\newpage
\section{Inizio}
All'avvio il chatbot vi da il benvenuto con un messaggio suggerendo anche delle possibili funzionalità.
---IMG--- \newline
Inizialmente non si è loggati e il bot risponde soltanto al saluto e per tutto il resto ricorda la necessità di autenticarsi.
---IMG--- \newline
\subsection{Presentazione grafica e pulsanti}
\begin{itemize}
    \item IMG \textbf{Impostazioni:} in alto a destra il pulsante blu attualmente è solo per il login ma in futuro potrà essere implementato con le impostazioni;
    \item IMG \textbf{Annulla:} in basso a sinistra il pulsante arancione serve per annullare la richiesta / operazione in corso, ritornando all'inizio. Non è possibile annullare un'operazione conclusa;
    \item IMG \textbf{Elimina:} in basso a sinistra il pulsante rosso serve per cancellare ciò che si sta scrivendo, non è possibile premere il pulsante se la casella di testo è vuota;
    \item IMG \textbf{Elimina:} in basso a destra il pulsante azzurro serve per inviare il messaggio scritto o alternativamente si può premere l'invio della tastiera;
    \item IMG \textbf{Elimina:} in basso a destra il pulsante verde serve per registrare vocalmente il messaggio che verrà automaticamente trascritto, con la possibilità di modificarlo manualmente nella casella di testo, e poi basterà cliccare sul pulsante azzurro o l'invio della tastiera; 
\end{itemize}
\subsection{Autenticazione}
Per autenticarsi si può scrivere il messaggio "login" oppure cliccare sul pulsante Impostazioni in alto a destra. Verrà chiesto di inserire la propria \glossario{API-KEY} di autenticazione, se errata verrà comunicato che è fallita perchè non valida e si può riprovare; se corretta comunica che l'autenticazione è avvenuta con successo. ---IMG--- \newline

\newpage