\subsection{Gestione dell'infrastruttura}
\subsubsection{Scopo}
Lo scopo di questa sezione del documento è quello di definire gli strumenti con cui vengono gestite le comunicazioni e
l'ambiente di lavoro.

\subsubsection{Aspettative}
Le aspettative sono di avere un insieme di strumenti e modalità che i membri del gruppo dovranno utilizzare e rispettare.

\subsubsection{Descrizione}
In questa parte vengono stabiliti:
\begin{itemize}
    \item Le modalità da seguire per le comunicazioni.
    \item Gli strumenti da adottare per le comunicazioni.
    \item Gli strumenti per la gestione del lavoro.
\end{itemize}

\subsubsection{Gestione delle comunicazioni}
\paragraph{Comunicazioni interne}
Per le comunicazioni interne il gruppo si avvale di due applicazioni, ovvero Zoom Meetings e Telegram.
Zoom è una piattaforma di videoconferenza conosciuta da tutti i membri del team e adatta per le riunioni, mentre Telegram
è una applicazione di messaggistica utile per le conversazioni più rapide.

\paragraph{Comunicazioni esterne}
Le comunicazioni esterne avvengono attraverso:
\begin{itemize}
    \item Email con l'indirizzo di posta elettronica del gruppo (\href{mailto:swe7.team@gmail.com}{swe7.team@gmail.com}).
    \item Telegram per comunicazioni veloci con Imola Informatica.
    \item Zoom per le riunioni con il proponente.
\end{itemize}

\subsubsection{Strumenti per la comunicazione}
\paragraph{Telegram}
Telegram viene usato sia per le comunicazioni interne al gruppo sia per le comunicazioni con il proponente e
sono stati creati rispettivamente due canali di comunicazione. L'applicazione viene usata anche con lo scopo di:
\begin{itemize}
    \item Scambiare brevi messaggi per coordinare il gruppo.
    \item Scambiare link, documenti e altre risorse informative.
    \item Comunicazioni tra i singoli membri.
\end{itemize}

\paragraph{Zoom Meetings}
Programma per le videoconferenze che viene utilizzato per le riunioni sia all'interno del team che con l'azienda. L'applicazione è ritenuta semplice da usare,
è conosciuta da tutti i membri ed è adatta per gli incontri in quanto offre tutte le funzionalità di cui si ha bisogno come:
\begin{itemize}
    \item Comunicazione video e audio.
    \item Servizio di chat e possibilità di salvare quest'ultima.
    \item Condivisione dello schermo.
    \item Registrazione del meeting.
\end{itemize}

\paragraph{Gmail}
Servizio di posta elettronica usato per le comunicazioni esterne con il proponente e il commitente. La email ufficiale
del gruppo è \href{mailto:swe7.team@gmail.com}{swe7.team@gmail.com}.

\subsubsection{Gestione degli strumenti di coordinamento}
\paragraph{Ticketing}
Il sistema di ticketing consente di sapere quali siano le attività da svolgere e quelle che stanno svolgendo i componenti.
Il Responsabile definisce i compiti e li assegna usando l'issue tracking system fornito da GitHub. Un'attività quindi viene:
\begin{itemize}
    \item Creata dal Responsabile.
    \item Assegnata dal Responsabile ai vari membri.
    \item Assegnata ad una milestone.
    \item Dopo essere stata svolta viene verificata.
    \item In caso di esito positivo da parte dei verificatori la issue relativa viene chiusa altrimenti deve essere modificata fino a quando passa l'attività di verifica. 
\end{itemize}
Per ogni issue vengono stabiliti:
\begin{itemize}
    \item titolo che contiene il nome della issue e un codice che la identifica.
    \item descrizione dello scopo della issue.
    \item etichette per specificarne la categoria.
\end{itemize}

\subsubsection{Strumenti per la gestione del lavoro}
\paragraph{Git}
Version control system gratuito e open source che permette di mantenere una cronologia di tutte le modifiche dei file e consente
di fare la maggior parte delle operazioni in locale.

\paragraph{GitHub}
Piattaforma di hosting su cui verranno creati i repository del progetto la quale si integra con Git per la parte di versionamento. 
GitHub mette a disposizione un issue tracking system fondamentale per gestire le attività da svolgere e creare le milestone del progetto.