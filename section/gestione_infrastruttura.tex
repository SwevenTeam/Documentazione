\subsection{Gestione dell'infrastruttura}
\subsubsection{Scopo}
Lo scopo di questa sezione del documento è quello di definire gli strumenti con cui vengono gestite le 
comunicazioni e l'ambiente di lavoro.

\subsubsection{Descrizione}
In questa parte vengono stabiliti:
\begin{itemize}
    \item le modalità da seguire per le comunicazioni;
    \item gli strumenti da adottare per le comunicazioni;
    \item gli strumenti per la gestione del lavoro.
\end{itemize}

\subsubsection{Aspettative}
L'aspettativa principale è quella di avere un insieme di strumenti e modalità che i membri del gruppo 
dovranno utilizzare e rispettare.

\subsubsection{Gestione delle comunicazioni} 
\paragraph{Comunicazioni interne} \hfill \break
Per le comunicazioni interne il gruppo si avvale di due applicazioni, ovvero \glossario{Zoom Meetings} e 
\glossario{Telegram}. \glossario{Zoom} è una piattaforma di videoconferenza conosciuta da tutti i membri del 
team e adatta per le riunioni, mentre \glossario{Telegram} è una applicazione di messaggistica utile per le 
conversazioni più rapide.

\paragraph{Comunicazioni esterne} \hfill \break
Le comunicazioni esterne avvengono attraverso:
\begin{itemize}
    \item Email con l'indirizzo di posta elettronica del gruppo 
        (\href{mailto:swe7.team@gmail.com}{swe7.team@gmail.com}).
    \item \glossario{Telegram} per comunicazioni veloci con Imola Informatica.
    \item \glossario{Zoom} per le riunioni con il proponente.
\end{itemize}

\subsubsection{Strumenti per la comunicazione}
\paragraph{\glossario{Telegram}} \hfill \break
Telegram viene usato sia per le comunicazioni interne al gruppo sia per le comunicazioni con il proponente:
sono stati infatti creati due distinti canali di comunicazione. L'applicazione viene usata anche con lo scopo 
di:
\begin{itemize}
    \item scambiare brevi messaggi per coordinare il gruppo;
    \item scambiare link, documenti e altre risorse informative;
    \item comunicare tra i singoli membri.
\end{itemize}

\paragraph{\glossario{Zoom Meetings}} \hfill \break
Programma per le videoconferenze che viene utilizzato per le riunioni, sia all'interno del team che con l'azienda. L'applicazione è ritenuta semplice da usare,
è conosciuta da tutti i membri ed è adatta per gli incontri, in quanto offre tutte le funzionalità di cui si necessita, tra le quali:
\begin{itemize}
    \item comunicazione video e audio;
    \item servizio di chat e possibilità di salvare quest'ultima;
    \item condivisione dello schermo;
    \item registrazione del meeting.
\end{itemize}

\paragraph{Gmail} \hfill \break
Servizio di posta elettronica usato per le comunicazioni esterne con il proponente e il commitente. L'email 
ufficiale del gruppo è \href{mailto:swe7.team@gmail.com}{swe7.team@gmail.com}.

\subsubsection{Gestione degli strumenti di coordinamento}
\paragraph{Ticketing} \hfill \break
Il sistema di ticketing consente di sapere quali siano le attività da svolgere e quelle che stanno svolgendo i 
componenti. Il Responsabile definisce i compiti e li assegna usando l'\glossario{issue tracking system} fornito 
da GitHub. Un'attività viene quindi:
\begin{itemize}
    \item creata dal Responsabile;
    \item assegnata dal Responsabile ai vari membri;
    \item assegnata ad una milestone;
    \item verificata dopo essere stata svolta;
    \item chiusa in caso di esito positivo da parte dei verificatori, altrimenti modificata fino a quando passa l'attività di verifica. 
\end{itemize}
Per ogni issue vengono stabiliti:
\begin{itemize}
    \item un titolo che contiene il nome della issue e un codice che la identifica;
    \item una descrizione dello scopo della issue;
    \item eventuali etichette per specificarne la categoria.
\end{itemize}

\subsubsection{Gestione dei rischi}
Nel corso di un progetto possono esserci eventuali rischi; questi vanno individuati e riportati nel documento 
\emph{Piano Di Progetto}.

\paragraph{Struttura dei rischi} \hfill \break
La classificazione dei rischi verrà effetuata tramite il seguente codice:
\begin{center}
    \textbf{R [TYPE] [NUMBER]}
\end{center}
\begin{center}
    \renewcommand{\arraystretch}{1.8} %aumento ampiezza righe
    \begin{tabular}{ |m{7em}|m{30em}| }
        \hline
        \textbf{Nome} & \textbf{Descrizione} \\
        \hline
        TYPE & Indica il tipo di rischio :\\
             & \textbf{T} : rischio tecnologico\\
             & \textbf{P} : rischio personale\\
             & \textbf{O} : rischio organizzativo\\
             & \textbf{R} : rischio per requisito\\
        \hline
        NUMBER & Codice numerico identificativo\\
        \hline
    \end{tabular}
\end{center}
Ogni rischio è strutturato come segue:
\begin{itemize}
    \item codice del rischio;
    \item descrizione del rischio;
    \item livello di gravità;
    \item conseguenze nel caso si manifesti;
    \item precauzioni;
    \item contromisure da prendere.
\end{itemize}

\subsubsection{Strumenti per la gestione del lavoro}
\paragraph{Git} \hfill \break
\glossario{Version control system} gratuito e open source che permette di mantenere una cronologia di tutte le 
modifiche dei file e consente di fare la maggior parte delle operazioni in locale.

\paragraph{GitHub} \hfill \break
Piattaforma di hosting su cui verranno creati i repository del progetto, la quale si integra con Git per la 
parte di versionamento. GitHub mette a disposizione un \glossario{issue tracking system} fondamentale per 
gestire le attività da svolgere e creare le milestone del progetto.