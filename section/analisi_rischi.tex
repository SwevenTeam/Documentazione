\subsection{Descrizione}
Durante lo svolgimento del progetto è inevitabile riscontrare vari problemi e imprevisti, quindi il gruppo ritiene opportuno svolgere l’attività dell’analisi dei rischi per evitare o ridurre al minimo i danni possibili.\newline
L’identificazione e la gestione dei rischi vengono effettuati nelle seguenti fasi:
\begin{enumerate}
\item \textbf{identificazione:}  il gruppo identifica tutti i possibili rischi che possono danneggiare la qualità del lavoro;
\item \textbf{analisi dei rischi:} per ogni rischio individuare il suo livello di gravità e la possibile conseguenza;
\item \textbf{pianificazione:} vengono definite le precauzioni necessarie per evitare il rischio e le contromisure nel caso in cui si verifichi;
\item \textbf{controllo:} monitorare continuamente durante lo svolgimento dell’attività e agire di conseguenza.
\end{enumerate}

\subsection{Tipi di rischi}
Il gruppo individua quattro tipologie di rischi principali:
\begin{itemize}
\item rischi tecnologici;
\item rischi personali;
\item rischi organizzativi;
\item rischi per requisiti.
\end{itemize}

\subsection{Elenchi dei rischi}
In questo capitolo vengono riportati quattro elenchi di rischi divisi per tipologia che il gruppo ha attualmente individuato.
Per ogni rischio viene assegnato un codice di riferimento, la gravità valutata dal gruppo, le sue conseguenze, le precauzioni necessarie e le contromisure da applicare nel momento in cui si presenta. 
\subsubsection{Rischi tecnologici}
\textbf RT1:
 Ritardo dovuto all'uso delle nuove tecnologie
\begin{itemize}
\item Livello di gravità: 2;
\item Conseguenze: maggior tempo di apprendimento;
\item Precauzioni: ogni membro si impegna continuamente a comunicare il progresso della propria attività e in caso di difficoltà comunica col gruppo, durante e non al termine della scadenza, per cercare un aiuto;
\item Contromisure: per ogni nuova tecnologia usata, si cerca di capire la complessità a priori e di stabilire le tempistiche più lasche valutando la difficoltà della tecnologia.
\end{itemize}

\textbf RT2: 
Perdita di dati dovuta al malfunzionamento hardware
\begin{itemize}
\item Livello d gravità: 2;
\item Conseguenze: una parte del lavoro viene perso completamente e tutti i lavori dipendenti da quest'ultimo vengono ritardati, può causare un effetto collaterale;
\item Precauzioni: tutti i lavori completati devono essere salvati nello strumento condiviso, ogni membro si impegna a salvare il proprio lavoro frequentemente in un dispositivo alternativo;
\item Contromisure: il componente si impegna a recuperare velocemente il lavoro perso ed il responsabile si occupa di riassegnare i task al resto dei componenti.
\end{itemize}

\textbf RT3: 
Tecnologia non utilizzabile poiché non soddisfa i requisiti o è troppo complessa
\begin{itemize}
\item Livello di gravità: 1;
\item Conseguenze: può causare una ridotta perdita di tempo di lavoro;
\item Precauzioni: nei confronti delle nuove tecnologie, l'analista deve sempre fare una ricerca a priori per valutare l'utilizzabilità;
\item Contromisure: il gruppo dovrà rivalutare un possibile sostituto velocemente.
\end{itemize}

\subsubsection{Rischi personali}
\textbf RP1: 
Contrasto fra i membri
\begin{itemize}
\item Livello di gravità: 3;
\item Conseguenze: possibili ritardi dei lavori per la mancata collaborazione;
\item Precauzioni: tutti i membri devono portare rispetto per gli altri, usando un linguaggio lecito ed educato;
\item Contromisure: il responsabile deve riassegnare il lavoro evitando che il problema peggiori e successivamente deve cercare una soluzione con il team.
\end{itemize}
\textbf RP2:
Mancata consegna del task assegnato
\begin{itemize}
\item Livello di gravità: 2;
\item Conseguenza: interrompe il normale flusso del lavoro progettato;
\item Precauzioni: prima di assegnare un task deve essere analizzata la sua complessità ed ogni membro deve essere assegnato a un task fattibile nell'arco di tempo dedicato;
\item Contromisure: il soggetto deve comunicare al responsabile il motivo della mancanza ed, eventualmente, sarà quest'ultimo a riassegnare il task suddiviso agli altri membri del gruppo.
\end{itemize}

\subsubsection{Rischi organizzativi}
\textbf RO1: 
Impegni organizzativi per problemi accademici o di lavoro
\begin{itemize}
\item Livello di gravità: 1;
\item Conseguenze: indisponibilità per alcune riunioni o attività in un certo periodo;
\item Precauzioni: ogni membro deve comunicare al gruppo i propri impegni personali ed il responsabile deve assegnare i task in base alla disponibilità indicata;
\item Contromisure: nel momento in cui un membro del gruppo non sia disponibile a causa di un problema temporaneo lo comunica immediatamente al responsabile, il quale riassegna i task agli altri membri oppure prolunga il tempo progettato per quel task.
\end{itemize}

\subsubsection{Rischi per requisiti}
\textbf RR1: 
Errata comprensione dei requisiti
\begin{itemize}
\item Livello di gravità: 4;
\item Conseguenze: nel peggior caso, può richiedere una ristrutturazione notevole del lavoro svolto;
\item Precauzioni: durante la progettazione e lo sviluppo, comunicare spesso con l’azienda e, se necessario, chiedere un riunione per chiarire i dubbi;
\item Contromisure: da evitare sempre.
\end{itemize}
\textbf RR2:
Cambiamento dei requisiti
\begin{itemize}
\item Livello di gravità: 5;
\item Conseguenza: può causare un enorme ristrutturazione del lavoro;
\item Precauzioni: dialogare frequentemente con l'azienda per essere aggiornati in ogni momento;
\item Contromisure: il responsabile deve subito chiedere una riunione interna, valutare con il gruppo i cambiamenti riscontrati e, quando è essenziale, ripianificare subito il lavoro.
\end{itemize}



