\subsection{Descrizone}
Durante lo svolgimento del progetto è inevitabile riscontrare vari problemi e imprevisti, quindi il gruppo ritiene opportuno svolgere l’attività dell’analisi dei rischi per evitare o mantenere al minimo i danni possibili.\newline
L’identificazione e la gestione dei rischi vengono effettuato nelle seguenti fase:
\begin{enumerate}
\item identificazione:  il gruppo identifica tutti i possibili rischi che possono danneggiare la qualità del lavoro;
\item aanalisi dei rischio: per ogni rischio individuare il suo livello di gravità e la possibile conseguenza;
\item pianificazione: vengono definite le precauzioni necessarie per evitare il rischio, e le azioni necessarie nel caso in cui il rischio avvenga;
\item controllo: monitorare continuo durante lo svolgimento dell’attività, agire di conseguenza.
\end{enumerate}

\subsection{Tipi di rischi}
Il gruppo individua quattro tipologie di rischi principali:
\begin{itemize}
\item rischi tecnologiche;
\item rischi personali;
\item rischi organizzativi;
\item rischi per requisiti.
\end{itemize}

\subsection{Elenchi dei rischi}
In questo capitolo vengono riportati quattro elenchi di rischi divisi per tipologia che il gruppo ha attualmente individuato.
Per ogni rischio viene assegnato un codice di riferimento, la gravità valutato dal gruppo, la sua conseguenza, le precauzioni necessarie e le contromisure da applicare nel momento in cui il rischio accade. 
\subsubsection{Rischi tecnologiche}
\textbf RT1:
 Ritardo dovuto all'uso delle nuove tecnologie
\begin{itemize}
\item Livello di gravità: 2;
\item Conseguenze: maggior tempo di apprendimento;
\item Precauzioni: ogni membro si impegna continuamente a comunicare il progresso della propria attività, in caso di difficoltà comunica al gruppo durante e non al termine della scadenza, per cercare un aiuta da parte del gruppo;
\item Contromisure: per ogni nuove tecnologie usata, si cerca di capire la complessità a priori, e di stabilire le tempistiche più lasche valutando la difficoltà della tecnologia.
\end{itemize}

\textbf RT2: 
Perdita di dati dovuti al malfunzionamento del hardware
\begin{itemize}
\item Livello d gravità: 2;
\item Conseguenze: una parte del lavoro viene perso completamente, tutti i lavori dipendenti da questo lavoro vengono ritardati, può causare un effetto collaterale;
\item Precauzioni: tutti i lavori completati devono essere salvati nello strumento condiviso, ogni membro si impegna a salvare il proprio lavoro in corso frequentemente in un dispositivo alternativo;
\item Contromisure: il componente si impegna a recuperare velocemente il lavoro perso, il responsabile si occupa di riassegnare i task al resto del componente.
\end{itemize}

\textbf RT3: 
Tecnologia non utilizzabile dovuto alla mancanza dei requisiti o alla complessità
\begin{itemize}
\item Livello d gravità: 1;
\item Conseguenze: può causere una piccola perdita di tempo di lavoro;
\item Precauzioni: nei confronti delle nuove tecnologie, l'analista deve sempre fare una ricerca a priori per valutare l'utilizzabilità;
\item Contromisure: il gruppo dovrà rivalutare un possibile sostituto velocemente.
\end{itemize}

\subsubsection{Rischi personali}
\textbf RP1: 
Contrasto fra i membri
\begin{itemize}
\item Livello di gravità: 3;
\item Conseguenze: possibili ritardi dei lavori per il mancato collaborazione;
\item Precauzioni: tutti i membri devono portare rispetto per gli altri, usando un linguaggio lecito ed educato;
\item Contromisure: il responsabile deve riassegnare il lavoro, evitando che il problema peggiorasse, in seguito cercare una soluzione con il team.
\end{itemize}
\textbf RP2:
Mancato consegna del task assegnato
\begin{itemize}
\item Livello di gravità: 2;
\item Conseguenza: interrompe il normale flusso di lavoro progettato;
\item Precauzioni: prima di assegnare un task, esso deve essere analizzato la sua complessità, ogni membro deve essere assegnato un task fattibile nell'arco di tempo dedicato;
\item Contromisure: il soggetto deve comunicare al responsabile il motivo della mancanza, eventualmente il responsabile riassegna il task suddiviso agli altri membri del gruppo.
\end{itemize}

\subsubsection{Rischi organizzativi}
\textbf RO1: 
iìImpegno organizzativi dovuto ai problemi accademici o di lavoro
\begin{itemize}
\item Livello di gravità: 1;
\item Conseguenze: indisponibilità di qualche unione o attività per un certo periodo;
\item Precauzioni: ogni membro deve comunicare al gruppo i propri impegni personali, il responsabile deve assegnare i task in base alla disponibilità indicata;
\item Contromisure: al momento in cui un membro del gruppo non avesse la disponibilità a causa di un problema provvisorio, comunica immediatamente al responsabile, il responsabile riassegna i task agli altri membri oppure prolungare il tempo progettato per quel task.
\end{itemize}

\subsubsection{Rischi per requisiti}
\textbf RR1: 
Errata comprensione dei requisiti
\begin{itemize}
\item Livello di gravità: 4;
\item Conseguenze: nel peggior caso, può richiedere una ristrutturazione notevole del lavoro svolto;
\item Precauzioni: durante la progettazione e lo sviluppo, scambiare spesso l’informazione con l’azienda, se necessario, chiedere un unione per chiarire domande non chiare;
\item Contromisure: evitare sempre.
\end{itemize}
\textbf RR2:
Cambiamento dei requisiti
\begin{itemize}
\item Livello di gravità: 5;
\item Conseguenza: puo causare un enorme ristrutturazione del lavoro;
\item Precauzioni: un dialogo frequente con l'azienda per essere aggiornato ad ogni momento;
\item Contromisure: il responsabile deve subito chiedere una riuniune interna, valutare con il gruppo i cambiamenti neccessati, ripianificare subito il lavoro quando dovesse.
\end{itemize}



