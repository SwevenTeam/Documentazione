\subsection{Miglioramento}
\subsubsection{Scopo}
Il processo di miglioramento serve a controllare tutto ciò che riguarda il progetto per poterlo migliorare in qualche suo aspetto, come efficenza ed efficacia.
\subsubsection{Aspettative}
Con il processo di miglioramento si cerca di tenere sotto controllo tutto quello che viene eseguito nell'ambito del progetto, per ottenere da ogni controllo dei feedback, di questi si continuerà a mantenere gli aspetti positivi e si cercherà di migliorare quelli negativi. Particolare attenzione verrà posta all'efficacia e all'efficenza dei processi, queste andranno costantemente monitorate per essere, se serve, migliorate.
\subsubsection{Attività}
Per avere un miglioramento continuo il gruppo si confronterà ad intervalli opportuni su ogni processo, in modo da poter revisionare il lavoro svolto. Ogni controllo sarà atto a confermare gli aspetti positivi e definire i miglioramenti da apportare al processo in esame.