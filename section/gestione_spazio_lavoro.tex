\subsection{Gestione dello spazio di lavoro}
Questa sezione serve a dare un modello alle procedure di lavoro e a normalizzare i comportamenti
necessari per favorire lo svolgersi ordinato del processo nella sua totalità.
\subsubsection{Stato dei documenti}
In precedenza è stato normalizzato il versionamento dei documenti.
Un'altra caratteristica altrettanto importante che essi presentano è lo stato.
Lo stato di un documento è in una e una sola delle seguenti due alternative:
\begin{itemize}
    \item In lavorazione;
    \item Approvato.
\end{itemize}
Ogni documento parte dalla fase ``In lavorazione''; solo l'approvatore può, dopo aver analizzato il 
documento, decidere se renderlo ``Approvato'' o meno. \newline
Per scelta del gruppo, ogni documento può essere aggiunto al ramo principale della repository di GitHub 
in uso solo se presenta lo stato ``Approvato''.
\subsubsection{Versionamento dei software}
Similmente ai documenti, anche i software sono versionati. Ciò avviene secondo il formato:
\begin{center}
    \textbf{[x].[y].[z]}
\end{center}
\begin{description}
    \item \textbf{x}: Indica la versione del software.
            Viene aggiornato solo quando si soddisfano tutti i requisiti obbligatori. 
    \item \textbf{y}: Viene incrementato ad ogni requisito aggiornato.
    \item \textbf{z}: Viene incrementato quando si modifica il software senza raggiungere requisiti.
\end{description}
Come nei documenti, anche per i software la modifica di un qualsiasi campo della versione resetta tutti gli 
altri campi presenti alla destra di esso. \newline
In sintesi x indica il raggiungimento dell'obiettivo primario, y gli obiettivi raggiunti dall'ultimo cambiamento di x,
z il numero di piccole modifiche dall'ultimo cambiamento di x e y. \newline
Quindi il software alla sua creazione è già alla versione 0.0.1 e a fine progetto è necessario che 
tutti i software siano alla versione 1.y.z.
\subsubsection{Organizzazione della repository}
I partecipanti del gruppo lavorano singolarmente in locale. Al termine della loro task, condividono 
il lavoro effettuato tramite un commit del file, o dei file, nel ramo adeguato della repository.
I suffissi dei file ricadono nei seguenti casi:
\begin{itemize}
    \item \textbf{.tex}: Il codice \LaTeX;
    \item \textbf{.jpg .png}: Immagini necessarie, sempre contenute nella cartella ``images'';
    \item \textbf{.pdf}: Il formato PDF del documento da visionare.
\end{itemize}
\subsubsection{Task}
Le task da svolgere singolarmente o in sottogruppi vengono assegnate ai membri del gruppo Sweven tramite
GitHub. La piattaforma infatti mette a disposizione degli utenti uno strumento di creazione e gestione delle 
issues. Tali issues verranno create con i seguenti requisiti:
\begin{itemize}
    \item \textbf{Titolo}: un titolo semplice, immediato e in grado di riassumere la descrizione;
    \item \textbf{Descrizione}: la descrizione dell'obiettivo da raggiungere, sia in complessivo sia singolarmente; 
                                vengono quindi elencate le task che i partecipanti dovranno seguire;
    \item \textbf{Label}: un'etichetta che descrive il tipo di lavoro da effettuare;
    \item \textbf{Partecipanti}: la lista di tutti i partecipanti alla issue;
    \item \textbf{Milestone}: il macrogruppo di obiettivi cui la issue appartiene;
    \item \textbf{Project}: indica a quale revisione la issue appartiene.
\end{itemize}