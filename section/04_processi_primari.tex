
\subsection{Processo di Fornitura}
	
	\subsubsection{Descrizione}
		Il processo di fornitura determina le principali attività e risorse necessarie alla realizzazione del progetto, al fine di comprendere al meglio le richieste del proponente.\\
		Lo scopo del processo di fornitura è di poter fornire un prodotto che rispetti i requisiti concordati. Il Processo è formato dalle seguenti fasi : 
		\begin{itemize}
			\item Avvio;
			\item Contrattazione;
			\item Pianificazione;
			\item Esecuzione e controllo;
			\item Revisione e valutazione;
			\item Consegna e completamento.
		\end{itemize}
	
	\subsubsection{Scopo e Aspettative}
		Questo documento contiene le norme che il gruppo dovrà seguire, necessarie al fine di ottenere il ruolo di fornitori del proponente. In particolare, elementi cardine del documento sono : 
		\begin{itemize}
			\item Comprendere e soddisfare le richieste del proponente;
			\item Definire le tempistiche di lavoro;
			\item Chiarire eventuali dubbi con il proponente.
		\end{itemize}
		Per assolvere al meglio a questi obiettivi, il gruppo instaura un canale di comunicazione Telegram con i referenti della ditta Imola Informatica.
	
	\subsubsection{Documenti}
		
		Per poter redigere i documenti nel modo più completo possibile, vengono utilizzati a pieno le documentazioni fornite per la candidatura e vengono organizzate riunioni informative con l'azienda Imola Informatica.
		
		\paragraph{Studio di Fattibilità}  \hfill \break
	 		Documento contenente un'attenta analisi sui capitolati proposti, al fine di dare più informazione possibili al gruppo per poter compiere una scelta adeguata nella candidatura per l'appalto. \\
	 		Vengono forniti per ogni capitolato :
	 		\begin{itemize}
	 			\item \textbf{Informazioni Generali}
	 			\item \textbf{Descrizione del capitolato}: descrizione riassuntiva sul capitolato;
	 			\item \textbf{Finalità del progetto}: descrizione degli obiettivi fissati per la realizzazione del capitolato;
	 			\item \textbf{Tecnologie Interessate}: elenco di tecnologie, spesso consigliate dal proponente stesso, di interesse per la realizzazione del progetto;
	 			\item \textbf{Aspetti positivi e Criticità}: elementi positivi e criticità riscontrate da una prima analisi sul capitolato;
	 			\item \textbf{Conclusione}: nella quale si motiva la scelta, positiva o negativa, sulla candidatura per il capitolato.
	 		\end{itemize} 
 		
		\paragraph{Piano di Progetto}  \hfill \break
			Documento redatto da amministratori e responsabile di progetto. Questo sarà composto da :
			\begin{itemize}
				\item \textbf{Analisi dei rischi} % da aggiungere più tardi
				\item \textbf{Modello di sviluppo}: contiene descrizioni e motivazioni sulla scelta del modello per l'organizzazione del progetto.
				\item \textbf{Pianificazione}: descrive una prima versione su come verrà suddiviso il lavoro del gruppo nel tempo, insieme ad una descrizione degli strumenti che verranno utilizzati per questo scopo.
				\item \textbf{Preventivo e Consuntivo}: vengono riportati rispettivamente Preventivi e Consuntivi.
			\end{itemize}
	
		\paragraph{Piano di Qualifica}  \hfill \break
			%riempi quando abbiamo i dati
			Documento redatto dal .
			Questo documento è necessario affinché il gruppo possa produrre materiale di qualità.\\
			Esso contiene : 
			\begin{itemize}
				\item \textbf{Qualità di processo}
				\item \textbf{Qualità di prodotto}
				\item \textbf{Test}
				\item \textbf{Resoconto attività di verifica}
			\end{itemize}
	
	\subsubsection{Altra Documentazione Fornita}
		\begin{center}
			\renewcommand{\arraystretch}{1.8} %aumento ampiezza righe
			\begin{tabular}{ |m{10em}|m{30em}| }
				
				\hline
				\textbf{Nome}			& \textbf{Descrizione} \\
				\hline
				Proof of concept 		& Documento che fornisce una visuale generale dell'applicazione \\
				\hline
				Technology Baseline & Definisce l'insieme delle tecnologie utilizzate\\
				\hline
				Product Baseline & Definisce l'insieme di scelte a livello implementativo fatte dal gruppo\\
				\hline
			\end{tabular}
		\end{center}
		
	\subsubsection{Strumenti}
		\paragraph{Gantt Project Manager} \hfill \break
			Strumento utilizzato per la realizzazione di diagrammi di
			Gantt. Questi sono utili per tenere traccia delle attività
			coinvolte nella realizzazione del prodotto e delle relazioni
			tra loro, offrendo inoltre una scansione temporale delle attività.
	
\subsection{Processo di Sviluppo}
	\subsubsection{Descrizione}
		Lo scopo del processo di Sviluppo è di trasformare l'insieme dei requisiti,l'architettura ed i design presentati in un prodotto in grado di soddisfare i requisiti proposti dal proponente.
		
		
	\subsubsection{Scopo}
		Il compito di questo documento è di descrivere i compiti e le attività necessarie al corretto svolgimento del prodotto. Queste possono riassumersi in :
		\begin{itemize}
			\item stabilire gli obiettivi di sviluppo;
			\item stabilire i vincoli tecnologici;
			\item stabilire i vincoli di design;
			\item realizzare un prodotto che superi test, soddisfi requisiti e richieste del proponente.
		\end{itemize}
	
	\subsubsection{Analisi dei Requisiti}
		L'analisi dei requisiti è composta da uno studio sul capitolato che porta alla stesure dell'omonimo documento. Questo deve essere scritto in maniera efficace, poiché rappresenta una base da cui partire per effettuare eventuali miglioramenti successivi, al fine di garantire un perfezionamento del prodotto. 
		
		\paragraph{Struttura} 
		\begin{itemize}
			\item Descrizione generale del prodotto
			\item Elenco dei Casi d'Uso
			\item Elenco dei Requisiti, divisi per categoria
		\end{itemize}
	
		\newpage	
		\paragraph{Classificazione dei Requisiti} \hfill \break
			La classificazione dei requisiti verrà identificata tramite il seguente codice : 
			\begin{center}
				\textbf{R [PRIORITY] - [TYPE] - [NUMBER]}
			\end{center}

			\renewcommand{\arraystretch}{1.8} %aumento ampiezza righe
			\begin{tabular}{ |m{7em}|m{30em}| }
				\hline
				\textbf{Nome} & \textbf{Descrizione} \\
				\hline
				 R & Acronimo di Requisito \\
				\hline
				PRIORITY 	& 	Indica il tipo di priorità : \\
				 			&	\textbf{O} : Requisito obbligatorio \\
				 			&	\textbf{D} : Requisito desiderabile \\
				 			&	\textbf{F} : Requisito Funzionale \\
				 \hline
					 	 
				TYPE 		& 	Indica il tipo di requisito : \\
							& 	\textbf{F} : Requisto Facoltativo; definizione di una 	caratteristica necessaria nel software \\
							&	\textbf{V} : Requisito di Vincolo; rappresenta un vincolo avanzato \\
							&	\textbf{Q} : Requisito di Qualità; inerente le regole di qualità \\
				\hline
				 NUMBER & Codice Numerico Identificativo \\
				 \hline
			\end{tabular}
		
		\paragraph{Classificazione dei Casi d'Uso} \hfill \break
			La classificazione dei casi d'uso verrà identificata tramite il seguente codice : 
			\begin{center}
				\textbf{UC [CASECODE] . [CASESUBCODE]}
			\end{center}
		
		\begin{tabular}{ |m{10em}|m{27em}| }
			\hline
			\textbf{Nome} 	& \textbf{Descrizione} \\
			\hline
			UC 				& Acronimo di Caso d'Uso \\
			\hline
			CASECODE		& Identifica un'instanza generica del Caso d'Uso  \\
			\hline
			CASESUBCODE 	& Identifica un'istanza specifica del Caso d'Uso \\
			\hline
		\end{tabular}
		
		
		\paragraph{Composizione dei Casi d'Uso} \hfill \break
		Un Caso d'Uso è definito dalle seguenti componenti : 
		
		\begin{itemize}
			\item Identificativo
			\item Nome
			\item Descrizione Grafica, creata con UML 2.0
			\item Attori :
			\begin{itemize}
				\item Primari, interagisce direttamente col sistema per ottenere un obiettivo
				\item Secondari, aiuta l'Attore Primario nel raggiungimento dell'obiettivo
			\end{itemize}
			\item Precondizione, ovvero lo stato del sistema affinché la funzione sia disponibile
			\item Post-condizione, ovvero lo stato del sistema dopo il Caso d'Uso
			\item Scenario Principale, ovvero una rappresentazione del flusso degli eventi
			\item Estensione, opzionale; usato per casi alternativi al verificarsi di una specifica condizione
		\end{itemize}
		
%		\paragraph{Qualità dei Requisiti}
%			\begin{center}
%				\renewcommand{\arraystretch}{1.8} %aumento ampiezza righe
%				\begin{tabular}{ |c|c| }
%					\hline
%					\textbf{Nome} & \textbf{Descrizione} \\
%					\hline
%					Completezza 	& \\ 
%					Consistenza 	& \\
%					Correttezza 	& \\
%					Univocità 		& \\
%					Modificabilità 	& \\
%					Non Ambiguità 	& \\
%					Priorità 		& \\ 
%					Verificabilità  & \\
%					Tracciabilità 	& \\
%			%		\hline
%				\end{tabular}
%			\end{center}

		\subsubsection{Progettazione} 
			Definisce le caratteristiche necessarie per la soddisfazione dei requisiti presentati nell'Analisi dei Requisiti.
			
			\paragraph{Descrizione} \hfill \break
				La fase di Progettazione è divisa in due parti :
				\begin{itemize}
					\item \textbf{Technology Baseline} : descrive le specifiche del prodotto; contiene i diagrammi IML ed i test di verifica
					\item \textbf{Product Baseline} : descrive in maniera dettagliata l'attivita di progettazione integrando i contenuti presenti in Technology Baseline; definisce inoltre i test necessari alla verifica
				\end{itemize} 	
			
			\paragraph{Design Patterns} \hfill \break
				Ogni design pattern utilizzato per realizzare l'architettura deve essere descritto e accompagnato da un diagramma che ne esponga significato e struttura.
				
			\paragraph{Diagrammi UML} \hfill \break
				\begin{tabular}{ |m{15em}|m{25em}| }
					\hline
					\textbf{Diagramma delle attività}	& descrive le operazioni di una attività \\
					\hline
					\textbf{Diagramma delle classi}		& elenca attributi, metodi di classi, tipi \\
					\hline
					\textbf{Diagramma del package}		& mostra raggruppamenti tra classi  \\
					\hline
					\textbf{Diagramma di sequenza} 		& descrive sequenze di azioni  \\
					\hline
				\end{tabular}
			
			\paragraph{Test} \hfill \break
				Ogni progettista dovrà occuparsi di definire opportuni test, al fine di individuare eventuali errori o anomalie. \\
				I test dovranno essere accompagnati da eventuali classi utili a raggiungere lo scopo.
			
		
		\subsubsection{Codifica} 
		
			\paragraph{Descrizione} \hfill \break
				Lo scopo della Codifica è di fornire una base comune su la realizzazione del prodotto software, al fine di poter creare codice uniforme e agevolare manutenzione ed eventuali modifiche future. \\
				I programmatori si dovranno attenere a queste norme durante le fasi di progettazione ed implementazione. 
			
			\paragraph{Convenzioni per I nomi} 
				\begin{itemize}
					\item nomi univoci;
					\item Pascal Case per variabili ;
					\item Snake case per metodi e classi. % da rivedere dopo
				\end{itemize}
			
			\paragraph{Convenzioni sulle pratiche} \hfill \break
			\begin{tabular}{ |m{15em}|m{25em}| }
				\hline
				\textbf{Pratica}				& \textbf{Descrizione}\\
				\hline
				\textbf{Variabili Utili}		& Non definire variabili se queste non sono poi utilizzate.\\
				\hline
				\textbf{Funzioni Ricorsive}		& Evitare quando possibile funzioni ricorsive.\\
				\hline
				\textbf{Delimitatori Stringhe}	& Ogni stringa deve essere delimitata dal singolo apice.\\
				\hline
				\textbf{Operatori}				& Gli operatori, quando utilizzati, dovranno essere preceduti e seguiti da spazi.\\
				\hline
				\textbf{Indentazione}			& Il codice, per una maggiore leggibilità, dovrà essere indentato con 4 spazi.\\ %poi possiamo cambiare il valore
				\hline
				\textbf{Commenti}				& Il codice dovrà essere commentato. I commenti devono essere separati dall'identificatore di commento da uno spazio. Se il commento è inline, dovrà essere presente uno spazio prime dell'identificatore di commento.\\
				\hline
				\textbf{Complessità Ciclomatica}& Viene evitato il più possibile l'annidamento dei cicli, poiché questo renderebbe le funzioni troppo onerose.\\
				\hline

			\end{tabular}
			
			\paragraph{Strumenti} \hfill \break
			% aggiunta in seguito quando scegliamo come fare
			
		\subsubsection{Metriche} \hfill \break
				
			\paragraph{Metriche di Documentazione} \hfill \break
			Le Metriche di documentazione vengongo utilizzate per fornire informazioni numeriche sulla qualità dei documenti redatti.  Queste sono :
			\begin{itemize}
				\item \textbf{MD-01: Indice di Gulpease}; \\ Indica il grado di leggibilità di un testo in lingua italiana. Utilizza la formula : \\
				\begin{center}
					$ \textbf{indiceDiGulp} = 89 + \frac{ [300 \cdot numeroDelleFrasi] - [10 \cdot numeroDelleLettere]}{numeroDelleParole}$...
				\end{center}
					
				\item \textbf{MD-02: Correttezza ortografica} \\ Permette di misurare la correttezza lessicografica della documentazione.\\
			\end{itemize}			
		
			% AGGIUNGERE METRICHE CODICE