\subsection{Processo di Fornitura}
\subsubsection{Scopo}
Il processo di fornitura descrive le principali attività ed i compiti necessari alla realizzazione del 
progetto, al fine di comprendere al meglio le richieste del proponente.

\subsubsection{Descrizione}
Le varie attività da svolgere vengono in seguito ripartite nel documento \textit{Piano di Progetto}. \\
Il processo di fornitura è formato dalle seguenti fasi: 
\begin{itemize}
    \item avvio;
    \item contrattazione;
    \item pianificazione;
    \item esecuzione e controllo;
    \item revisione e valutazione;
    \item consegna e completamento.
\end{itemize}

\subsubsection{Aspettative}
Per tutta la durata del progetto, il gruppo \textit{Sweven Team} manterrà un dialogo costante con l'azienda \textit{Imola Informatica}. 
Per assolvere al meglio a questa aspettativa, viene creata una chat sull'applicazione \glossario{Telegram} con la quale comunicare direttamente con i referenti dell'azienda proponente. 
Grazie ad un dialogo continuo, sarà possibile:
\begin{itemize}
\item comprendere e soddisfare al meglio le richieste del proponente ed i requisiti che il prodotto dovrà soddisfare;
\item definire al meglio le tempistiche di lavoro ed i costi associati ad esse;
\item effettuare una verifica costante e continua tramite il feedback fornitoci direttamente dai referenti aziendali;
\item chiarire eventuali dubbi con il proponente.
\end{itemize}

\subsubsection{Attività}
\paragraph{Documenti}
Per poter redigere i documenti nel modo più completo possibile, vengono utilizzate a pieno le documentazioni fornite per la candidatura e vengono organizzate riunioni informative con l'azienda  \textit{Imola Informatica}.
    
    \subparagraph*{Studio di Fattibilità}  \hfill \break
     Documento contenente un'attenta analisi sui capitolati proposti, il cui scopo è quello di dare più informazioni possibili al gruppo per poter compiere una scelta adeguata nella candidatura per l'appalto. \\
     L'analisi di ogni capitolato è composta da:
     \begin{itemize}
         \item \textbf{Informazioni Generali};
         \item \textbf{Descrizione del capitolato}: descrizione riassuntiva sul capitolato;
        \item \textbf{Finalità del progetto}: descrizione degli obiettivi fissati per la realizzazione del capitolato;
         \item \textbf{Tecnologie Interessate}: elenco di tecnologie, spesso consigliate dal proponente stesso, di interesse per la realizzazione del progetto;
         \item \textbf{Aspetti positivi e Criticità}: elementi positivi e criticità riscontrate da una prima analisi sul capitolato;
         \item \textbf{Conclusione}: qui si motiva la scelta sulla candidatura, positiva o negativa, per lo specifico capitolato.
    \end{itemize} 
     
    \subparagraph*{Piano di Progetto}  \hfill \break
    Documento redatto da amministratori e responsabile di progetto. \\
Questo sarà composto da:
    \begin{itemize}
        \item \textbf{Analisi dei rischi}: contiene un'analisi sui rischi riscontrabili durante il progetto. Ad ogni possibile rischio viene assegnato un livello di gravità, alcune precauzioni per evitare il rischio e alcune contromisure per ridurre l'entità del problema in caso questo avvenga; 
        \item \textbf{Modello di sviluppo}: contiene descrizioni e motivazioni sulla scelta del modello per l'organizzazione del progetto;
        \item \textbf{Pianificazione}: descrive una prima versione di suddivisione del lavoro del gruppo nel corso del tempo, insieme ad una descrizione degli strumenti che verranno utilizzati;
        \item \textbf{Preventivo e Consuntivo}: vengono riportati rispettivamente preventivi e consuntivi.
    \end{itemize}

    \subparagraph*{Piano di Qualifica}  \hfill \break
    Documento necessario affinché il gruppo possa produrre materiale di qualità. \\
Questo documento viene redatto dagli amministratori e contiene:
    \begin{itemize}
        \item \textbf{Qualità di processo}: definisce i valori delle metriche necessari al fine di ottenere processi di qualità;
        \item \textbf{Qualità di prodotto}: definisce i valori delle metriche necessari al fine di ottenere un prodotto di qualità;
        \item \textbf{Test};
        \item \textbf{Resoconto attività di verifica}.
    \end{itemize}

  \paragraph{Revisioni} \hfill \break
    Attività che prevede l'interazione con proponente e committente per revisionare il progetto in corso d'opera.
    \begin{itemize}
      \item \textbf{Proponente:} incontro ogni volta che il team lo ritiene necessario per sciogliere dubbi o mostrare un notevole avanzamento del prodotto;
      \item \textbf{Committente:} 2 revisioni intermedie:
            \begin{enumerate}
              \item Requirements and Technology Baseline dopo l'analisi dei requisiti e una demo del prodotto;
              \item Product Baseline dopo lo sviluppo del prodotto e prima del collaudo.
            \end{enumerate}          
    \end{itemize}
    \paragraph{Verifica e consegna finale} \hfill \break
    Attività dell'ultimo periodo di progetto che prevede: 
    \begin{itemize}
      \item \textbf{Validazione e collaudo:} del prodotto progettato e creato dal team tramite numerosi test;
      \item \textbf{Rilascio:} del prodotto e della documentazione;
      \item \textbf{Consegna finale:} Customer Acceptance è la revisione finale in cui il team presenta e consegna il prodotto, alla presenza sia del committente che del proponente.
    \end{itemize}
    
  \paragraph{Strumenti}
    \subparagraph*{Gantt Project Manager} \hfill \break
    Strumento utilizzato per la realizzazione di diagrammi di Gantt. Questi sono utili per tenere traccia 
delle attività coinvolte nella realizzazione del prodotto e delle relazioni
    tra loro, offrendo inoltre una scansione temporale delle attività.
    \subparagraph*{ \glossario{Google Sheet} } \hfill \break
Strumento che consiste in fogli di calcolo. Viene utilizzato per il calcolo dei preventivi e la suddivisione delle ore per il \textit{Piano di Progetto}.
    \subparagraph*{Zoom} \hfill \break
Piattaforma che permette l'organizzazione di riunioni online usato per effettuare le revisioni con proponente e committente.
  \paragraph{Metriche}  \hfill \break
  Non sono state rilevati particolare metriche per il processo di Fornitura.

\subsection{Processo di Sviluppo}
\subsubsection{Scopo}
Lo scopo del processo di Sviluppo è quello di descrivere e normare i compiti e le attività di analisi, progettazione, codifica del prodotto da realizzare.

\subsubsection{Descrizione}
Le attività principali che questo processo definisce sono:
\begin{itemize}
\item analisi dei requisiti;
\item progettazione architetturale;
\item codifica software.
\end{itemize}
    
\subsubsection{Aspettative}
Le aspettative del gruppo possono riassumersi in:
\begin{itemize}
    \item stabilire gli obiettivi di sviluppo;
    \item stabilire i vincoli tecnologici;
    \item stabilire i vincoli di design;
    \item realizzare un prodotto che superi test, soddisfi requisiti e richieste del proponente.
\end{itemize}

\subsubsection{Analisi dei Requisiti}
\paragraph{Descrizione} \hfill \break
L'analisi dei requisiti è una attività che viene eseguita dagli analisti e porta alla creazione 
dell'omonimo documento \textit{Analisi dei Requisiti}.
Questo sarà composto da:
\begin{itemize}
  \item descrizione generale del prodotto;
  \item elenco dei Casi d'Uso;
  \item elenco dei requisiti, divisi per categoria.
\end{itemize}

\paragraph{Scopo} \hfill \break
Lo scopo dell'analisi dei requisiti è quello di:
\begin{itemize}
  \item fornire requisiti semplici e precisi;
  \item fissare con il proponente le funzionalità che il prodotto dovrà possedere;
  \item assistere l'attività dei verificatori, rendendo l'attività di tracciamento dei requisiti più semplice;
  \item fornire informazioni utili per la stima del calcolo della mole del lavoro.
\end{itemize}

\paragraph{Classificazione dei Requisiti} \hfill \break
La classificazione dei requisiti verrà identificata tramite il seguente codice: 
\begin{center}
  \textbf{R [PRIORITY] - [TYPE] - [NUMBER]}
\end{center}

\renewcommand{\arraystretch}{1.8} %aumento ampiezza righe
\begin{tabular}{ |m{7em}|m{30em}| }
  \hline
  \textbf{Nome} & \textbf{Descrizione} \\
  \hline
  R & Acronimo di Requisito \\
  \hline
  PRIORITY & Indica il tipo di priorità: \\
        &	\textbf{O}: Requisito obbligatorio \\
        &	\textbf{D}: Requisito desiderabile \\
        &	\textbf{F}: Requisito facoltativo  \\
  \hline 	 
  TYPE & Indica il tipo di requisito: \\
        & \textbf{F}: Requisto funzionale; definizione di una 	caratteristica necessaria nel software \\
        &	\textbf{V}: Requisito di vincolo; rappresenta un vincolo avanzato \\
        &	\textbf{Q}: Requisito di qualità; inerente le regole di qualità \\
  \hline
  NUMBER & Codice Numerico Identificativo \\
  \hline
\end{tabular}
  
\paragraph{Classificazione dei Casi d'Uso} \hfill \break
La classificazione dei Casi d'Uso verrà identificata tramite il seguente codice : 
\begin{center}
  \textbf{UC [CASECODE] . [CASESUBCODE]}
\end{center}
  
\begin{tabular}{ |m{10em}|m{27em}| }
  \hline
  \textbf{Nome} & \textbf{Descrizione} \\
  \hline
  UC & Acronimo di caso d'uso \\
  \hline
  CASECODE & Identifica un'instanza generica del caso d'uso  \\
  \hline
  CASESUBCODE & Identifica un'istanza specifica del caso d'uso \\
  \hline
\end{tabular}
  
\paragraph{Composizione dei Casi d'Uso} \hfill \break
Un caso d'uso è definito dalle seguenti componenti: 
\begin{itemize}
  \item Identificativo;
  \item Nome;
  \item Descrizione grafica, creata con UML 2.0;
  \item Attori:
  \begin{itemize}
    \item Primario, interagisce direttamente col sistema per ottenere un obiettivo;
    \item Secondari (opzionali), aiutano l'attore primario nel raggiungimento dell'obiettivo.
  \end{itemize}
    \item Precondizione, ovvero lo stato del sistema affinché la funzione sia disponibile;
    \item Post-condizione, ovvero lo stato del sistema dopo il caso d'uso;
    \item Scenario principale, ovvero una rappresentazione del flusso degli eventi;
    \item Estensioni (opzionali), usate per casi alternativi e al verificarsi di specifiche condizioni.
  \end{itemize}
  
\paragraph{Strumenti} \hfill \break
Tutti i diagrammi UML vengono realizzati utilizzando \glossario{StarUML} nella versione 2.0.

\paragraph{Metriche} \hfill \break
\textbf{M1PRR}:
\begin{itemize}
  \item Nome: Percentuale di requisiti obbligatori soddisfatti;
  \item Descrizione: percentuale che rappresenta i requisiti obbligatori soddisfatti rispetto al totale dei requisiti obbligatori;
  \item Scopo: assicurarsi che il prodotto soddisfi tutte le richieste obbligatorie del proponente;
  \item Formula:
  \begin{center}
    $ \frac{\textit{requisiti obbligatori soddisfatti}}{\textit{requisiti obbligatori totali}}\cdot100$
  \end{center}
\end{itemize}
\textbf{M2PDR}:
\begin{itemize}
  \item Nome: Percentuale di requisiti desiderabili soddisfatti
  \item Descrizione: percentuale che rappresenta i requisiti desiderabili soddisfatti rispetto al totale dei requisiti desiderabili;
  \item Scopo: mostrare ulteriori elementi che portano valore aggiunto al progetto;
  \item Formula:
  \begin{center}
    $ \frac{\textit{requisiti desiderabili soddisfatti}}{\textit{requisiti desiderabili totali}}\cdot100$
  \end{center}
\end{itemize}
\textbf{M3POR}:
\begin{itemize}
  \item Nome: Percentuale di requisiti opzionali soddisfatti
  \item Descrizione: percentuale che rappresenta i requisiti opzionali soddisfatti rispetto al totale dei requisiti opzionali;
  \item Scopo: assicurarsi che il prodotto soffisfi le richieste del proponente;
  \item Formula:
    \begin{center}
      $ \frac{\textit{requisiti opzionali soddisfatti}}{\textit{requisiti opzionali totali}}\cdot100$
    \end{center}
\end{itemize}
\textbf{M4VR}:
\begin{itemize}
  \item Nome: Variazione dei requisiti
  \item Descrizione: numero che indica quanti requisiti sono cambiati nel tempo;
  \item Scopo: controllare quanti requisiti sono cambiati, al fine di supportare le metriche precedentemente descritte;
  \item Formula: $ RA + RE + RM $, posti
    \begin{itemize}
      \item RA: numero di requisiti aggiunti dall'ultimo incremento;
      \item RE: numero di requisiti eliminati dall'ultimo incremento;
      \item RM: numero di requisiti modificati dall'ultimo incremento.
    \end{itemize}
\end{itemize}

\subsubsection{Progettazione} 
\paragraph{Descrizione} \hfill \break
La fase di Progettazione è divisa in due parti:
    \begin{itemize}
        \item \textbf{\glossario{Technology Baseline}}: descrive le specifiche della progettazione del prodotto ad alto livello; contiene inoltre diagrammi UML;
        \item \textbf{\glossario{Product Baseline}}: descrive in maniera dettagliata l'attivita di progettazione, integrando i contenuti presenti in Technology Baseline; definisce inoltre i test necessari alla verifica.
    \end{itemize} 	

\paragraph{Scopo} \hfill \break
Definisce le caratteristiche necessarie per la soddisfazione dei requisiti presentati nell'Analisi dei Requisiti.
  Questo affinché si possa ottenere la miglior soluzione in grado di soddisfare a pieno gli \glossario{stakeholder}.\\
     
\paragraph{Technology Baseline} \hfill \break
Viene redatto dai progettisti e contiene:
\begin{itemize}
  \item \textbf{diagrammi UML};
  \item \textbf{tecnologie utilizzate}: descrizione sulle tecnologie utilizzate e scelte che hanno portato alla specifica tecnologia;
  \item \textbf{tracciamento delle componenti}: ogni requisto soddisfatto da un componente dovrà fare riferimento a quest'ultimo;
  \item \textbf{design pattern};
  \item \textbf{\glossario{Proof of Concept}}: prototipo creato per dimostrare l'effettiva fattibilità e funzione del prodotto.
\end{itemize}
  \subparagraph{Diagrammi UML} 
  \begin{itemize}
    \item \textbf{Diagramma delle attività}: descrive le operazioni di una attività;
    \item \textbf{Diagramma delle classi}: elenca attributi, metodi di classi, tipi; 
    \item \textbf{Diagramma del package}: mostra raggruppamenti tra classi;
    \item \textbf{Diagramma di sequenza}: descrive sequenze di azioni.
  \end{itemize}
     
  \subparagraph{Design Pattern} \hfill \break
  Ogni design pattern utilizzato per realizzare l'architettura deve essere descritto e accompagnato da un diagramma che ne esponga significato e struttura.

\paragraph{Product Baseline} \hfill \break
Viene redatta dal progettista e contiene:
\begin{itemize}
  \item \textbf{Test}:
  \begin{itemize}
    \item di Unità: verificano il funzionamento di una unità;
    \item di Integrazione: verificano l'interazione tra unità;
    \item di Sistema: verificano il funzionamento del sistema.
  \end{itemize}
  \item \textbf{Definizione delle classi};
  \item \textbf{Tracciamento delle classi}: ogni requisto soddisfatto da un classe dovrà fare riferimento a quest'ultima.
  
\end{itemize}

\paragraph{Metriche} \hfill \break
\textbf{M5ATC}:
\begin{itemize}
  \item Nome: Accoppiamento tra classi;
  \item Descrizione: indica il numero di classi da cui una classe dipende;
  \item Scopo: ridurre il numero di dipendenze per rendere il prodotto più semplice e più facilmente modificabile nel tempo.
\end{itemize}
\textbf{M6PDG}:
\begin{itemize}
  \item Nome: Profondità delle gerarchie;
  \item Descrizione: numero che indica, data una gerarchia, il massimo numero di classi parenti;
  \item Scopo: limitare la possibilità di avere gerarchie troppo ampie, poiché renderebbe il prodotto troppo complesso e più difficilmente modificabile.
\end{itemize}
\textbf{M7FDU}:
\begin{itemize}
  \item Nome: Facilità di utilizzo;
  \item Descrizione: numero di click che un utente deve compiere per ottenere il voluto risultato;
  \item Scopo: limitare il numero di azioni non necessarie al fine di ottenere un'esperienza utente il più semplice e piacevole possibile.
\end{itemize}

\subsubsection{Codifica} 
    \paragraph{Descrizione e Scopo} \hfill \break
    La codifica fornisce una base comune sulla realizzazione del prodotto software, con lo scopo di poter creare codice uniforme e agevolare manutenzione ed eventuali modifiche future. \\
    I programmatori si dovranno attenere a queste norme durante le fasi di progettazione ed implementazione. \\
    Per agevolare la lettura e l'integrazione del codice, verrà utilizzato il sistema di \glossario{Continuos Integration} fornito da Github, ovvero \glossario{Github Action}: grazie a quest'ultimo, verranno
    eseguiti dei controlli nel codice che viene inviato alla repository e, in caso il codice in questione non soddisfi i requisiti di seguito riportati, verrà chiesto di eseguire un'operazione di merge con una versione corretta del codice. \\
    Verranno quindi implementate le convenzioni introdotte da PEP 8, che forniscono uno standard per lo stile da seguire nella scrittura del codice \glossario{Python}.

    \paragraph{Convenzioni per i Nomi} 
    \begin{itemize}
        \item i nomi dovranno essere univoci;
        \item dovrà essere utilizzato il PascalCase o lo snake\_case per le variabili;
        \item dovrà essere utilizzato lo snake\_case per metodi e classi. 
    \end{itemize}
        
    \paragraph{Convenzioni sulle Pratiche} \hfill \break
\newline
    \begin{tabular}{ |m{15em}|m{25em}| }
        \hline
        \textbf{Pratica}			    & \textbf{Descrizione}\\
        \hline
        \textbf{Variabili utili}		& Non definire variabili se queste non sono poi utilizzate.\\
        \hline
        \textbf{Funzioni ricorsive}		& Evitare quando possibile funzioni ricorsive.\\
        \hline
        \textbf{Delimitatori stringhe}	& Ogni stringa deve essere delimitata dal singolo apice.\\
        \hline
        \textbf{Operatori}			    & Gli operatori, quando utilizzati, dovranno essere preceduti e seguiti da spazi.\\
        \hline
        \textbf{Indentazione}			& Il codice, per una maggiore leggibilità, dovrà essere indentato con 4 spazi.\\ %poi possiamo cambiare il valore
        \hline
        \textbf{Commenti}			    & Il codice dovrà essere commentato. I commenti devono essere separati dall'identificatore di commento da uno spazio. Se il commento è inline, dovrà essere presente uno spazio prime dell'identificatore di commento.\\
        \hline
        \textbf{Complessità ciclomatica}& Viene evitato il più possibile l'annidamento dei cicli, poiché questo renderebbe le funzioni troppo onerose.\\
        \hline
\end{tabular}
    
    \paragraph{Strumenti} \hfill \break
    \begin{itemize}
  \item \textbf{\glossario{PyCharm}}: utilizzato per la scrittura di codice in linguaggio \glossario{Python};
  \item \textbf{\glossario{Visual Studio Code}}: utilizzato per la scrittura di codice in linguaggio \glossario{Python} e di codice \glossario{HTML}/\glossario{CSS};
  \item \textbf{\glossario{Flask}}: framework \glossario{Python} per la visualizzazione e creazione di semplici applicazioni web.
\end{itemize}

\paragraph{Metriche}\hfill \break
\textbf{M8CC}:
\begin{itemize}
  \item Nome: Code coverage;
  \item Descrizione: determina il numero di linee di codice che vengono validate correttamente;
  \item Scopo: aiuta ad analizzare quanto codice è verificato correttamente.
\end{itemize}
\textbf{M9NAC}:
\begin{itemize}
  \item Nome: Numero di attributi per classe;
  \item Descrizione: indica il numero di attributi dichiarati all'interno di una classe;
  \item Scopo: limitare il numero di attributi utilizzati per mantenere il codice il più semplice possibile, al fine di renderlo il più facilmente modificabile e manutentibile in futuro.
\end{itemize}
\textbf{M10PF}:
\begin{itemize}
  \item Nome: Parametri per funzione;
  \item Descrizione: indica il numero di parametri che vengono forniti ad una funzione;
  \item Scopo: limitare il numero di parametri inviati ad una singola funzione, per mantere il codice il più semplice possibile.
\end{itemize}
\textbf{M11LCF}:
\begin{itemize}
  \item Nome: Linee di codice per funzione;
  \item Descrizione: indica il numero di righe di codice all'interno di una funzione;
  \item Scopo: mantenere le funzioni il più semplici possibili, al fine di avere del codice più semplice e più facilmente gestibile.
\end{itemize}    
\textbf{M12PI}:
\begin{itemize}
  \item Nome: Profondità di innestamento;
  \item Descrizione: numero intero che, data una funzione, indica la quantità di blocchi condizionali e di cicli innestati all'interno di essa;
  \item Scopo: diminuire questo valore al fine di rendere il codice più semplice possibile e gestibile nel corso del tempo.
\end{itemize}
\textbf{M13CPC}:
\begin{itemize}
  \item Nome: Linee di commento per linee di codice;
  \item Descrizione: indica il numero di linee di commento a confronto con le effettive righe totali di codice;
  \item Scopo: il codice deve essere commentato, per permettere una comprensione più facile da membri del gruppo diversi dal creatore del codice.
\end{itemize}