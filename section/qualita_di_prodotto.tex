\section{Qualità di Prodotto}
I prodotti, in questo progetto, vengono intesi quali la documentazione e il software. Per garantire la qualità di questi prodotti, è stato scelto come riferimento lo standard ISO/IEC 9126. In questa sezione troviamo i parametri scelti dal gruppo, che quantificano il grado di raggiungimento della qualità dei prodotti. La descrizione dettagliata delle metriche viene riportata nel documento NormeDiProgetto.

\subsection{Documenti}
I documenti sono una parte importante del progetto, devono essere corretti e leggibili agli utenti.
\subsubsection{Obiettivo}
Rendere il contenuto dei documenti comprensibile agli utenti.
\subsubsection{Metriche}
\begin{center}
\renewcommand{\arraystretch}{1.8}
\begin{tabular}{ |c|c|c|c|}
	\hline
	\textbf{Codice} & \textbf{Nome} & \textbf{Valore Accettabile} & \textbf{Valore Ottimale} \\
	\hline
	M14IG & Indice di Gulpease &  $\geq 40 $ & $\geq 80 $ \\
	\hline
\end{tabular}
\end{center}

\subsection{Software}
Il software costituisce una parte fondamentale del progetto è, quindi, importante controllare la qualità di quest'ultimo.

\subsubsection{Obiettivi}
\begin{center}
	\renewcommand{\arraystretch}{1.8}
	\begin{tabular}{ |c|m{18em}|m{8em}|}
		\hline
		\textbf{Obiettivo} & \textbf{Descrizione} & \textbf{Metrica} \\
		\hline
		Funzionalità & Soddisfare i requisiti individuati perseguendo accuratezza. & M1PRR, M2PDR, M3POR \\
		\hline
		Usabilità & Facilitare l'uso del prodotto sviluppato affinché gli utenti possano usarlo per i propri scopi. & M7FDU \\
		\hline
		Manutenibilità & Facilitare le modifiche e aggiornamenti apportati al software. & M5ATC, M6PDG, M9NAC, M10PF, M11LCF, M12PI, M13CPC \\
		\hline
	\end{tabular}
\end{center}

\subsubsection{Metriche}
\begin{center}
	\renewcommand{\arraystretch}{1.8}
	\begin{tabular}{ |c|m{14em}|c|c|}
		\hline
		\textbf{Codice} & \textbf{Nome} & \textbf{Valore Accettabile} & \textbf{Valore Ottimale} \\
		\hline
		M1PRR & Percentuale di requisiti obbligatori soddisfatti & $ 100\% $ & $ 100\% $ \\
		\hline
		M2PDR & Percentuale di requisiti desiderabili soddisfatti & $ \geq 0\% $ & $ 100\% $ \\
		\hline
		M3POR & Percentuale di requisiti opzionali soddisfatti & $ \geq 0\% $ & $ 100\% $ \\
		\hline
		M5ATC & Accoppiamento tra Classe & $ \leq 5 $ & $ \leq 2 $\\
		\hline
		M6PDG & Profondità delle Gerarchie & $ \leq 5 $ & $ \leq 3 $\\
		\hline
		M7FDU & Facilità di Utilizzo & $ \leq 7 $ & $ \leq 5 $\\
		\hline
		M9NAC & Numero di Attributi per Classe & $ \leq 9 $ & $ \leq 5 $\\
		\hline
		M10PF & Parametri per Funzione & $ \leq 6 $ & $ \leq 4 $\\
		\hline
		M11LCF & Linee di Codice per Funzione & $ \leq 35 $ & $ \leq 25 $\\
		\hline
		M12PI & Profondità di Innestamento & $ \leq 4 $ & $ \leq 3 $\\
		\hline
		M13CPC & Linee di Commento per Linee di Codice & $ \geq 0.2 $ & $ \geq 0.3 $\\
		\hline
	\end{tabular}
\end{center}