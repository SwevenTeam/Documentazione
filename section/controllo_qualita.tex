\subsection{Controllo della Qualità}
Questa sezione serve a normalizzare le procedure atte a controllare la qualità del prodotto.
Tale qualità viene approfondita maggiormente nel PdQ. \newline
La qualità va verificata sia sul software che sulla documentazione. \newline
Risulta dunque necessario non solo assicurarsi che tutti i prodotti soddisfino i requisiti richiesti, 
ma anche trovare misure per verificare con oggettività una simile richiesta.
\subsubsection{Definizione di Qualità}
Per qualità si intende la capacità di superare processi di verifica e di validazione con esiti positivi. \newline 
La verifica certifica la correttezza del processo, mentre la validazione certifica il raggiungimento dei requisiti. \newline
Per ogni processo bisogna garantirne efficacia, ovvero la capacità di raggiungimento degli obiettivi 
prefissati, ed efficienza, ovvero la minimizzazione dei costi (che siano in termini di peso, di tempi o di soldi).
\subsubsection{Misure di Qualità}
Per mantenere l'oggettività nella misurazione della qualità di un ente, è necessario introdurre delle metriche.
Tali metriche saranno dipendenti dall'ente cui sono legate. \newline
Per identificarle si utilizza il seguente formato:
\begin{center}
	\textbf{M [NUMBER] [INITIALS]}
\end{center}
\renewcommand{\arraystretch}{1.8} %aumento ampiezza righe
    \begin{tabular}{ |m{7em}|m{30em}| }
        \hline
        \textbf{Nome} & \textbf{Descrizione} \\
        \hline
            M & Indica una metrica di qualità \\
        \hline
            NUMBER & Codice numerico identificativo \\
        \hline
            INITIALS & Lettere iniziali del nome \\
        \hline
    \end{tabular} \newline \newline
Per ogni metrica è bene definire su cosa si basa, lo scopo e l'eventuale formula di calcolo. Ulteriori dati sono da aggiungere
a discrezione.
\subsubsection{Classificazione dei Processi}
Per tenere sotto controllo la qualità dei processi è bene identificarli. 
Ciò viene eseguito secondo il seguente formato:
\begin{center}
    \textbf{Q\_PRC [NUMBER]}
\end{center} 
\renewcommand{\arraystretch}{1.8} %aumento ampiezza righe
    \begin{tabular}{ |m{7em}|m{30em}| }
        \hline
        \textbf{Nome} & \textbf{Descrizione} \\
        \hline
            Q\_PRC & Indica la memorizzazione dei processi secondo qualità \\
        \hline
            NUMBER & Codice numerico identificativo \\
        \hline
    \end{tabular}
\subsubsection{Classificazione dei Prodotti}
Analogamente alla sottosezione precedente, anche i prodotti sono identificati, secondo il formato:
\begin{center}
    \textbf{Q\_PRD [NUMBER]}
\end{center}
\renewcommand{\arraystretch}{1.8} %aumento ampiezza righe
    \begin{tabular}{ |m{7em}|m{30em}| }
        \hline
        \textbf{Nome} & \textbf{Descrizione} \\
        \hline
            Q\_PRD & Indica la memorizzazione dei prodotti secondo qualità \\
        \hline
            NUMBER & Codice numerico identificativo \\
        \hline
    \end{tabular}