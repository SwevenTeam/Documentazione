\section{Descrizione funzionalità}
\subsection{Presenza in sede}
Dopo essersi autenticati è possibile registrare la propria presenza in sede scrivendo "Presenza" e il chatbot comunica l'avvio dell'operazione richiedendo il nome di una sede. 
Quindi si scrive il nome, se il nome fa parte della lista sedi verrà accettato e il chatbot conclude l'operazione di registrazione presenza, altrimenti il bot comunica che la sede non è stata accettata e si invita a reinserire il nome. ---IMG--- \newline
\subsection{Consuntivazione}
Dopo essersi autenticati è possibile consuntivare le proprie ore di lavoro su un determinato progeetto, la richiesta di questa operazione si avvia tramite la parola chiave "consuntiva", quindi il chatbot avvia la consuntivazione e tramite vari messaggi di botta e risposta chiede tutte le informazioni necessarie:
\begin{itemize}
    \item inserire il codice del progetto;
    \item inserire la data di consuntivazione;
    \item inserire le ore fatturabili;
    \item inserire le ore di viaggio;
    \item inserire le ore di viaggio fatturabili;
    \item inserire la sede;
\end{itemize}
Se si inserisce un formato o una parola non valida chiede di reinserire, infine riepiloga tutte le informazioni ricevute e chiede conferma se procedere o se si vuole modificare qualche dato o se si vuole annullare la consuntivazione. ---IMG---

\newpage