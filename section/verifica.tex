\subsection{Verifica}
Il processo di verifica è la fase obbligatorio per il procedimento alla validazione , applicata ogni volta che c’è un rischio di errore.
\subsubsection{Analisi}
Ci sono due tipologie di analisi del prodotto software disponibile, vengono effettuate entrambi se il gruppo ritiene necessario:
\begin{itemize}
\item Analisi statico: non richiede l’oggetto di verifica, viene studiato la documentazione e il codice non eseguibile, accerta conformità a regola assenza di 			difetti, presenza di proprietà desiderate descritte nel Piano di Qualifica. 
\item Analisi dinamica: serve l’oggetto testabile, vengono effettuate dei test su di essa.
\end{itemize}

\subsubsection{Metod di lettura}
Vengono definite le due modalità di lettura per l’analisi statico:
\begin{itemize}
\item Walkthough: la lettura parte da zero, analizza tutto il documento, il verificatore deve presuppore di cercare i difetti del codice o del documento.
\item Inspection: viene fatto la lettura mirata, con una lista di checklist.
\end{itemize}

\subsubsection{Test}
Il test serve per testare che il prodotto software funziona effettivamente come desiderato, serve l’oggetto testabile, viene effettuato qualvolta esiste un minimo componente funzionante testabile.
Il test deve essere: 
\begin{itemize}
\item Ripetibile: il test deve dare lo stesso risultato con lo stesso input. 
\item Automaticizzabile: il test deve essere eseguibile da un processo automatizzato, minimizzare il costo delle persone e velocizzare il processo.
\end{itemize}
Il test è composto da tre elementi:
\begin{itemize}
\item Ambiente d’esecuzione: hardware/software in cui è stato eseguito il test e lo stato iniziale del sistema.
\item Attesa: input richiesti per l’esecuzione del test , e il suo corrispondente output atteso.
\item Procedura: il procedimento in cui l’oggetto viene analizzato.
\end{itemize}

\paragraph{Test di unità}
Un test automatico effettuato sul minimo componente che ha bisogno di un test specifico.\newline
Per l’identificazione si usa il seguente formato
\begin{center}
    \textbf{T\_U [NUMBER]}
\end{center}
\renewcommand{\arraystretch}{1.8} 
 \begin{tabular}{ |m{7em}|m{30em}| }
        \hline
        \textbf{Nome} & \textbf{Descrizione} \\
        \hline
            T\_U & Indica il test di unità \\
        \hline
            NUMBER & Codice Numerico Identificativo \\
        \hline
 \end{tabular}

\paragraph{Test di integrazione}
Test da eseguire quando viene effettuato un unione di componenti verificati, assicurare che l’integrazione avviene correttamente.\newline
Per l’identificazione si usa il seguente formato
\begin{center}
    \textbf{T\_I [NUMBER]}
\end{center}
\renewcommand{\arraystretch}{1.8} 
 \begin{tabular}{ |m{7em}|m{30em}| }
        \hline
        \textbf{Nome} & \textbf{Descrizione} \\
        \hline
            T\_I & Indica il test di integrazione \\
        \hline
            NUMBER & Codice Numerico Identificativo \\
        \hline
 \end{tabular}

\paragraph{Test di sistema}
Il test da eseguire alla fine del programma, agisce su tutti i requisiti, testare tutti i requisiti sono stati sodisfati.\newline
Per l’identificazione si usa il seguente formato
\begin{center}
    \textbf{T\_S [NUMBER]}
\end{center}
\renewcommand{\arraystretch}{1.8} 
 \begin{tabular}{ |m{7em}|m{30em}| }
        \hline
        \textbf{Nome} & \textbf{Descrizione} \\
        \hline
            T\_I & Indica il test di sistema \\
        \hline
            NUMBER & Codice Numerico Identificativo \\
        \hline
 \end{tabular}

\paragraph{Test di regressione}
Test da eseguire dopo ogni aggiustamento di un errore, per evitare che la modifica comporta un errore in un’altra parte, dove già effettuato la verifica.\newline
Per l’identificazione si usa il seguente formato
\begin{center}
    \textbf{T\_R [NUMBER]}
\end{center}
\renewcommand{\arraystretch}{1.8} 
 \begin{tabular}{ |m{7em}|m{30em}| }
        \hline
        \textbf{Nome} & \textbf{Descrizione} \\
        \hline
            T\_I & Indica il test di regressione \\
        \hline
            NUMBER & Codice Numerico Identificativo \\
        \hline
 \end{tabular}
