\subsection{Verifica}
Il processo di verifica è la fase obbligatoria per il procedimento alla validazione, applicata ogni volta 
che c’è un rischio di errore.
\subsubsection{Analisi}
Ci sono due tipologie di analisi del prodotto software disponibile; se il gruppo lo ritiene necessario 
vengono effettuate entrambe; esse sono:
\begin{itemize}
\item \textbf{Analisi Statica}: non richiede l’oggetto di verifica, viene studiata la documentazione e il 
codice non eseguibile; accerta conformità alle regole, assenza di difetti e presenza di proprietà desiderate 
descritte nel Piano di Qualifica;
\item \textbf{Analisi Dinamica}: richiede l’oggetto testabile, effettua test su di esso.
\end{itemize}

\subsubsection{Metodi di Lettura}
Vengono definite le due modalità di lettura per l’analisi statica:
\begin{itemize}
\item Walkthough: la lettura parte da zero e analizza tutto il documento; il verificatore deve 
presuppore di cercare i difetti del codice o del documento;
\item Inspection: viene fatta la lettura mirata, con una checklist.
\end{itemize}

\subsubsection{Test}
Il test serve per confermare che il prodotto software funzioni effettivamente come desiderato; per questo 
motivo serve l’oggetto testabile; viene effettuato ogniqualvolta esiste un minimo componente funzionante 
testabile. Il test deve essere: 
\begin{itemize}
\item Ripetibile: il test deve dare lo stesso risultato ad ogni inserimento del medesimo input;
\item Automatizzabile: il test deve essere eseguibile da un processo automatizzato, deve minimizzare il costo 
delle persone e deve velocizzare il processo.
\end{itemize}
Il test è caratterizzato da tre elementi:
\begin{itemize}
\item Ambiente d’esecuzione: hardware/software, in cui viene eseguito il test, e stato iniziale del sistema;
\item Attesa: input richiesti per l’esecuzione del test e corrispettivo output;
\item Procedura: il procedimento in cui l’oggetto viene analizzato.
\end{itemize}

\paragraph{Test di Unità} \hfill \break
Un test automatico effettuato sul minimo componente che ha bisogno di un test specifico.\newline
Per l’identificazione si usa il seguente formato:
\begin{center}
    \textbf{T\_U [NUMBER]}
\end{center}
\renewcommand{\arraystretch}{1.8} 
 \begin{tabular}{ |m{7em}|m{30em}| }
        \hline 
        \textbf{Nome} & \textbf{Descrizione} \\
        \hline
            T\_U & Indica il test di unità \\
        \hline
            NUMBER & Codice numerico identificativo \\
        \hline
 \end{tabular}

\paragraph{Test di Integrazione}  \hfill \break
Test da eseguire quando viene effettuata un'unione di componenti verificati; assicura che l’integrazione 
avvenga correttamente.
Per l’identificazione si usa il seguente formato:
\begin{center}
    \textbf{T\_I [NUMBER]}
\end{center}
\renewcommand{\arraystretch}{1.8} 
 \begin{tabular}{ |m{7em}|m{30em}| }
        \hline
        \textbf{Nome} & \textbf{Descrizione} \\
        \hline
            T\_I & Indica il test di integrazione \\
        \hline
            NUMBER & Codice numerico identificativo \\
        \hline
 \end{tabular}

\paragraph{Test di Sistema}  \hfill \break
Il test da eseguire alla fine del programma, quello che agisce su tutti i requisiti; verifica che tutti i 
requisiti siano stati sodisfati.
Per l’identificazione si usa il seguente formato:
\begin{center}
    \textbf{T\_S [NUMBER]}
\end{center}
\renewcommand{\arraystretch}{1.8} 
 \begin{tabular}{ |m{7em}|m{30em}| }
        \hline
        \textbf{Nome} & \textbf{Descrizione} \\
        \hline
            T\_I & Indica il test di sistema \\
        \hline
            NUMBER & Codice numerico identificativo \\
        \hline
 \end{tabular}

\paragraph{Test di Regressione}  \hfill \break
Test da eseguire dopo ogni aggiustamento di un errore, per evitare che la modifica comporti un errore in 
un’altra parte dove sia stata già effettuata la verifica.\newline
Per l’identificazione si usa il seguente formato:
\begin{center}
    \textbf{T\_R [NUMBER]}
\end{center}
\renewcommand{\arraystretch}{1.8} 
 \begin{tabular}{ |m{7em}|m{30em}| }
        \hline
        \textbf{Nome} & \textbf{Descrizione} \\
        \hline
            T\_I & Indica il test di regressione \\
        \hline
            NUMBER & Codice numerico identificativo \\
        \hline
 \end{tabular}