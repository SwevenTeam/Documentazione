\subsection{Descrizione}
Data la scarsa esperienza pregressa da parte dei componenti del gruppo nella gestione di un progetto, si è deciso di gestire l'organizzazione nella maniera più efficacie ed efficiente possibile adottando un modello \glossario{agile} per lo sviluppo del progetto. In particolare si è deciso di seguire le linee guida del modello incrementale, il quale facilita l'analisi e la stesura dei requisiti. 
\subsection{Modello Incrementale}
Il modello incrementale prevede una serie di rilasci multipli e successivi, ciascuno dei quali è costituito dai seguenti passi: 
\begin{itemize}
    \item analisi requisiti;
    \item implementazione;
    \item test;
    \item valutazione.
\end{itemize}
\subsubsection{Vantaggi del Modello Incrementale}
L'adozione di un modello ciclico come quello incrementale permette di usufruire di una serie di vantaggi: 
\begin{itemize}
    \item ogni incremento permette di avere indicazioni utili all'incremento successivo;
    \item il rischio di fallimento viene ridotto ad ogni incremento;
    \item gli errori vengono limitati in quanto la possibilità che si verifichino è limitata all'interno del singolo incremento, infatti al termine dello stesso segue una fase di verifica;
    \item le funzionalità primarie hanno priorità più elevata permettendo di avere fin da subito un prototipo funzionante che il proponente può valutare.
\end{itemize}
\newpage