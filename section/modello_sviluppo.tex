\subsection{Descrizione}
Data la scarsa esperienza pregressa da parte dei componenti del gruppo nella gestione di un progetto, si è deciso di gestire l'organizzazione nella maniera più efficacie ed efficiente possibile adottando un modello AGILE\textsubscript{G} per lo sviluppo del progetto. In particolare si è deciso di seguire le linee guida del modello incrementale, il quale facilita l'analisi e la stesura dei requisiti. 
\subsection{Modello Incrementale}
Il modello incrementale prevede una serie di rilasci multipli e successivi, ciascuno dei quali è costituito dai seguenti passi: 
\begin{itemize}
    \item Analisi requisiti
    \item Implementazione
    \item Test 
    \item Valutazione
\end{itemize}
\subsubsection{Vantaggi Modello Incrementale}
L'adozione di un modello ciclico come quello incrementale permette di usufruire di una serie di vantaggi: 
\begin{itemize}
    \item Ogni incremento permette di avere indicazioni utili all'incremento successivo.
    \item Rischio di fallimento ridotto ad ogni incremento.
    \item Errori vengono limitati in quanto la possibilità che si verifichino è limitata all'interno del singolo incremento infatti al termine dello stesso segue una fase di verifica.
    \item Funzionalità primarie hanno la priorità più elevata permettendo di avere fin da subito un prototipo funzionante che il proponente può valutare.
\end{itemize}
\subsection{Incrementi individuati}
Di seguito viene proposta una tabella che riassume gli incrementi che sono stati individuali per lo svolgimento del progetto. Ad ogni fase vengono associati i rispettivi requisiti da soddisfare. 

\begin{center}
	\renewcommand{\arraystretch}{1.8} %aumento ampiezza righe
	\begin{tabular}{ |c|m{12em}|m{7em}| }
	\hline
	\textbf{Incremento} & \textbf{Descrizione} &  \textbf{Requisiti} \\ %Aggiungere le nuove righe sopra la prima
	\hline 
    1 & & \\
    \hline
    2 & & \\

    \hline
	\end{tabular}
\end{center}
\newpage