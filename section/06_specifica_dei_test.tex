\section{Specifica dei Test}
In questa parte vengono riportati i test da implementare allo scopo
di soddisfare i requisiti individuati e la corretta esecuzione del prodotto.
I test si suddividono in:
\begin{itemize}
    \item Test di unità
    \item Test di integrazione
    \item Test di sistema
\end{itemize}
I codici identificativi delle tipologie di test sono definiti nel documento Norme di Progetto {\docVersionNdP}.
Nelle tabelle vengono utilizzate ulteriori sigle, di seguito viene descritto il loro significato:
\begin{itemize}
    \item Stato del test:
    \begin{itemize}
        \item NI non implementato;
        \item I implementato;
    \end{itemize}
    \item Esito del test:
    \begin{itemize}
        \item NS non superato;
        \item S superato.
    \end{itemize}
\end{itemize}

\subsection{Test di Unità}
Questi test servono per verificare il funzionamento di singoli elementi dell'applicazione.  \\
Vengono suddivisi in cartelle \textit{Test} e contengono dei file \textit{test\_NomeElementoTestato}. \\
Ogni modulo utilizzato ha il suo corrispettivo file \textit{test\_}, ad eccezione del file app.py.
\begin{center}
  \renewcommand{\arraystretch}{1.8}
  \begin{tabular}{ |m{3em}|m{23em}|m{3em}|m{3em}| }
      \hline
      \textbf{Codice} & \textbf{Descrizione}  & \textbf{Stato} & \textbf{Esito}\\
      \hline
      T\_U1 & Controllo \textit{State} del Client alla sua inizializzazione & I & S \\
      \hline
      T\_U2 & Controllo valore Api del Client alla sua inizializzazione & I & S \\
      \hline
      T\_U3 & Controllo \textit{State} del Client dopo aggiornamento del valore & I & S \\
      \hline
      T\_U4 & Controllo valore Api del Client dopo aggiornamento del valore & I & S \\
      \hline
      T\_U5 & Richiesta risposta dal server senza aver effettuato il login & I & S \\
      \hline
      T\_U6 & Richiesta risposta dal server senza possibili adapter & I & S \\
      \hline
      T\_U7 & Richiesta risposta dal server con multipli adapter & I & S \\
      \hline
      T\_U8 & Reset \textit{State} cliente dopo risposta & I & S \\
      \hline
  \end{tabular}
  \newpage
  \begin{tabular}{ |m{3em}|m{23em}|m{3em}|m{3em}| }
    \hline
    \textbf{Codice} & \textbf{Descrizione}  & \textbf{Stato} & \textbf{Esito}\\
    \hline
      T\_U9 & Controllo se un nuovo \textit{State} inizializzato non contenga valori  & I & S \\
      \hline
      % Activity
      T\_U10 & Controllo se un nuovo \textit{State\_Activity} inizializzato non contenga valori  & I & S\\
      \hline
      T\_U11 & Controllo se uno \textit{State\_Activity} modifichi i propri dati dopo un'aggiunta  & I & S \\
      \hline
      T\_U12 & Controllo se uno \textit{State\_Activity} non modifichi i propri dati dopo un'aggiunta nel caso di indice scorretto  & I & S \\
      \hline
      % Gate
      T\_U13 &  Controllo se un nuovo \textit{State\_Gate} inizializzato non contenga valori  & I & S\\
      \hline
      T\_U14 &  Controllo se uno \textit{State\_Gate} modifichi i propri dati dopo un'aggiunta  & I & S \\
      \hline
      T\_U15 & Controllo se uno \textit{State\_Gate} non modifichi i propri dati dopo un'aggiunta nel caso di indice scorretto  & I & S \\
      \hline
      % Get Activity
      T\_U16 &  Controllo se un nuovo \textit{State\_Get\_Activity} inizializzato non contenga valori & I & S \\
      \hline
      % Login
      T\_U17 & Controllo se un nuovo \textit{State\_Login} inizializzato non contenga valori  & I & S\\
      \hline
      T\_U18 & Controllo se uno \textit{State\_Login} modifichi i propri dati dopo un'aggiunta  & I & S \\
      \hline
      T\_U19 & Controllo se uno \textit{State\_Login} non modifichi i propri dati dopo un'aggiunta nel caso di indice scorretto  & I & S \\
      \hline
      % Null
      T\_U20 & Controllo se un nuovo \textit{State\_Null} inizializzato non contenga valori  & I & S\\
      \hline
      % Presence
      T\_U21 & Controllo se un nuovo \textit{State\_Presence} inizializzato non contenga valori & I & S\\
      \hline
      T\_U22 & Controllo se uno \textit{State\_Presence} modifichi i propri dati dopo un'aggiunta  & I & S \\
      \hline
      T\_U23 & Controllo se uno \textit{State\_Presence} non modifichi i propri dati dopo un'aggiunta nel caso di indice scorretto   & I & S\\
      \hline
    \end{tabular}
    \newpage
    \begin{tabular}{ |m{3em}|m{23em}|m{3em}|m{3em}| }
      \hline
      \textbf{Codice} & \textbf{Descrizione}  & \textbf{Stato} & \textbf{Esito}\\
      \hline
      % Project Creation
      T\_U24 & Controllo se un nuovo \textit{State\_Project\_Creation} inizializzato non contenga valori & I & S \\
      \hline
      T\_U25 & Controllo se uno \textit{State\_Project\_Creation} modifichi i propri dati dopo un'aggiunta & I & S \\
      \hline
      T\_U26 & Controllo se uno \textit{State\_Project\_Creation} non modifichi i propri dati dopo un'aggiunta nel caso di indice scorretto & I & S \\
      % Adapter
      \hline
      T\_U27 & Corretta Attivazione Adapter & I & S   \\
      \hline
      T\_U28 & Errore Attivazione Adapter per Input Scorretto & I & S  \\
      % Adapter Activity
      \hline
      T\_U29 & Corretta Attivazione \textit{Adapter\_Activity} & I & S    \\
      \hline
      T\_U30 & Errore Attivazione \textit{Adapter\_Activity} & I & S   \\
      % Adapter Gate
      \hline
      T\_U31 & Corretta Attivazione \textit{Adapter\_Gate} & I & S  \\
      \hline
      T\_U32 & Errore Attivazione \textit{Adapter\_Gate} & I & S   \\
      % Adapter Get Activity
      \hline
      T\_U33 & Corretta Attivazione \textit{Adapter\_Get\_Activity} & I & S   \\
      \hline
      T\_U34 & Errore Attivazione \textit{Adapter\_Activity} & I & S   \\
      % Adapter Login
      \hline
      T\_U35 & Corretta Attivazione \textit{Adapter\_Login} & I & S  \\
      \hline
      T\_U36 & Errore Attivazione \textit{Adapter\_Login} & I & S   \\
      % Adapter Logout
      \hline
      T\_U37 & Corretta Attivazione \textit{Adapter\_Logout} & I & S  \\
      \hline
      T\_U38 & Errore Attivazione \textit{Adapter\_Logout} & I & S  \\
      % Adapter Presence
      \hline
      T\_U39 & Corretta Attivazione \textit{Adapter\_Presence} & I & S   \\
      \hline
      T\_U40 & Errore Attivazione \textit{Adapter\_Presence} & I & S \\
      \hline
    \end{tabular}
    \newpage
    \begin{tabular}{ |m{3em}|m{23em}|m{3em}|m{3em}| }
      \hline
      \textbf{Codice} & \textbf{Descrizione}  & \textbf{Stato} & \textbf{Esito}\\
      \hline
      T\_U41 & Errore Sede \textit{Adapter\_Presence} non corretta  & I & S \\
      % Adapter Project Creation
      \hline 
      T\_U42 & Corretta Attivazione \textit{Adapter\_Project\_Creation } & I & S\\
      \hline
      T\_U43 & Errore Attivazione \textit{Adapter\_Project\_Creation } & I & S \\
      % Adapter Undo
      \hline
      T\_U44 & Corretta Attivazione \textit{Adapter\_Undo } & I & S \\
      \hline
      T\_U45 & Errore Attivazione \textit{Adapter\_Undo } & I & S\\
      \hline
      % Request Activity
      T\_U46 & Controllo se \textit{Request\_Activity} è pronto a soddisfare la richiesta di \textit{Adapter\_Activity}  & I & S \\
      \hline
      T\_U47 & Controllo se \textit{Request\_Activity} non è pronto a soddisfare la richiesta di \textit{Adapter\_Activity}  & I & S \\
      \hline
      % Request Gate
      T\_U48 & Controllo se \textit{Request\_Gate} è pronto a soddisfare la richiesta di \textit{Adapter\_Gate}  & I & S\\
      \hline
      T\_U49 & Controllo se \textit{Request\_Gate} non è pronto a soddisfare la richiesta di \textit{Adapter\_Gate}  & I & S \\
      \hline
      % Request Presence
      T\_U50 & Controllo se \textit{Request\_Presence} è pronto a soddisfare la richiesta di \textit{Adapter\_Presence}  & I & S\\
      \hline
      T\_U51 &Controllo se \textit{Request\_Presence} non è pronto a soddisfare la richiesta di \textit{Adapter\_Presence}  & I & S\\
      \hline
      % Request Project Creation
      T\_U52 & Controllo se \textit{Request\_Project\_Creation} è pronto a soddisfare la richiesta di \textit{Adapter\_Project\_Creation}  & I & S\\
      \hline
      T\_U53 & Controllo se \textit{Request\_Project\_Creation} non è pronto a soddisfare la richiesta di \textit{Adapter\_Project\_Creation}  & I & S \\
      \hline
\end{tabular}
\end{center}
\newpage
\subsubsection{Tracciamento - Test di Unità}
\begin{center}
  \renewcommand{\arraystretch}{1.8}
  \begin{tabular}{|m{6em}|m{33em}|}
      \hline
      \textbf{Codice} & \textbf{Modulo Testato} \\
      \hline
      T\_U1 &\textbackslash server\textbackslash Test\textbackslash test\_Client\textbackslash test\_Client\_State \\
      \hline
      T\_U2 &\textbackslash server\textbackslash Test\textbackslash test\_Client\textbackslash test\_Client\_Api \\
      \hline
      T\_U3 &\textbackslash server\textbackslash Test\textbackslash test\_Client\textbackslash test\_Client\_Upgrade\_State \\
      \hline
      T\_U4 &\textbackslash server\textbackslash Test\textbackslash test\_Client\textbackslash test\_Client\_Upgrade\_State \\
      \hline
      T\_U5 &\textbackslash server\textbackslash Test\textbackslash test\_Server\textbackslash test\_Server\_No\_Login \\
      \hline
      T\_U6 &\textbackslash server\textbackslash Test\textbackslash test\_Server\textbackslash test\_Server\_No\_Adapter \\
      \hline
      T\_U7 &\textbackslash server\textbackslash Test\textbackslash test\_Server\textbackslash test\_Double\_Adapter \\
      \hline
      T\_U8 &\textbackslash server\textbackslash Test\textbackslash test\_Server\textbackslash test\_Adapter\_Update \\
      \hline
      T\_U9 &\textbackslash server\textbackslash State\textbackslash Test\textbackslash test\_State\textbackslash test\_State \\
      \hline
      % Activity
      T\_U10 &\textbackslash server\textbackslash State\textbackslash Test\textbackslash test\_State\_Activity\textbackslash test\_State\_Activity \\
      \hline
      T\_U11 &\textbackslash server\textbackslash State\textbackslash Test\textbackslash test\_State\_Activity\textbackslash test\_State\_Activity\_Correct \\
      \hline
      T\_U12 &\textbackslash server\textbackslash State\textbackslash Test\textbackslash test\_State\_Activity\textbackslash test\_State\_Activity\_Incorrect \\
      \hline
      % Gate
      T\_U13 &\textbackslash server\textbackslash State\textbackslash Test\textbackslash test\_State\_Gate\textbackslash test\_State\_Gate \\
      \hline
      T\_U14 &\textbackslash server\textbackslash State\textbackslash Test\textbackslash test\_State\_Gate\textbackslash test\_State\_Gate\_Correct \\
      \hline
      T\_U15 &\textbackslash server\textbackslash State\textbackslash Test\textbackslash test\_State\_Gate\textbackslash test\_State\_Gate\_Incorrect \\
      \hline
      % Get Activity
      T\_U16 &\textbackslash server\textbackslash State\textbackslash Test\textbackslash test\_State\_Get\_Activity\textbackslash test\_State\_Get\_Activity \\
      \hline
      % Login
      T\_U17 &\textbackslash server\textbackslash State\textbackslash Test\textbackslash test\_State\_Login\textbackslash test\_State\_Login \\
      \hline
      T\_U18 &\textbackslash server\textbackslash State\textbackslash Test\textbackslash test\_State\_Login\textbackslash test\_State\_Login\_Data \\
      \hline
      T\_U19 &\textbackslash server\textbackslash State\textbackslash Test\textbackslash test\_State\_Login\textbackslash test\_State\_Login\_Error \\
      \hline
      % Null
      T\_U20 &\textbackslash server\textbackslash State\textbackslash Test\textbackslash test\_State\_Null\textbackslash test\_State\_Null \\
      \hline
      % Presence
      T\_U21 &\textbackslash server\textbackslash State\textbackslash Test\textbackslash test\_State\_Presence\textbackslash test\_State\_Presence  \\
      \hline
      T\_U22 &\textbackslash server\textbackslash State\textbackslash Test\textbackslash test\_State\_Presence\textbackslash test\_State\_Presence\_Correct \\
      \hline
      T\_U23 &\textbackslash server\textbackslash State\textbackslash Test\textbackslash test\_State\_Presence\textbackslash test\_State\_Presence\_Incorrect  \\
      \hline
    \end{tabular}
    \begin{tabular}{|m{6em}|m{33em}|}
      \hline
      \textbf{Codice} & \textbf{Modulo Testato} \\
      \hline
      % Project Creation
      T\_U24 &\textbackslash server\textbackslash State\textbackslash Test\textbackslash test\_State\_Project\_Creation \textbackslash test\_State\_Project\_Creation \\
      \hline
      T\_U25 &\textbackslash server\textbackslash State\textbackslash Test\textbackslash test\_State\_Project\_Creation \textbackslash test\_State\_Project\_Creation\_Correct \\
      \hline
      T\_U26 &\textbackslash server\textbackslash State\textbackslash Test\textbackslash test\_State\_Project\_Creation \textbackslash test\_State\_Project\_Creation\_Incorrect \\
      % Adapter
      \hline
      T\_U27 &\textbackslash server\textbackslash Adapter\textbackslash Test\textbackslash test\_Adapter\textbackslash test\_Adapter \\
      \hline
      T\_U28 &\textbackslash server\textbackslash Adapter\textbackslash Test\textbackslash test\_Adapter\textbackslash test\_Adapter\_Error \\
      % Adapter Activity
      \hline
      T\_U29 &\textbackslash server\textbackslash Adapter\textbackslash Test\textbackslash test\_Adapter\_Activity\textbackslash test\_Adapter\_Activity\_Activate \\
      \hline
      T\_U30 &\textbackslash server\textbackslash Adapter\textbackslash Test\textbackslash test\_Adapter\_Activity\textbackslash test\_Adapter\_Activity\_Error \\
      % Adapter Gate
      \hline

      T\_U31 &\textbackslash server\textbackslash Adapter\textbackslash Test\textbackslash test\_Adapter\_Gate\textbackslash test\_Adapter\_Gate\_Activate \\
      \hline
      T\_U32 &\textbackslash server\textbackslash Adapter\textbackslash Test\textbackslash test\_Adapter\_Gate\textbackslash test\_Adapter\_Gate\_Error \\
      % Adapter Get Activity
      \hline
      T\_U33 &\textbackslash server\textbackslash Adapter\textbackslash Test\textbackslash test\_Adapter\_Get\_Activity  \textbackslash test\_Adapter\_Get\_Activity\_Activate \\
      \hline
      T\_U34 &\textbackslash server\textbackslash Adapter\textbackslash Test\textbackslash test\_Adapter\_Get\_Activity \textbackslash test\_Adapter\_Get\_Activity\_Code\_Incorrect\_Number \\
      % Adapter Login
      \hline
      T\_U35 &\textbackslash server\textbackslash Adapter\textbackslash Test\textbackslash test\_Adapter\_Login\textbackslash test\_Adapter\_Login\_Activate \\
      \hline
      T\_U36 &\textbackslash server\textbackslash Adapter\textbackslash Test\textbackslash test\_Adapter\_Login \newline \textbackslash test\_Adapter\_Login\_Already\_Logged \\
      % Adapter Login
      \hline
      T\_U37 &\textbackslash server\textbackslash Adapter\textbackslash Test\textbackslash test\_Adapter\_Logout\textbackslash test\_Adapter\_Logout\_Correct \\
      \hline
      T\_U38 &\textbackslash server\textbackslash Adapter\textbackslash Test\textbackslash test\_Adapter\_Logout\textbackslash test\_Adapter\_Logout\_Incorrect \\
      \hline
      % Adapter Logout
      T\_U39 &\textbackslash server\textbackslash Adapter\textbackslash Test\textbackslash test\_Adapter\_Presence \newline \textbackslash test\_Adapter\_Presence\_Activate \\
      \hline
      T\_U40 &\textbackslash server\textbackslash Adapter\textbackslash Test\textbackslash test\_Adapter\_Presence \newline \textbackslash test\_Adapter\_Presence\_Location\_Correct \\
      \hline
      T\_U41 &\textbackslash server\textbackslash Adapter\textbackslash Test\textbackslash test\_Adapter\_Presence \newline \textbackslash test\_Adapter\_Presence\_Location\_Incorrect \\
      % Adapter Project Creation
      \hline 
    \end{tabular}
    \begin{tabular}{|m{6em}|m{33em}|}
      \hline
      \textbf{Codice} & \textbf{Modulo Testato} \\
      \hline
      T\_U42 &\textbackslash server\textbackslash Adapter\textbackslash Test\textbackslash test\_Adapter\_Project\_Creation \textbackslash test\_Adapter\_Creation\_Activate \\
      \hline
      T\_U43 &\textbackslash server\textbackslash Adapter\textbackslash Test\textbackslash test\_Adapter\_Project\_Creation \textbackslash test\_Adapter\_Creation\_Incorrect \\
      % Adapter Undo
      \hline
      T\_U44 &\textbackslash server\textbackslash Adapter\textbackslash Test\textbackslash test\_Adapter\_Undo\textbackslash test\_Adapte\_Undo\_Presence \\
      \hline
      T\_U45 &\textbackslash server\textbackslash Adapter\textbackslash Test\textbackslash test\_Adapter\_Undo\textbackslash test\_output\_No\_Operation \\
      \hline
      % Request Activity
      T\_U46 &\textbackslash server\textbackslash Request\textbackslash Test\textbackslash test\_Request\_Activity\textbackslash test\_Request\_Activity\_isReady \\
      \hline
      T\_U47 &\textbackslash server\textbackslash Request\textbackslash Test\textbackslash test\_Request\_Activity \newline \textbackslash test\_Request\_Activity\_isReady\_Error\_Not\_Ready\\
      \hline
      % Request Gate
      T\_U48 &\textbackslash server\textbackslash Request\textbackslash Test\textbackslash test\_Request\_Gate\textbackslash test\_Request\_Gate\_isReady \\
      \hline
      T\_U49 &\textbackslash server\textbackslash Request\textbackslash Test\textbackslash test\_Request\_Gate\textbackslash test\_Request\_Gate\_Error \\
      \hline
      % Request Presence
      T\_U50 &\textbackslash server\textbackslash Request\textbackslash Test\textbackslash test\_Request\_Presence\newline \textbackslash test\_Request\_Presence\_isReady \\
      \hline
      T\_U51 &\textbackslash server\textbackslash Request\textbackslash Test\textbackslash test\_Request\_Presence\newline \textbackslash test\_Request\_Presence\_isReady\_Error\_State \\
      \hline
      % Request Project Creation
      T\_U52 &\textbackslash server\textbackslash Request\textbackslash Test\textbackslash test\_Request\_Project\_Creation\newline \textbackslash test\_Request\_Project\_Creation\_isReady \\
      \hline
      T\_U53 &\textbackslash server\textbackslash Request\textbackslash Test\textbackslash test\_Request\_Project\_Creation\newline \textbackslash test\_Request\_Project\_Creation\_isReady\_Error\_Not\_Ready \\
      \hline
  \end{tabular}
\end{center}

\subsection{Test di Integrazione}
Questi test servono per verificare il corretto funzionamento di diversi componenti del programma quando vengono utilizzati in maniera congiunta. \\
Questi risultano essere fondamentali per il tipo di applicazione che si sta creando, poiché diversi elementi dell'applicazione devono utilizzare moduli diversi per poter funzionare correttamente.
\newline
\begin{tabular}{ |m{3em}|m{23em}|m{3em}|m{3em}| }
  \hline
  \textbf{Codice} & \textbf{Descrizione}  & \textbf{Stato} & \textbf{Esito}\\
  \hline
  T\_I1 & Corretta Integrazione tra Client e Server & I & S \\
  \hline
  T\_I2 & Corretta Chiamata da Adapter\_Activity alle funzioni di Request\_Activity & I & S \\
  \hline
  T\_I3 & Corretta Chiamata da Adapter\_Gate alle funzioni di Request\_Gate & I & S \\
  \hline
  T\_I4 & Corretta Chiamata da Adapter\_Get\_Activity alle funzioni di Request\_Get\_activity & I & S \\
  \hline
  T\_I5 & Corretta Chiamata da Adapter\_Presence alle funzioni di Request\_Presence & I & S \\
  \hline
  T\_I6 & Corretta Chiamata da Adapter\_Project\_Creation alle funzioni di Request\_Project\_Creation & I & S \\
  \hline
  T\_I7 & Corretta Chiamata da Adapter alle funzioni di Util Request & I & S \\
  \hline

\end{tabular}
\subsubsection{Tracciamento - Test di Integrazione}
\renewcommand{\arraystretch}{1.8}
\begin{tabular}{|m{6em}|m{33em}|}
    \hline
    \textbf{Codice} & \textbf{Modulo Testato} \\
    \hline
    T\_I1 & \textbackslash server\textbackslash Test\textbackslash test\_Server\\
    \hline
    T\_I2 & \textbackslash server\textbackslash Adapter\textbackslash Test\textbackslash test\_Adapter\_Activity  \\
    \hline
    T\_I3 & \textbackslash server\textbackslash Adapter\textbackslash Test\textbackslash test\_Adapter\_Gate \\
    \hline
    T\_I4 & \textbackslash server\textbackslash Adapter\textbackslash Test\textbackslash test\_Adapter\_Get\_Activity \\
    \hline
    T\_I5 & \textbackslash server\textbackslash Adapter\textbackslash Test\textbackslash test\_Adapter\_Presence  \\
    \hline
    T\_I6 & \textbackslash server\textbackslash Adapter\textbackslash Test\textbackslash test\_Adapter\_Project\_Creation  \\
    \hline 
    T\_I7 &  \textbackslash server\textbackslash Adapter\textbackslash Test\textbackslash test\_Adapter \\
    \hline 
\end{tabular}

\subsection{Test di Sistema}
Questi test servono per verificare che il funzionamento complessivo dell'applicazione rispetti i requisiti stabiliti nell'Analisi dei requisiti {\docVersionAdR}.
\begin{center}
    \renewcommand{\arraystretch}{1.8}
    \begin{tabular}{ |m{3em}|m{23em}|m{3em}|m{3em}| }
        \hline
        \textbf{Codice} & \textbf{Descrizione} & \textbf{Stato} & \textbf{Esito} \\
        \hline
        T\_S1 & Controllare che il \glossario{chatbot} possa rispondere a messaggi testuali. & I & S \\
        \hline
        T\_S2 & Controllare che venga eseguita la trascrizione di un messaggio vocale. & NI & - \\
        \hline
        T\_S3 & Controllare che l'utente riesca ad autenticarsi tramite \glossario{token}. & I & S \\
        \hline
        T\_S4 & Controllare che l'utente possa inserire, in maniera testuale, un token per autenticarsi. & NI & - \\
        \hline
        T\_S5 & Controllare che il \glossario{chatbot} sia in grado di riconoscere un \glossario{token} non valido. & I & S \\
        \hline
        T\_S6 & Controllare che se l'utente fornisce un \glossario{token} non valido venga visualizzato un messaggio di errore. & I & S \\
        \hline
        T\_S7 & Controllare che l'utente possa fornire \glossario{token} diversi per l'autenticazione. & NI & - \\
        \hline
        T\_S8 & Controllare che l'utente possa registrare la propria presenza in sede. & I & S \\
        \hline
        T\_S9 & Controllare che, durante il \glossario{check-in}, la sede inserita dall'utente sia valida. & I & S \\
        \hline
        T\_S10 & Controllare che, durante il \glossario{check-in}, l'utente venga informato se la sede fornita non è valida. & I & S \\
        \hline
        T\_S11 & Controllare che l'utente possa inserire un'\glossario{attività} nel \glossario{sistema emt}. & I & S \\
        \hline
        T\_S12 & Controllare che l'utente possa specificare il tipo di \glossario{attività} da inserire nel \glossario{sistema emt}. & NI & - \\
        \hline
        T\_S13 & Controllare che venga inviato un messaggio di errore se il tipo di \glossario{attività} non è valido nel \glossario{sistema emt}. & NI & - \\
        \hline
        T\_S14 & Controllare che l'utente possa specificare il numero di ore da consuntivare. & I & S \\
        \hline
        T\_S15 & Controllare che venga inviato un messaggio di errore se il numero di ore non è valido. & I & S \\
        \hline
    \end{tabular}
    \newpage
    \renewcommand{\arraystretch}{1.8}
    \begin{tabular}{ |m{3em}|m{23em}|m{3em}|m{3em}| }
        \hline
        T\_S16 & Controllare che l'utente possa specificare il progetto correlato all'\glossario{attività}. & I & S \\
        \hline
        T\_S17 & Controllare che venga inviato un messaggio di errore se il progetto non è valido.  & I & S \\
        \hline
        T\_S18 & Controllare che l'utente possa specificare il luogo in cui ha svolto l'\glossario{attività}. & I & S \\
        \hline
        T\_S19 & Controllare che venga inviato un messaggio di errore se il luogo non è valido. & I & S \\
        \hline
        T\_S20 & Controllare se l'utente riesce, tramite il \glossario{chatbot}, ad aprire il cancello di una sede. & I & S \\
        \hline    
        T\_S21 & Controllare che la sede inserita dall'utente per aprire il cancello sia valida. & I & S \\
        \hline    
        T\_S22 & Controllare che venga inviato un messaggio di errore se la sede, per l'apertura del cancello, non è valida. & I & S \\
        \hline
        T\_S23 & Controllare che l'utente possa creare una riunione su una piattaforma per videoconferenze. & NI & - \\
        \hline
        T\_S24 & Controllare che l'utente possa scegliere la piattaforma esterna su cui creare la riunione. & NI & - \\
        \hline
        T\_S25 & Controllare che all'utente venga inviato il link per fare il login e ottenere l'\glossario{access token} per la piattaforma esterna. & NI & - \\
        \hline
        T\_S26 & Controllare che venga inviato un messaggio di errore se la piattaforma non è valida o non è supportata. & NI & - \\
        \hline
        T\_S27 & Controllare che l'utente possa inserire l'\glossario{access token} ottenuto. & NI & - \\
        \hline
        T\_S28 & Controllare che l'utente possa inserire la data della riunione da creare. & NI & - \\
        \hline
        T\_S29 & Controllare che venga inviato un messaggio di errore se la data non è valida o è indisponibile. & NI & - \\
        \hline
        T\_S30 & Controllare che l'utente possa inserire l'ora della riunione da creare. & NI & - \\
        \hline
        T\_S31 & Controllare che venga inviato un messaggio di errore se l'ora non è valida o è indisponibile. & NI & - \\
        \hline
    \end{tabular}
    \newpage
    \renewcommand{\arraystretch}{1.8}
    \begin{tabular}{ |m{3em}|m{23em}|m{3em}|m{3em}| }
        \hline
        T\_S32 & Controllare che l'utente possa specificare i partecipanti della riunione. & NI & - \\
        \hline
        T\_S33 & Controllare che venga inviato un messaggio di errore se i partecipanti non sono stati inseriti in modo corretto. & NI & - \\
        \hline
        T\_S34 & Controllare che l'utente possa effettuare una ricercare dei documenti. & NI & - \\
        \hline
        T\_S35 & Controllare che l'utente possa specificare il progetto in cui ricercare i documenti. & NI & - \\
        \hline
        T\_S36 & Controllare che venga inviato un messaggio di errore se il progetto non è valido. & NI & - \\
        \hline
        T\_S37 & Controllare che l'utente possa inserire il nome del documento da ricercare. & NI & - \\
        \hline
        T\_S38 & Controllare che venga inviato un messaggio di errore se il nome del documento non è valido. & NI & - \\
        \hline
        T\_S39 & Controllare se l'utente può creare un \glossario{ticket}. & NI & - \\
        \hline
        T\_S40 & Controllare che l'utente possa specificare l'oggetto del \glossario{ticket}. & NI & - \\
        \hline
        T\_S41 & Controllare che venga inviato un messaggio di errore se l'oggetto del \glossario{ticket} non è valido. & NI & - \\
        \hline
        T\_S42 & Controllare che l'utente possa aggiungere una descrizione al \glossario{ticket}. & NI & - \\
        \hline
        T\_S43 & Controllare che l'utente possa specificare lo status del \glossario{ticket}. & NI & - \\
        \hline
        T\_S44 & Controllare che l'utente possa specificare la priorità del \glossario{ticket}. & NI & - \\
        \hline
        T\_S45 & Controllare che venga inviato un messaggio di errore se la priorità non è valida. & NI & - \\
        \hline
        T\_S46 & Controllare che l'utente possa interrompere un'operazione in corso. & I & S \\
        \hline
        T\_S47 & Controllare che venga inviato dal \glossario{chatbot} un messaggio che conferma l'annullamento dell'operazione. & I & S \\
        \hline
    \end{tabular}
    \newpage
    \renewcommand{\arraystretch}{1.8}
    \begin{tabular}{ |m{3em}|m{23em}|m{3em}|m{3em}| }
        \hline
        T\_S48 & Controllare che l'utente possa verificare lo stato di \glossario{check-in}/\glossario{check-out}. & NI & - \\
        \hline
        T\_S49 & Controllare che venga inviato un messaggio di errore se è impossibile vedere lo stato di \glossario{check-in}/\glossario{check-out}. & NI & - \\
        \hline
        T\_S50 & Controllare che l'utente possa richiedere al \glossario{chatbot} le ore consuntivate durante la giornata. & NI & - \\
        \hline
        T\_S51 & Controllare che venga inviato un messaggio di errore se è impossibile vedere le ore consuntivate. & NI & - \\
        \hline
        T\_S52 & Controllare che l'utente possa visualizzare le ore rimanenti da consuntivare. & NI & - \\
        \hline
        T\_S53 & Controllare che venga inviato un messaggio di errore se è impossibile vedere le ore rimanenti da consuntivare. & NI & - \\
        \hline
        T\_S54 & Controllare che l'utente possa visualizzare le riunioni della giornata. & NI & - \\
        \hline
        T\_S55 & Controllare che venga inviato un messaggio di errore se è impossibile vedere le riunioni della giornata. & NI & - \\
        \hline
        T\_S56 & Controllare che l'utente possa visualizzare le proprie impostazioni. & NI & - \\
        \hline
        T\_S57 & Controllare che l'utente possa autenticarsi su una \glossario{piattaforma riunioni} esterna. & NI & - \\
        \hline
        T\_S58 & Controllare che venga inviato un messaggio di errore se l'autenticazione sulla \glossario{piattaforma riunioni} è fallita. & NI & - \\
        \hline
        T\_S59 & Controllare che venga inviato un messaggio di errore se l'\glossario{access token} ottenuto non è valido. & NI & - \\
        \hline
	      T\_S60 & Controllare che il sistema riesce a decifrare corretamente il messaggio cifrato con il metodo scelto. & NI & - \\
        \hline
    \end{tabular}
\end{center}

\subsubsection{Tracciamento Test di Sistema - Requisiti}
\begin{center}
    \renewcommand{\arraystretch}{1.8}
    \begin{tabular}{|m{6em}|m{8em}|}
        \hline
        \textbf{Codice test} & \textbf{Codice requisito}\\
        \hline
        T\_S1 & RO-F-1\\
        \hline
        T\_S2 & RO-F-2\\
        \hline
        T\_S3 & RO-F-3\\
        \hline
        T\_S4 & RO-F-6\\
        \hline
        T\_S5 & RO-F-4\\
        \hline
        T\_S6 & RO-F-5\\
        \hline
        T\_S7 & RD-F-7\\
        \hline
        T\_S8 & RO-F-8\\
        \hline
        T\_S9 & RO-F-9\\
        \hline
        T\_S10 & RO-F-10\\
        \hline
        T\_S11 & RO-F-11\\
        \hline
        T\_S12 & RO-F-12\\
        \hline
        T\_S13 & RO-F-16\\
        \hline
        T\_S14 & RO-F-13\\
        \hline
        T\_S15 & RO-F-17\\
        \hline
        T\_S16 & RO-F-14\\
        \hline
        T\_S17 & RO-F-18\\
        \hline
        T\_S18 & RO-F-15\\
        \hline
        T\_S19 & RO-F-19\\
        \hline
        T\_S20 & RD-F-20\\
        \hline
        T\_S21 & RD-F-21\\
        \hline
        T\_S22 & RD-F-22\\
        \hline
        T\_S23 & RD-F-23\\
        \hline
    \end{tabular}
    \newpage
    \renewcommand{\arraystretch}{1.8}
    \begin{tabular}{|m{6em}|m{8em}|}
        \hline
        T\_S24 & RD-F-24\\
        \hline
        T\_S25 & RD-F-54\\
        \hline
        T\_S26 & RD-F-28\\
        \hline
        T\_S27 & RD-F-56\\
        \hline
        T\_S28 & RD-F-25\\
        \hline
        T\_S29 & RD-F-29\\
        \hline
        T\_S30 & RD-F-26\\
        \hline
        T\_S31 & RD-F-30\\
        \hline
        T\_S32 & RD-F-27\\
        \hline
        T\_S33 & RD-F-31\\
        \hline
        T\_S34 & RD-F-32\\
        \hline
        T\_S35 & RD-F-33\\
        \hline
        T\_S36 & RD-F-35\\
        \hline
        T\_S37 & RD-F-34\\
        \hline
        T\_S38 & RD-F-36\\
        \hline
        T\_S39 & RD-F-37\\
        \hline
        T\_S40 & RD-F-38\\
        \hline
        T\_S41 & RD-F-41\\
        \hline
        T\_S42 & RD-F-39\\
        \hline
        T\_S43 & RD-F-40\\
        \hline
        T\_S44 & RD-F-40\\
        \hline
        T\_S45 & RD-F-42\\
        \hline
        T\_S46 & RD-F-43\\
        \hline
        T\_S47 & RD-F-44\\
        \hline
        T\_S48 & RD-F-45\\
        \hline
      \end{tabular}
      \newpage
      \renewcommand{\arraystretch}{1.8}
      \begin{tabular}{|m{6em}|m{8em}|}
          \hline
        T\_S49 & RD-F-46\\
        \hline
        T\_S50 & RD-F-47\\
        \hline
        T\_S51 & RD-F-48\\
        \hline
        T\_S52 & RD-F-49\\
        \hline
        T\_S53 & RD-F-50\\
        \hline
        T\_S54 & RD-F-51\\
        \hline
        T\_S55 & RD-F-52\\
        \hline
        T\_S56 & RD-F-53\\
        \hline
        T\_S57 & RD-F-54\\
        \hline
        T\_S58 & RD-F-55\\
        \hline
        T\_S59 & RD-F-57\\
        \hline
        T\_S60 & RD-F-58\\
        \hline
    \end{tabular}
\end{center}