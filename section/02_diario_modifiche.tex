\section*{Diario delle modifiche}
	\begin{center}
	\renewcommand{\arraystretch}{1.8} %aumento ampiezza righe
	\begin{tabular}{ |c|c|m{12em}|m{7em}|c| }
	\hline
	
	\textbf{Versione} & \textbf{Data} & \textbf{Descrizione} &  \textbf{Autore} &  \textbf{Ruolo} \\ %Aggiungere le nuove righe sopra la prima
	\hline
    0.0.10 & 2022-05-08 & Scrittura \$4 &Pan Qi Fan\newline Andrea & Analista\\
	\hline
    0.0.9 & 2022-05-08 & Scrittura \$5 & Pietro Macrì & Analista\\
	\hline
    0.0.8 & 2022-05-04 & Scrittura da \$3.8 a \$3.20 & Tommaso \newline Berlaffa & Analista\\
    \hline
    0.0.7 & 2022-05-03 & Scrittura \$3.5, \$3.6, \$3.7 & Mattia \newline Episcopo & Analista\\
	\hline
    0.0.6 & 2022-05-01 & Scrittura \$2 & Irene Benetazzo & Analista\\
	\hline
    0.0.5 & 2022-05-01 & Scrittura \$3.4 & Matteo Pillon & Analista \\ % Se il nome dell'approvatore non ci sta, metterlo a mano con aggiunta di \newline (esempio: Nome \newline Cognome)
	\hline
    0.0.4 & 2022-04-30 & Scrittura \$3.2 e \$3.3 & Matteo Pillon & Analista \\ % Se il nome dell'approvatore non ci sta, metterlo a mano con aggiunta di \newline (esempio: Nome \newline Cognome)
	\hline
    0.0.3 & 2022-04-28 & Scrittura \$3.1 & Matteo Pillon & Analista \\ % Se il nome dell'approvatore non ci sta, metterlo a mano con aggiunta di \newline (esempio: Nome \newline Cognome)
	\hline
    0.0.2 & 2022-04-27 & Scrittura \$1 & Irene Benetazzo & Amministratore\\ % Se il nome dell'approvatore non ci sta, metterlo a mano con aggiunta di \newline (esempio: Nome \newline Cognome)
	\hline
    0.0.1 & 2022-04-27 & Creazione documento & Irene Benetazzo & Amministratore\\
	\hline
	\end{tabular}
	\end{center}
	\newpage