\subsection{API Rest Imola Informatica} L'azienda ha fornito delle \glossario{API Rest} che permettono al \glossario{chatbot} di interagire con i loro sistemi aziendali. Sono facilmente consultabili a questo \href{https://apibot4me.imolinfo.it/}{\color{blue} link}.
\subsubsection{Consuntivazione di un attività}
\paragraph{API Url} \hfill \break
/projects/\{code\}/activities/me
\paragraph{Metodo di richiesta \glossario{HTTP}} \hfill \break
POST
\paragraph{Headers \glossario{HTTP}}
\begin{itemize}
    \item \textbf{accept}: application/json;
    \item \textbf{api\_key}: api\_key, per autorizzare le richieste;
    \item \textbf{Content-Type}: application/json.
\end{itemize}
\paragraph{Parametri URL} \hfill \break
\begin{center}
    \renewcommand{\arraystretch}{1.8}
    \begin{tabular}{ |m{10em}|m{4em}|m{20em}| }
        \hline
        \textbf{Nome} & \textbf{Tipo} & \textbf{Descrizione} \\
        \hline
        code & string & rappresenta il codice del progetto di cui fa parte l’attività da rendicontare.\\
        \hline
    \end{tabular}
\end{center}
\paragraph{Parametri Body} \hfill \break
Nel body della richiesta viene passato un array contenente un json con questi parametri
\begin{center}
    \renewcommand{\arraystretch}{1.8}
    \begin{tabular}{ |m{10em}|m{4em}|m{20em}| }
        \hline
        \textbf{Nome} & \textbf{Tipo} & \textbf{Descrizione} \\
        \hline
        date & string & data in cui si è svolta l'attività.\\
        \hline
        billableHours & int & ore fatturabili dell'attività.\\
        \hline
        travelHours & int & ore di viaggio spese per l'attività.\\
        \hline
        billableTravelHours & int & ore di viaggio fatturabili per l'attività.\\
        \hline
        location & string & nome della sede in cui si è svolta l'attività.\\
        \hline
        billable & bool & indica se l'attività è fatturabile.\\
        \hline
        note & string & descrizione dell'attività.\\
        \hline
    \end{tabular}
\end{center}
\paragraph{Risposte}
\begin{center}
    \renewcommand{\arraystretch}{1.8}
    \begin{tabular}{ |m{9em}|m{24em}| }
        \hline
        \textbf{Status code \glossario{HTTP}} & \textbf{Descrizione} \\
        \hline
        204 & indica che la richiesta di consuntivazione è andata a buon fine\\
        \hline
        404 & indica che il codice specificato del progetto non è corretto\\
        \hline
        401 & indica che l'api key non è stata specificata o quella utilizzata non è valida.\\
        \hline
    \end{tabular}
\end{center}
\subsubsection{Apertura del cancello}
\paragraph{API Url} \hfill \break
/locations/\{location\_name\}/devices/\{device\}/status
\paragraph{Metodo di richiesta \glossario{HTTP}} \hfill \break
PUT
\paragraph{Headers \glossario{HTTP}}
\begin{itemize}
    \item \textbf{accept}: application/json;
    \item \textbf{api\_key}: api\_key, per autorizzare le richieste;
    \item \textbf{Content-Type}: application/json.
\end{itemize}
\paragraph{Parametri URL} \hfill \break
\begin{center}
    \renewcommand{\arraystretch}{1.8}
    \begin{tabular}{ |m{10em}|m{4em}|m{20em}| }
        \hline
        \textbf{Nome} & \textbf{Tipo} & \textbf{Descrizione} \\
        \hline
        location\_name & string & sede di cui aprire il cancello.\\
        \hline
        device & string & nome del dispositivo da utilizzare, in questo caso il cancello.\\
        \hline
    \end{tabular}
\end{center}
\paragraph{Parametri Body} \hfill \break
Nel body della richiesta viene passato un json contenente questi parametri
\begin{center}
    \renewcommand{\arraystretch}{1.8}
    \begin{tabular}{ |m{10em}|m{4em}|m{20em}| }
        \hline
        \textbf{Nome} & \textbf{Tipo} & \textbf{Descrizione} \\
        \hline
        status & string & rappresenta il nuovo stato del dispositivo.\\
        \hline
    \end{tabular}
\end{center}
\paragraph{Risposte}
\begin{center}
    \renewcommand{\arraystretch}{1.8}
    \begin{tabular}{ |m{9em}|m{24em}| }
        \hline
        \textbf{Status code \glossario{HTTP}} & \textbf{Descrizione} \\
        \hline
        204 & indica che la richiesta di apertura del cancello è andata a buon fine.\\
        \hline
        404 & indica che la sede di cui aprire il cancello non è valida.\\
        \hline
        401 & indica che l'api key non è stata specificata o quella utilizzata non è valida.\\
        \hline
    \end{tabular}
\end{center}
\subsubsection{Registrazione della presenza}
\paragraph{API Url} \hfill \break
/locations/\{location\_name\}/presence
\paragraph{Metodo di richiesta \glossario{HTTP}} \hfill \break
POST
\paragraph{Headers \glossario{HTTP}}
\begin{itemize}
    \item \textbf{accept}: application/json;
    \item \textbf{api\_key}: api\_key, per autorizzare le richieste;
    \item \textbf{Content-Type}: application/json.
\end{itemize}
\paragraph{Parametri URL} \hfill \break
\begin{center}
    \renewcommand{\arraystretch}{1.8}
    \begin{tabular}{ |m{10em}|m{4em}|m{20em}| }
        \hline
        \textbf{Nome} & \textbf{Tipo} & \textbf{Descrizione} \\
        \hline
        location\_name & string & sede in cui registrare la presenza.\\
        \hline
    \end{tabular}
\end{center}
\paragraph{Risposte}
\begin{center}
    \renewcommand{\arraystretch}{1.8}
    \begin{tabular}{ |m{9em}|m{24em}| }
        \hline
        \textbf{Status code \glossario{HTTP}} & \textbf{Descrizione} \\
        \hline
        200 & indica che la registrazione della presenza è stata effettuata con successo.\\
        \hline
        404 & indica che la sede specificata non è valida.\\
        \hline
        401 & indica che l'api key non è stata specificata o quella utilizzata non è valida.\\
        \hline
    \end{tabular}
\end{center}
\subsubsection{Creazione di un nuovo progetto}
\paragraph{API Url} \hfill \break
/projects
\paragraph{Metodo di richiesta \glossario{HTTP}} \hfill \break
POST
\paragraph{Headers \glossario{HTTP}}
\begin{itemize}
    \item \textbf{accept}: application/json;
    \item \textbf{api\_key}: api\_key, per autorizzare le richieste;
\end{itemize}
\paragraph{Parametri Body} \hfill \break
Nel body della richiesta viene passato un json contenente questi parametri
\begin{center}
    \renewcommand{\arraystretch}{1.8}
    \begin{tabular}{ |m{10em}|m{4em}|m{20em}| }
        \hline
        \textbf{Nome} & \textbf{Tipo} & \textbf{Descrizione} \\
        \hline
        code & string & codice del progetto da creare.\\
        \hline
        detail & string & descrizione del progetto da creare.\\
        \hline
        customer & string & cliente per cui viene creato il progetto.\\
        \hline
        manager & string & manager del progetto da creare.\\
        \hline
        status & string & stato del progetto.\\
        \hline
        area & string & sede in cui svolgere il progetto.\\
        \hline
        startDate & string & data di inizio del progetto.\\
        \hline
        endDate & string & data di fine del progetto.\\
        \hline
    \end{tabular}
\end{center}
\paragraph{Risposte}
\begin{center}
    \renewcommand{\arraystretch}{1.8}
    \begin{tabular}{ |m{9em}|m{24em}| }
        \hline
        \textbf{Status code \glossario{HTTP}} & \textbf{Descrizione} \\
        \hline
        204 & indica che è l'operazione di creazione del progetto è andata a buon fine.\\
        \hline
        400 & indica che almeno uno dei parametri non è stato specificato.\\
        \hline
        401 & indica che l'api key non è stata specificata o quella utilizzata non è valida.\\
        \hline
    \end{tabular}
\end{center}
\subsubsection{Recupero delle sedi}
\paragraph{API Url} \hfill \break
/locations
\paragraph{Metodo di richiesta \glossario{HTTP}} \hfill \break
GET
\paragraph{Headers \glossario{HTTP}}
\begin{itemize}
    \item \textbf{accept}: application/json;
    \item \textbf{api\_key}: api\_key, per autorizzare le richieste;
\end{itemize}
\paragraph{Risposte}
\begin{center}
    \renewcommand{\arraystretch}{1.8}
    \begin{tabular}{ |m{9em}|m{24em}| }
        \hline
        \textbf{Status code \glossario{HTTP}} & \textbf{Descrizione} \\
        \hline
        200 & ritorna una lista contenente le informazioni delle sedi.\\
        \hline
        401 & indica che l'api key non è stata specificata o quella utilizzata non è valida.\\
        \hline
    \end{tabular}
\end{center}
\subsubsection{Recupero della rendicontazione per il progetto corrente}
\paragraph{API Url} \hfill \break
/projects/\{code\}
\paragraph{Metodo di richiesta \glossario{HTTP}} \hfill \break
GET
\paragraph{Headers \glossario{HTTP}}
\begin{itemize}
    \item \textbf{accept}: application/json;
    \item \textbf{api\_key}: api\_key, per autorizzare le richieste;
\end{itemize}
\paragraph{Parametri URL} \hfill \break
\begin{center}
    \renewcommand{\arraystretch}{1.8}
    \begin{tabular}{ |m{10em}|m{4em}|m{20em}| }
        \hline
        \textbf{Nome} & \textbf{Tipo} & \textbf{Descrizione} \\
        \hline
        code & string & codice del progetto corrente.\\
        \hline
    \end{tabular}
\end{center}
\paragraph{Risposte}
\begin{center}
    \renewcommand{\arraystretch}{1.8}
    \begin{tabular}{ |m{9em}|m{24em}| }
        \hline
        \textbf{Status code \glossario{HTTP}} & \textbf{Descrizione} \\
        \hline
        200 & ritorna un json contenente le informazioni per il progetto corrente.\\
        \hline
        401 & indica che l'api key non è stata specificata o quella utilizzata non è valida.\\
        \hline
    \end{tabular}
\end{center}