Seguendo le Norme di Progetto ogni requisito è identificato da il suo codice, la sua descrizione e la fonte di provenienza.
\subsection{Requisiti Funzionali}
\begin{center}
\renewcommand{\arraystretch}{1.8} %aumento ampiezza righe
\begin{tabular}{ | m{8em} | m{18em} | m{12em} | }
\hline
Codice&Descrizione&Fonte\\
\hline
RO-F-1 & Il \glossario{ChatBot} deve poter riconoscere input testuale & Capitolato, UC1.1.1, UC2.1.1, UC3.1.1, UC3.2.1, UC3.3.1, UC3.4.1, UC4.1.1, UC5.1.1, UC5.2.1, UC5.3.1, UC5.4.1, UC6.1.1, UC6.2.1, UC7.1.1, UC7.2.1, UC7.3.1, UC8.1.1, UC9.1.1, UC10.1.1, UC11.1.1, UC12.1.1\\
\hline
RO-F-2 & Il \glossario{ChatBot} deve poter riconoscere input vocale & Capitolato, UC2.1.2, UC3.1.2, UC3.2.2, UC3.3.2, UC3.4.2, UC4.1.2, UC5.1.2, UC5.2.2, UC5.3.2, UC5.4.2, UC6.1.2, UC6.2.2, UC7.1.2, UC7.2.2, UC7.3.2, UC8.1.2, UC9.1.2, UC10.1.2, UC11.1.2, UC12.1.2\\
\hline
RO-F-3&L’utente deve poter autenticarsi tramite un \glossario{token}&UC1\\
\hline
RO-F-4&Il programma deve poter riconoscere un \glossario{token} non valido&UC1\\
\hline
RO-F-5&Il programma deve poter far visualizzare un errore se il \glossario{token} non è valido&UC1.2\\
\hline
RO-F-6&L’utente non autenticato deve poter richiedere un \glossario{token} per autenticarsi&UC1.1\\
\hline
RD-F-7&Se l’utente ha più \glossario{token} a disposizione, può decidere quale usare.&Verbale Esterno\\
\hline
RO-F-8&L’utente deve poter effettuare la registrazione della propria presenza &UC2 \\
\hline
\end{tabular}
\end{center}
\begin{center}
\renewcommand{\arraystretch}{1.8} %aumento ampiezza righe
\begin{tabular}{ | m{8em} | m{18em} | m{12em} | }
\hline
RO-F-9&L’utente deve poter inserire la sede per la registrazione della presenza &UC2.1 \\
\hline
RO-F-10&Il \glossario{ChatBot} deve notificare l’utente se la sede per la registrazione non è valida &UC2.2 \\
\hline
RO-F-11&L’utente deve poter inserire un'\glossario{attività} all’interno del \glossario{sistema EMT} &UC3 \\
\hline
RO-F-12&L’utente deve poter inserire il tipo dell’\glossario{attività} &UC3.1 \\
\hline
RO-F-13&L’utente deve poter inserire le ore da consuntivare nell’\glossario{attività} &UC3.2 \\
\hline  
RO-F-14&L’utente deve poter inserire il progetto da consuntivare nell’\glossario{attività} &UC3.3 \\
\hline
RO-F-15&L’utente deve poter inserire il luogo in cui ha svolto l’\glossario{attività} &UC3.4 \\
\hline
RO-F-16&Il \glossario{ChatBot} deve notificare l'utente se l'\glossario{attività} è in un formato non valido &UC3.5 \\
\hline
RO-F-17&Il \glossario{ChatBot} deve notificare l'utente se le ore sono in un formato non valido &UC3.6 \\
\hline
RO-F-18&Il \glossario{ChatBot} deve notificare l'utente se il nome del progetto di un'\glossario{attività} è in un formato non valido &UC3.7 \\
\hline
RO-F-19&Il \glossario{ChatBot} deve notificare l'utente se il luogo relativo all’\glossario{attività} è in un formato non valido &UC3.8 \\
\hline
RD-F-20&L’utente deve poter aprire il cancello di un sede tramite \glossario{ChatBot} &UC4 \\
\hline
RO-F-21&L’utente deve poter inserire la sede in cui vuole aprire il candello &UC4.1 \\
\hline
RD-F-22&Il \glossario{ChatBot} deve notificare l'utente se la sede indicata non è valida &UC4.2 \\
\hline
RD-F-23&L’utente deve poter creare, tramite il \glossario{ChatBot}, una riunione su un'applicazione esterna &UC5 \\
\hline
\end{tabular}
\end{center}
\begin{center}
\renewcommand{\arraystretch}{1.8} %aumento ampiezza righe
\begin{tabular}{ | m{8em} | m{18em} | m{12em} | }
\hline
RD-F-24&L’utente deve poter inserire il nome della piattaforma esterna per la videoconferenza &UC5.1 \\
\hline
RD-F-25&L’utente deve poter inserire la data per la riunione &UC5.2 \\
\hline
RD-F-26&L’utente deve poter inserire l’ora per la riunione &UC5.3 \\
\hline
RD-F-27&L’utente deve poter inserire i partecipanti alla riunione &UC5.4 \\
\hline
RD-F-28&Il \glossario{ChatBot} deve notificare l’utente se la piattaforma inserita non è valida o non è supportata &UC5.5 \\
\hline
RD-F-29&Il \glossario{ChatBot} deve notificare l’utente se la data inserita non è valida o è indisponibile &UC5.6 \\
\hline
RD-F-30&Il \glossario{ChatBot} deve notificare l’utente se l’ora inserita non è valida o è indisponibile &UC5.7 \\
\hline
RD-F-31&Il \glossario{ChatBot} deve notificare l’utente se i partecipanti inseriti non sono corretti, chiedendo il reinserimento &UC5.8 \\
\hline
RD-F-32&L’utente deve poter chiedere di cercare un documento &UC6 \\
\hline
RD-F-33&L’utente deve poter inserire il nome del progetto per la ricerca del documento &UC6.1 \\
\hline
RD-F-34&L’utente deve poter inserire il nome del documento in cui vuole fare la ricerca &UC6.2 \\
\hline
RD-F-35&Il \glossario{ChatBot} deve notificare l’utente se l’inserimento del progetto non è corretto, chiedendo il reinserimento &UC6.3\\
\hline
RD-F-36&Il \glossario{ChatBot} deve notificare l’utente se il nome del documento inserito non è valido, chiedendo il reinserimento &UC6.4 \\
\hline
RD-F-37&L’utente deve poter inserire un \glossario{ticket} &UC7 \\
\hline
\end{tabular}
\end{center}
\begin{center}
\renewcommand{\arraystretch}{1.8} %aumento ampiezza righe
\begin{tabular}{ | m{8em} | m{18em} | m{12em} | }
\hline
RD-F-38&L’utente deve poter fornire l’oggetto per la creazione del \glossario{ticket} &UC7.1 \\
\hline
RD-F-39&L’utente deve poter inserire la descrizione per la creazione del \glossario{ticket} dopo aver fornito l’oggetto &UC7.2 \\
\hline
RD-F-40&L’utente deve poter inserire la priorità del \glossario{ticket} dopo aver comunicato lo status &UC7.3 \\
\hline
RD-F-41&Il \glossario{ChatBot} deve notificare l’utente se l’oggetto non è stato inserito in maniera idonea per la creazione del \glossario{ticket}  &UC7.4 \\
\hline
RD-F-42&Il \glossario{ChatBot} deve notificare l’utente se la priorità del \glossario{ticket} non è stato inserita in un formato valido &UC7.5 \\
\hline
RD-F-43&L’utente deve poter annullare un'operazione precedentemente richiesta&UC8, UC8.1 \\
\hline
RD-F-44&Il \glossario{ChatBot} deve notificare l’utente se annullamento di un'operazione è con successo oppure impossibile&UC.2 \\
\hline
RD-F-45&L’utente deve poter richiedere lo stato di \glossario{Check-in}/Check-out&UC9, UC9.1 \\
\hline
RD-F-46&Il \glossario{ChatBot} deve notificare l’utente se la richiesta dello stato di \glossario{Check-in}/Check-out è impossibile&UC9.2 \\
\hline
RD-F-47&L’utente deve poter richiedere Ore Consuntivate&UC10, UC10.1 \\
\hline
RD-F-48&Il \glossario{ChatBot} deve notificare l’utente se la richiesta di visualizzazione ore consuntivate è impossibile &UC10.2 \\
\hline
RD-F-49&L’utente deve poter richiedere Ore da Consuntivare rimanenti&UC11, UC11.1 \\
\hline
\end{tabular}
\end{center}
\begin{center}
\renewcommand{\arraystretch}{1.8} %aumento ampiezza righe
\begin{tabular}{ | m{8em} | m{18em} | m{12em} | }
\hline
RD-F-50&Il \glossario{ChatBot} deve notificare l’utente se la richiesta di visualizzazione di ore da consuntivare rimanenti è impossibile &UC11.2 \\
\hline
RD-F-51&L’utente deve poter visualizzare riunioni giornaliere&UC12, UC12.1 \\
\hline
RD-F-52&Il \glossario{ChatBot} deve notificare l’utente se la richiesta di visualizzazione riunioni giornaliere è impossibile &UC12.2 \\
\hline
RD-F-53&L’utente deve poter visualizzare le impostazioni&UC13\\
\hline
RD-F-54&L’utente deve poter autenticarsi su \glossario{Piattaforma Riunioni} esterna&UC14, UC14.1 \\
\hline
RD-F-55&Il \glossario{ChatBot} deve notificare l’utente l'autenticazione su \glossario{Piattaforma Riunioni esterna} è fallita &UC14.2 \\
\hline
RD-F-56&L’utente deve poter inserire il \glossario{access token} per autenticarsi su \glossario{Piattaforma Riunioni} esterna&UC14.3 \\
\hline
RD-F-57&Il \glossario{ChatBot} deve notificare l’utentese l'inserimento del \glossario{access roken} è fallita &UC14.4 \\
\hline
RD-F-58&Cifrare con il metodo \glossario{CBC-MAC} tutte le comunicazioni fra App e Server per garantire la validità delle informazioni&Capitolato\\
\hline
\end{tabular}
\end{center}
\newpage


\subsection{Requisiti di qualità}
\begin{center}
\renewcommand{\arraystretch}{1.8} %aumento ampiezza righe
\begin{tabular}{ | m{8em} | m{18em} | m{12em} | }
\hline
Codice&Descrizione&Fonte\\
\hline
RO-Q-1&Almeno l'80\% delle funzioni del programma deve essere testato e correlato dal report del test&Capitlato\\
\hline
RO-Q-2&Redarre documenti sulle scelte implementative e progettuali e relative motivazioni&Capitolato\\
\hline
RO-Q-3&Redarre documenti su problemi aperti e soluzioni possibili&Capitolato\\
\hline
RO-Q-4&Rispettare il piano di qualifica&Verbale interno\\
\hline
\end{tabular}
\end{center}

\subsection{Requisiti di vincolo}
\begin{center}
\renewcommand{\arraystretch}{1.8} %aumento ampiezza righe
\begin{tabular}{ | m{8em} | m{18em} | m{12em} | }
\hline
Codice&Descrizione&Fonte\\
\hline
RO-V-1&Utilizzare un applicazione mobile (per IOS 12 in poi o Android 4.1 in pois) che permette di utilizzare tutte le funzioni descritte dai requisiti funzionali&Capitolato\\
\hline
\end{tabular}
\end{center}