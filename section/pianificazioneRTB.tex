\subsection{Requirements Technology Baseline}
Questa macrofase è suddivisa in tre fasi:
\begin{itemize}
    \item \textbf{Baseline documentale} dal 2022-04-19 al 2022-05-02
    \item \textbf{Baseline dei requisiti} dal 2022-05-03 al 2022-05-09
    \item \textbf{Baseline delle tecnologie} dal 2022-05-10 al 2022-05-23
\end{itemize}

\subsubsection{Baseline documentale}
\begin{itemize}
    \item \textbf{Norme di Progetto:} l'amministratore redige le norme e gli strumenti necessari ad una buona organizzazione e realizzazione del progetto.
    \item \textbf{Piano di Progetto:} l'analista rileva i possibili rischi che possono sorgere durante il progetto. Il responsabile pianifica la macrofase RTB.
    \item \textbf{Piano di Qualifica:} l'analista stabilisce e redige i parametri di qualità stabilendo soglie minime accettabili e soglie desiderate. 
    \item \textbf{Glossario:} viene aggiornato se ritenuto necessario durante altre attività.
    \item \textbf{Verifica:} il verificatore controlla che siano state rispettate le norme e verifica il materiale scritto durante questa fase.
    \item \textbf{Checkpoint:} per la descrizione dettagliata si veda \$4.1.2 
\end{itemize}

\subsubsection{Baseline dei requisiti}
\begin{itemize}
    \item \textbf{Analisi dei Requisiti:} l'analista analizza e scrive tutti i requisiti del progetto commissionato dal proponente, 
                comprese tutte le richieste opzionali e desiderabili (indipendentemente dalla possibilità di realizzarli tutti).
    \item \textbf{Norme di Progetto, Piano di Progetto, Piano di Qualifica:} documenti che vengono incrementati e aggiornati.
    \item \textbf{Glossario:} viene aggiornato se ritenuto necessario durante altre attività.
    \item \textbf{Verifica:} il verificatore controlla che siano state rispettate le norme e verifica il materiale scritto prodotto durante questa fase.
    \item \textbf{Checkpoint:} per la descrizione dettagliata si veda \$4.1.2
\end{itemize}

\subsubsection{Baseline delle tecnologie}
\begin{itemize}
    \item \textbf{Specifica Architetturale:} il progettista analizzando pro e contro sceglie le tecnologie per il progetto, poi il gruppo le studia autonomamente.
    \item \textbf{Proof of Concept:} il progettista implementa la dimostrazione del progetto integrando le varie tecnologie e alcune funzionalità principali.
    \item \textbf{Norme di Progetto, Piano di Progetto, Piano di Qualifica:} documenti che vengono incrementati e aggiornati.
    \item \textbf{Glossario:} viene aggiornato se ritenuto necessario durante altre attività.
    \item \textbf{Verifica:} il verificatore controlla che siano state rispettate le norme e verifica il materiale scritto e il prodotto realizzato durante questa fase.
    \item \textbf{Approvazione:} il responsabile controlla, approva il materiale e il prodotto realizzato durante questa macrofase.
    \item \textbf{Checkpoint:} per la descrizione dettagliata si veda \$4.1.2
    \item \textbf{Presentazione RTB:} viene preparata la presentazione per il colloquio e pubblicato il materiale nella repository pubblica.
\end{itemize}


\begin{landscape}
    \begin{center}
        Diagramma di Gantt - Macrofase RTB
    \end{center}
	\begin{figure}
	\includegraphics[width=\linewidth]{images/RTB.png} 
	\end{figure}
\end{landscape}