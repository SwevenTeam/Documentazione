\subsection{Introduzione}
Il gruppo Sweven ha deciso di suddividere il lavoro in macrofasi seguendo le revisioni da effettuare con il committente:
\begin{itemize}
    \item \textbf{RTB:} Requirements Technology Baseline è prevista per fine Maggio 2022
    \item \textbf{PB:} Product Baseline prevista nel mese di Agosto 2022
    \item \textbf{CA:} Customer Acceptance prevista nel mese di Settembre 2022
\end{itemize}

\subsubsection{Diagramma di Gantt}
Nel diagramma di Gantt le varie attività vengono riportate mediante una dicitura sintetica. 
I colori indicano la tipologia dell'attività o la priorità di essa rispetto alle altre della stessa fase:
\begin{itemize}
    \item \textbf{Nero:} indica il periodo dell'intera fase
    \item \textbf{Rosso:} indica attività con priorità alta
    \item \textbf{Viola:} indica attività con priorità media
    \item \textbf{Azzurro:} indica attività da mantenere aggiornata o fare se necessaria in relazione alle altre attività
    \item \textbf{Giallo:} verifica delle attività eseguite durante la fase
    \item \textbf{Verde:} approvazione di ciò che è stato fatto finora (non solo durante la fase stessa)
    \item \textbf{Blù:} attività di consuntivazione delle ore e del costo
\end{itemize}

\subsubsection{Attività comuni a tutte le fasi}
\begin{itemize}
    \item \textbf{Glossario} viene aggiornato se ritenuto necessario durante altre attività.
    \item \textbf{Verifica} del materiale scritto e/o prodotto durante quella fase.
    \item \textbf{Checkpoint} riunione di gruppo per il punto della situazione:
        \begin{itemize}
            \item Redicontazione delle ore effettivamente necessarie
            \item Analisi rispetto alla pianificazione e preventivo costi
            \item Analisi di eventuali criticità
            \item Valutare eventuali modifiche sulla pianificazione futura
            \item Organizzare nel dettaglio, assegnando i singoli task, la fase successiva 
        \end{itemize}
\end{itemize}