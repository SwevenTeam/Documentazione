\subsection{Introduzione}
Il gruppo Sweven ha deciso di suddividere il lavoro in macrofasi seguendo le revisioni da effettuare con il committente:
\begin{itemize}
    \item \textbf{RTB:} Requirements Technology Baseline è prevista per fine Maggio 2022
    \item \textbf{PB:} Product Baseline prevista nel mese di Agosto 2022
    \item \textbf{CA:} Customer Acceptance prevista nel mese di Settembre 2022
\end{itemize}

\subsubsection{Diagramma di Gantt}
Nel diagramma di Gantt le varie attività vengono riportate mediante una dicitura sintetica. 
I colori indicano la tipologia dell'attività o la priorità di essa rispetto alle altre della stessa fase:
\begin{itemize}
    \item \textbf{Nero:} indica il periodo dell'intera fase
    \item \textbf{\textcolor{red}{Rosso:}} indica attività con priorità alta
    \item \textbf{\textcolor{violet}{Viola:}} indica attività con priorità media
    \item \textbf{\textcolor{cyan}{Azzurro:}} indica attività da mantenere aggiornata o fare se necessaria in relazione alle altre attività
    \item \textbf{\textcolor{yellow}{Giallo:}} verifica delle attività eseguite durante la fase
    \item \textbf{\textcolor{green}{Verde:}} approvazione di ciò che è stato fatto finora (non solo durante la fase stessa)
    \item \textbf{\textcolor{blue}{Blù:}} attività di consuntivazione delle ore e del costo
    \item \textbf{\textcolor{orange}{Arancione:}} indica preparazione per la revisione 
\end{itemize}

\subsubsection{Checkpoint}
Al termine di ogni fase è prevista una riunione interna di tutti i membri del gruppo:
    \begin{itemize}
        \item Redicontazione delle ore effettivamente necessarie per i task assegnati
        \item Analisi rispetto alla pianificazione e preventivo costi
        \item Analisi di eventuali criticità
        \item Valutare eventuali modifiche sulla pianificazione futura
        \item Organizzare nel dettaglio, assegnando i singoli task, la fase successiva 
    \end{itemize}
