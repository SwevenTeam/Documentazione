\section{Specifica dei test}
In questa parte vengono riportati i test da implementare allo scopo
di soddisfare i requisiti individuati e la corretta esecuzione del prodotto.
I test si suddividono in:
\begin{itemize}
    \item Test di unità
    \item Test di integrazione
    \item Test di sistema
\end{itemize}
I codici identificativi delle tipologie di test sono definiti nel documento \emph{Norme di Progetto}.
Nelle tabelle vengono utilizzate ulteriori sigle, di seguito viene descritto il loro significato:
\begin{itemize}
    \item Stato del test:
    \begin{itemize}
        \item NI non implementato;
        \item I implementato;
    \end{itemize}
    \item Esito del test:
    \begin{itemize}
        \item NS non superato;
        \item S superato.
    \end{itemize}
\end{itemize}

\subsection{Test di sistema}
Questi test servono per verificare che il funzionamento complessivo dell'applicazione rispetti i requisiti stabiliti nell'\emph{Analisi dei requisiti}.
\begin{center}
    \renewcommand{\arraystretch}{1.8}
    \begin{tabular}{ |m{3em}|m{23em}|m{3em}|m{3em}| }
        \hline
        \textbf{Codice} & \textbf{Descrizione} & \textbf{Stato} & \textbf{Esito} \\
        \hline
        T\_S1 & Controllare che il \glossario{chatbot} possa rispondere a messaggi testuali. & NI & - \\
        \hline
        T\_S2 & Controllare che il \glossario{chatbot} possa rispondere a messaggi vocali. & NI & - \\
        \hline
        T\_S3 & Controllare che il \glossario{chatbot} invii un link alla richiesta di autenticazione dell'utente. & NI & - \\
        \hline
        T\_S4 & Controllare che l'utente riesca ad autenticarsi tramite \glossario{token}. & NI & - \\
        \hline
        T\_S5 & Controllare che il \glossario{chatbot} sia in grado di riconoscere un \glossario{token} non valido. & NI & - \\
        \hline
        T\_S6 & Controllare che se l'utente fornisce un \glossario{token} non valido venga visualizzato un messaggio di errore. & NI & - \\
        \hline
    \end{tabular}
    \newpage
    \renewcommand{\arraystretch}{1.8}
    \begin{tabular}{ |m{3em}|m{23em}|m{3em}|m{3em}| }
        \hline
        T\_S7 & Controllare che l'utente possa fornire \glossario{token} diversi per l'autenticazione. & NI & - \\
        \hline
        T\_S8 & Controllare che l'utente possa registrare la propria presenza in sede. & NI & - \\
        \hline
        T\_S9 & Controllare che, durante il \glossario{check-in}, la sede inserita dall'utente sia valida. & NI & - \\
        \hline
        T\_S10 & Controllare che, durante il \glossario{check-in}, l'utente venga informato se la sede fornita non è valida. & NI & - \\
        \hline
        T\_S11 & Controllare che l'utente possa inserire un'\glossario{attività} nel \glossario{sistema emt}. & NI & - \\
        \hline
        T\_S12 & Controllare che l'utente possa specificare il tipo di \glossario{attività} da inserire nel \glossario{sistema emt}. & NI & - \\
        \hline
        T\_S13 & Controllare che venga inviato un messaggio di errore se il tipo di \glossario{attività} non è valido nel \glossario{sistema emt}. & NI & - \\
        \hline
        T\_S14 & Controllare che l'utente possa specificare il numero di ore da consuntivare. & NI & - \\
        \hline
        T\_S15 & Controllare che venga inviato un messaggio di errore se il numero di ore non è valido. & NI & - \\
        \hline
        T\_S16 & Controllare che l'utente possa specificare il progetto correlato all'\glossario{attività}. & NI & - \\
        \hline
        T\_S17 & Controllare che venga inviato un messaggio di errore se il progetto non è valido.  & NI & - \\
        \hline
        T\_S18 & Controllare che l'utente possa specificare il luogo in cui ha svolto l'\glossario{attività}. & NI & - \\
        \hline
        T\_S19 & Controllare che venga inviato un messaggio di errore se il luogo non è valido. & NI & - \\
        \hline
        T\_S20 & Controllare se l'utente riesce, tramite il \glossario{chatbot}, ad aprire il cancello di una sede. & NI & - \\
        \hline    
        T\_S21 & Controllare che la sede inserita dall'utente per aprire il cancello sia valida. & NI & - \\
        \hline    
        T\_S22 & Controllare che venga inviato un messaggio di errore se la sede, per l'apertura del cancello, non è valida. & NI & - \\
        \hline
    \end{tabular}
    \newpage
    \renewcommand{\arraystretch}{1.8}
    \begin{tabular}{ |m{3em}|m{23em}|m{3em}|m{3em}| }
        \hline
        T\_S23 & Controllare che l'utente possa creare una riunione su una piattaforma per videoconferenze. & NI & - \\
        \hline
        T\_S24 & Controllare che l'utente possa scegliere la piattaforma esterna su cui creare la riunione. & NI & - \\
        \hline
        T\_S25 & Controllare che all'utente venga inviato il link per fare il login e ottenere l'\glossario{access token} per la piattaforma esterna. & NI & - \\
        \hline
        T\_S26 & Controllare che venga inviato un messaggio di errore se la piattaforma non è valida o non è supportata. & NI & - \\
        \hline
        T\_S27 & Controllare che l'utente possa inserire l'\glossario{access token} ottenuto. & NI & - \\
        \hline
        T\_S28 & Controllare che l'utente possa inserire la data della riunione da creare. & NI & - \\
        \hline
        T\_S29 & Controllare che venga inviato un messaggio di errore se la data non è valida o è indisponibile. & NI & - \\
        \hline
        T\_S30 & Controllare che l'utente possa inserire l'ora della riunione da creare. & NI & - \\
        \hline
        T\_S31 & Controllare che venga inviato un messaggio di errore se l'ora non è valida o è indisponibile. & NI & - \\
        \hline
        T\_S32 & Controllare che l'utente possa specificare i partecipanti della riunione. & NI & - \\
        \hline
        T\_S33 & Controllare che venga inviato un messaggio di errore se i partecipanti non sono stati inseriti in modo corretto. & NI & - \\
        \hline
        T\_S34 & Controllare che l'utente possa effettuare una ricercare dei documenti. & NI & - \\
        \hline
        T\_S35 & Controllare che l'utente possa specificare il progetto in cui ricercare i documenti. & NI & - \\
        \hline
        T\_S36 & Controllare che venga inviato un messaggio di errore se il progetto non è valido. & NI & - \\
        \hline
        T\_S37 & Controllare che l'utente possa inserire il nome del documento da ricercare. & NI & - \\
        \hline
    \end{tabular}
    \newpage
    \renewcommand{\arraystretch}{1.8}
    \begin{tabular}{ |m{3em}|m{23em}|m{3em}|m{3em}| }
        \hline
        T\_S38 & Controllare che venga inviato un messaggio di errore se il nome del documento non è valido. & NI & - \\
        \hline
        T\_S39 & Controllare se l'utente può creare un \glossario{ticket}. & NI & - \\
        \hline
        T\_S40 & Controllare che l'utente possa specificare l'oggetto del \glossario{ticket}. & NI & - \\
        \hline
        T\_S41 & Controllare che venga inviato un messaggio di errore se l'oggetto del \glossario{ticket} non è valido. & NI & - \\
        \hline
        T\_S42 & Controllare che l'utente possa aggiungere una descrizione al \glossario{ticket}. & NI & - \\
        \hline
        T\_S43 & Controllare che l'utente possa specificare lo status del \glossario{ticket}. & NI & - \\
        \hline
        T\_S44 & Controllare che l'utente possa specificare la priorità del \glossario{ticket}. & NI & - \\
        \hline
        T\_S45 & Controllare che venga inviato un messaggio di errore se la priorità non è valida. & NI & - \\
        \hline
        T\_S46 & Controllare che l'utente possa interrompere un'operazione in corso. & NI & - \\
        \hline
        T\_S47 & Controllare che venga inviato dal \glossario{chatbot} un messaggio che conferma l'annullamento dell'operazione. & NI & - \\
        \hline
        T\_S48 & Controllare che l'utente possa verificare lo stato di \glossario{check-in}/\glossario{check-out}. & NI & - \\
        \hline
        T\_S49 & Controllare che venga inviato un messaggio di errore se è impossibile vedere lo stato di \glossario{check-in}/\glossario{check-out}. & NI & - \\
        \hline
        T\_S50 & Controllare che l'utente possa richiedere al \glossario{chatbot} le ore consuntivate durante la giornata. & NI & - \\
        \hline
        T\_S51 & Controllare che venga inviato un messaggio di errore se è impossibile vedere le ore consuntivate. & NI & - \\
        \hline
        T\_S52 & Controllare che l'utente possa visualizzare le ore rimanenti da consuntivare. & NI & - \\
        \hline
        T\_S53 & Controllare che venga inviato un messaggio di errore se è impossibile vedere le ore rimanenti da consuntivare. & NI & - \\
        \hline
    \end{tabular}
    \newpage
    \renewcommand{\arraystretch}{1.8}
    \begin{tabular}{ |m{3em}|m{23em}|m{3em}|m{3em}| }
        \hline
        T\_S54 & Controllare che l'utente possa visualizzare le riunioni della giornata. & NI & - \\
        \hline
        T\_S55 & Controllare che venga inviato un messaggio di errore se è impossibile vedere le riunioni della giornata. & NI & - \\
        \hline
        T\_S56 & Controllare che l'utente possa visualizzare le proprie impostazioni. & NI & - \\
        \hline
        T\_S57 & Controllare che l'utente possa autenticarsi su una \glossario{piattaforma riunioni} esterna. & NI & - \\
        \hline
        T\_S58 & Controllare che venga inviato un messaggio di errore se l'autenticazione sulla \glossario{piattaforma riunioni} è fallita. & NI & - \\
        \hline
        T\_S59 & Controllare che venga inviato un messaggio di errore se l'\glossario{access token} ottenuto non è valido. & NI & - \\
        \hline
    \end{tabular}
\end{center}

\subsubsection{Tracciamento test di sistema - requisiti}
\begin{center}
    \renewcommand{\arraystretch}{1.8}
    \begin{tabular}{|m{6em}|m{8em}|}
        \hline
        \textbf{Codice test} & \textbf{Codice requisito}\\
        \hline
        T\_S1 & RO-F-1\\
        \hline
        T\_S2 & RO-F-2\\
        \hline
        T\_S3 & RO-F-6\\
        \hline
        T\_S4 & RO-F-3\\
        \hline
        T\_S5 & RO-F-4\\
        \hline
        T\_S6 & RO-F-5\\
        \hline
        T\_S7 & RD-F-7\\
        \hline
        T\_S8 & RO-F-8\\
        \hline
        T\_S9 & RO-F-9\\
        \hline
        T\_S10 & RO-F-10\\
        \hline
        T\_S11 & RO-F-11\\
        \hline
        T\_S12 & RO-F-12\\
        \hline
        T\_S13 & RO-F-16\\
        \hline
    \end{tabular}
    \newpage
    \renewcommand{\arraystretch}{1.8}
    \begin{tabular}{|m{6em}|m{8em}|}
        \hline
        T\_S14 & RO-F-13\\
        \hline
        T\_S15 & RO-F-17\\
        \hline
        T\_S16 & RO-F-14\\
        \hline
        T\_S17 & RO-F-18\\
        \hline
        T\_S18 & RO-F-15\\
        \hline
        T\_S19 & RO-F-19\\
        \hline
        T\_S20 & RD-F-20\\
        \hline
        T\_S21 & RD-F-21\\
        \hline
        T\_S22 & RD-F-22\\
        \hline
        T\_S23 & RD-F-23\\
        \hline
        T\_S24 & RD-F-24\\
        \hline
        T\_S25 & RD-F-54\\
        \hline
        T\_S26 & RD-F-28\\
        \hline
        T\_S27 & RD-F-56\\
        \hline
        T\_S28 & RD-F-25\\
        \hline
        T\_S29 & RD-F-29\\
        \hline
        T\_S30 & RD-F-26\\
        \hline
        T\_S31 & RD-F-30\\
        \hline
        T\_S32 & RD-F-27\\
        \hline
        T\_S33 & RD-F-31\\
        \hline
        T\_S34 & RD-F-32\\
        \hline
        T\_S35 & RD-F-33\\
        \hline
        T\_S36 & RD-F-35\\
        \hline
        T\_S37 & RD-F-34\\
        \hline
        T\_S38 & RD-F-36\\
        \hline
    \end{tabular}
    \newpage
    \renewcommand{\arraystretch}{1.8}
    \begin{tabular}{|m{6em}|m{8em}|}
        \hline
        T\_S39 & RD-F-37\\
        \hline
        T\_S40 & RD-F-38\\
        \hline
        T\_S41 & RD-F-41\\
        \hline
        T\_S42 & RD-F-39\\
        \hline
        T\_S43 & RD-F-40\\
        \hline
        T\_S44 & RD-F-40\\
        \hline
        T\_S45 & RD-F-42\\
        \hline
        T\_S46 & RD-F-43\\
        \hline
        T\_S47 & RD-F-44\\
        \hline
        T\_S48 & RD-F-45\\
        \hline
        T\_S49 & RD-F-46\\
        \hline
        T\_S50 & RD-F-47\\
        \hline
        T\_S51 & RD-F-48\\
        \hline
        T\_S52 & RD-F-49\\
        \hline
        T\_S53 & RD-F-50\\
        \hline
        T\_S54 & RD-F-51\\
        \hline
        T\_S55 & RD-F-52\\
        \hline
        T\_S56 & RD-F-53\\
        \hline
        T\_S57 & RD-F-54\\
        \hline
        T\_S58 & RD-F-55\\
        \hline
        T\_S59 & RD-F-57\\
        \hline
    \end{tabular}
\end{center}