\section{Qualità di processo}
Al fine di misurare e controllare la qualità dei processi nella realizzazione del progetto si è deciso di 
adottare lo standard ISO/IEC 12207:1995.
Gli obiettivi sono:
\begin{itemize}
    \item Controllare l'andamento dei processi.
    \item Migliorare i processi rispettando gli standard adottati.
\end{itemize}
Le metriche utilizzate si possono consultare nel documento \emph{NormeDiProgetto} e di
seguito ne vengono riportati i valori accettabili e ottimali.

\subsection{Processi Primari}
\subsubsection{Sviluppo}
Il processo consiste nello sviluppare il prodotto secondo le scelte architetturali individuate.
\paragraph{Obiettivo} \hfill \break
Controllare che venga fatta una buona analisi dei requisiti e venga sviluppata un buona architettura.

\paragraph{Metriche}
\begin{center}
    \renewcommand{\arraystretch}{1.8}
    \begin{tabular}{ |c|m{12em}|c|c|}
        \hline
        \textbf{Metrica} & \textbf{Nome} & \textbf{Valore accettabile} & \textbf{Valore ottimale} \\
        \hline
        M4VR & Variazione dei requisiti & $ \leq 5 $ & $ 0 $ \\
        \hline
        M8CC & Code Coverage & $ \geq 80\% $ & $ 100\% $ \\
        \hline
    \end{tabular}
\end{center}

\subsection{Processi Organizzativi}
\subsubsection{Gestione Organizzativa}
Il processo consiste nella gestione dei membri, dell'infrastruttura tecnologica utilizzata 
dal gruppo e nella pianificazione delle attività.
\paragraph{Obiettivo} \hfill \break
Garantire una buona pianificazione delle attività del team.

\paragraph{Metriche}
\begin{center}
    \renewcommand{\arraystretch}{1.8}
    \begin{tabular}{ |c|m{12em}|c|c|}
        \hline
        \textbf{Metrica} & \textbf{Nome} & \textbf{Valore accettabile} & \textbf{Valore ottimale} \\
        \hline
        M15VC & Variazione di Costo & $ \geq -100 $ & $ \geq 0 $  \\
        \hline
        M16VP & Variazione di Piano & $ \geq -7 $ & $ \geq 0 $\\
        \hline
    \end{tabular}
\end{center}