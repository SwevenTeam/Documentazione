\section{Qualità di processo}
Al fine di misurare e controllare la qualità dei processi nella realizzazione del progetto si è deciso di 
adottare lo standard ISO/IEC 12207:1995.
Gli obiettivi sono:
\begin{itemize}
    \item Controllare l'andamento dei processi.
    \item Migliorare i processi rispettando gli standard adottati.
\end{itemize}
Le metriche utilizzate si possono consultare nel documento \emph{NormeDiProgetto} e di
seguito ne vengono riportati i valori accettabili e ottimali.

\subsection{Processi Primari}
\subsubsection{Fornitura}
Il processo ha lo scopo di stabilire le principali attività e tempistiche di lavoro
al fine di rispettare i requisiti del prodotto.

\paragraph{Obiettivo}
Controllare i progressi delle attività, le tempistiche e i costi necessari al completamento del progetto.

\paragraph{Metriche}
\begin{center}
    \renewcommand{\arraystretch}{1.8}
    \begin{tabular}{ |c|m{12em}|c|c|}
        \hline
        \textbf{Metrica} & \textbf{Nome} & \textbf{Valore accettabile} & \textbf{Valore ottimale} \\
        \hline
        QMPRC-1 & Budgeted Cost of \newline Work Scheduled & $ \geq 0 $ & $ \leq BAC $ \\
        \hline
        QMPRC-2 & Budgeted Cost of Work \newline Performed & $ > 0 $ & $ \geq BCWS $ \\
        \hline
        QMPRC-3 & Actual Cost of Work \newline Performed & $ \geq 0 $ & $ \leq BCWP $ \\
        \hline
        QMPRC-4 & Cost Variance & $ \geq -15\% $ & $ \geq 0\% $ \\
        \hline
        QMPRC-5 & Estimated At Completion & $ \leq BAC + 5\% $ & $ \leq BAC $ \\
        \hline
        QMPRC-6 & Schedule Variance & $ \geq 10\% $ & $ \leq BAC $ \\
        \hline
    \end{tabular}
\end{center}

\subsubsection{Sviluppo}
Il processo consiste nello sviluppare il prodotto secondo le scelte architetturali individuate.

\paragraph{Obiettivo}
Controllare come, durante il processo di sviluppo, cambino i requisiti e vedere quanto
impatta sul lavoro del gruppo.

\paragraph{Metriche}
\begin{center}
    \renewcommand{\arraystretch}{1.8}
    \begin{tabular}{ |c|m{12em}|c|c|}
        \hline
        \textbf{Metrica} & \textbf{Nome} & \textbf{Valore accettabile} & \textbf{Valore ottimale} \\
        \hline
        QMPRC-7 & Requirement Stability \newline Index & $ \geq 80\% $ & $ \geq 90\% $ \\
        \hline
    \end{tabular}
\end{center}

\subsection{Processi di Supporto}
\subsubsection{Verifica}
Il processo ha lo scopo di trovare gli eventuali errori generati durante la realizzazione del prodotto.

\paragraph{Obiettivo}
Garantire una buona copertura in merito all'attività di verifica del lavoro svolto dal gruppo.

\paragraph{Metriche}
\begin{center}
    \renewcommand{\arraystretch}{1.8}
    \begin{tabular}{ |c|m{12em}|c|c|}
        \hline
        \textbf{Metrica} & \textbf{Nome} & \textbf{Valore accettabile} & \textbf{Valore ottimale} \\
        \hline
        QMPRC-8 & Test implementati & $ \geq 80\% $ & 100\% \\
        \hline
    \end{tabular}
\end{center}

\subsection{Processi Organizzativi}
\subsubsection{Gestione Organizzativa}
Il processo consiste nella gestione dei membri e dell'infrastruttura tecnologica
utilizzata dal gruppo.

\paragraph{Obiettivo}
Garantire la corretta gestione dell'infrastruttura utilizzata dal team.

\paragraph{Metriche}
\begin{center}
    \renewcommand{\arraystretch}{1.8}
    \begin{tabular}{ |c|m{12em}|c|c|}
        \hline
        \textbf{Metrica} & \textbf{Nome} & \textbf{Valore accettabile} & \textbf{Valore ottimale} \\
        \hline
        QMPRC-9 & Rischi individuati & $ \leq 6 $ & 0 \\
        \hline
    \end{tabular}
\end{center}