\section{A}
\subsection{Access Token}
Si veda voce Token.
\subsection{API Key}
Chiave necessaria per autorizzare le richieste alle API Rest fornite da Imola Informatica.
\subsection{API Rest}
Un insieme di definizioni e protocolli con i quali vengono realizzati e integrati software applicativi (API) conformi ai vincoli dello stile architetturale REST.
\subsection{Attività}
Si intende un insieme di lavoro, da consuntivare, svolta per un progetto durante la giornata lavorativa.
\subsection{Agile}
È una metodologia che nell'ingegneria del software indica un insieme di metodi di sviluppo; sono approcci iterativi, nei quali il software viene sviluppato e consegnato ai clienti con incrementi consecutivi uno all'altro.
\newpage
\section{B}
\section{C}
\subsection{Carte di controllo}
Le carte di controllo sono uno strumento statistico per lo studio e il controllo di processi ripetitivi, come ad esempio un processo produttivo in un'industria.
\subsection{CBC-MAC}
In crittografia, un Cipher Block Chaining Message Authentication Code, abbreviato CBC-MAC, è una tecnica per costruire un codice di autenticazione di messaggio usando un cifrario a blocchi. Il messaggio è crittografato con qualche algoritmo di crittografia a blocchi in modalità CBC per creare una catena di blocchi in cui ognuno di essi dipende dalla cifratura del blocco precedente. Questa interdipendenza assicura che un cambiamento ad un qualsiasi bit del testo in chiaro causerà un cambiamento nel blocco finale crittografato che non può essere predetto o calcolato senza conoscere la chiave di codifica.
\subsection{Chatbot}
Un Chatbot è un software capace di conversare con un utente in linguaggio naturale, comprendendone 
le intenzioni e rispondendo secondo le linee guida impartite dall'azienda, oppure in 
base ai dati di cui dispone. L'intelligenza artificiale aiuta il chatbot a comprendere meglio il 
contesto e il tono della conversazione.
\subsection{Chatterbot}
Una libreria python che facilita lo sviluppo di un chatbot permettendo di stabilire quale sia la logica che deve avere la risposta.
\subsection{Check-in}
Registrazione dell'ingresso di un dipendete nella sede aziendale.
\subsection{CMIS}
Content Management Interoperability Services (CMIS) è uno standard che definisce uno strato di astrazione per il controllo di
sistemi di document management e di repositories attraverso protocolli web.
\subsection{Continuous Integration}
Pratica che consiste nell'automatizzazione dell'integrazione del codice in progetti software che coinvolgono
molteplici programmatori. È considerata una "best practice" e permette di mantenere il codice sempre aggiornato e 
aderente agli standard del progetto.
\newpage
\section{D}
\section{E}
\subsection{EMT}
Si veda voce Sistema EMT.
\section{F}
\subsection{Flask}
Micro Framework scritto in Python che permette lo sviluppo e la visualizzazione di semplici applicazioni web.
\section{G}
\subsection{Github Action}
Sistema di Continuous Integration interno, offerto dalla piattaforma Github.
\subsection{Google Sheet}
Software utilizzato per la creazione e utilizzo di fogli di calcolo. Parte della suite offerta da Google, compatibile con fogli di stile Excel.
\section{H}
\subsection{HTTP}
L'HyperText Transfer Protocol è un protocollo usato per la trasmissione di dati sul web.
\section{I}
\subsection{Issue Tracking System}
Software che permette di controllare e facilitare il processo di sviluppo attraverso la gestione di attività.
\newpage
\section{J}
\section{K}
\section{L}
\section{M}
\subsection{MQTT}
Message Queue Telemetry Transport o abbreviato MQTT è un protocollo di messaggistica progettato per le situazioni in cui è richiesto
un basso impatto e dove la banda è limitata.
\section{N}
\section{O}
\subsection{Overriding}
Nella programmazione ad oggetti, l'overrding di un metodo permette ad una classe figlia di fornire un'implementazione di un metodo diversa da quella già presente nella classe genitore.
La versione del metodo da invocare verrà determinata dall'oggetto che la invocherà.
\newpage
\section{P}
\subsection{Piattaforma Riunioni}
Applicazione utilizzata per riunioni online, comunicazione tramite audio e video; permette inoltre la condivisione schermo.
\subsection{Presenza}
Si veda la voce check-in. 
\subsection{Product Baseline}
Fase nella quale il prodotto non è ancora pronto per il rilascio ma funziona correttamente.
\subsection{Proof of Concept}
Realizzazione incompleta di un progetto al fine di dimostrarne la fattibilità.
\subsection{PyCharm}
IDE sviluppato da JetBrains per la codifica di codice Python.
\subsection{Python}
Linguaggio di programmazione di alto livello. Supporta diversi tipi di paradigmi di programmazione, come programmazione
orientata agli oggetti e programmazione funzionale.
\section{Q}
\section{R}
\subsection{React JS}
React JS è una libreria open-source per lo sviluppo front-end, realizzata in JavaScript, il cui scopo è quello di implementare interfacce utente.
\subsection{Redmine}
Redmine è un software per la pianificazione di progetti e per il tracciamento delle segnalazioni di bug tramite interfaccia web.
\newpage
\section{S}
\subsection{Sistema EMT}
Applicativo aziendale interno ad Imola Informatica, sviluppato per effettuare operazioni di Check-in, tracciamento delle attività svolte e prenotazione delle postazioni in sede.
\subsection{SQL}
Structured Query Language è un linguaggio standardizzato per la gestione di database relazionali.
\subsection{StarUML}
Software utilizzato per la creazione di diagrammi UML. I diagrammi UML vengono utilizzati nel campo dell'Ingegneria del Software per fornire una rappresentazione standardizzata di un sistema.
\subsection{Stateful}
Le applicazioni stateful conservano i dati di una sessione per renderli disponibili anche alle sessioni successive.
\subsection{Stateless}
Un’applicazione stateless è senza stato, in quanto non salva i dati generati in una determinata sessione per utilizzarli nel corso di quelle successive. Ogni sessione viene pertanto eseguita senza alcuna memoria del pregresso.
\subsection{Stakeholder}
Qualsiasi soggetto direttamente o indirettamente coinvolto in un progetto o in un'attività inerente al progetto.
\subsection{Sticky Session}
Consentono al sistema il bilanciamento del carico, associando una sessione utente a una destinazione specifica.
\newpage
\section{T}
\subsection{Technology Baseline}
Avanzamento continuo delle tecnologie che faranno parte del prodotto finale, le cui modifiche saranno incrementali.
\subsection{Telegram}
Applicazione di messaggistica istantanea basata su Cloud, disponibile in versione Desktop e Mobile.
\subsection{Ticket}
Report creato da un dipendente aziendale per comunicare un problema o un bug.
\subsection{Token}  
"Permesso" consegnato all'utente dopo l'autenticazione, il quale gli consente di accedere a determinati servizi riservati.
\section{U}
\section{V}
\subsection{Version Control System}
Software che permette di gestire e tracciare i cambiamenti apportati ad un insieme di file.
\subsection{Visual Studio Code}
Editor di codice che permette l'integrazione con molteplici linguaggi e tipologie di file.
\section{W}
\section{X}
\section{Y}
\section{Z}
\subsection{Zoom}
Programma di videoconferenza sviluppato da Zoom Video Communications.
\subsection{Zoom Meetings}
Di veda la voce Zoom.

