\section{A}
\subsection{Access Token}
Vedere voce Token.
\subsection{Attività}
Si intende un insieme di lavoto, da consuntivare, svolta per un progetto durante la giornata lavorativa.
\subsection{Agile}
È una metodologia che nell'ingegneria del software indica un insieme di metodi di sviluppo, sono approcci iteratici nei quali il software viene sviluppato e consegnato ai clienti con incrementi successivi.
\newpage
\section{B}
\section{C}
\subsection{Carte di controllo}
Le carte di controllo sono uno strumento statistico per lo studio e il controllo di processi ripetitivi, come ad esempio un processo produttivo in un industria.
\subsection{Chatbot}
Un Chatbot è un software capace di conversare con un utente in linguaggio naturale, comprendendone 
le intenzioni e rispondendo secondo le linee guida impartite dall'azienda oppure in 
base ai dati di cui dispone. L'intelligenza artificiale aiuta il chatbot a comprendere meglio il 
contesto e il tono della conversazione.
\subsection{Check-in}
Registrazione dell'ingresso di un dipendete nella sede aziendale.
\subsection{CMIS}
Content Management Interoperability Services (CMIS) è uno standard che definisce un strato di astrazione per il controllo di
sistemi di document management e di repositories attraverso protocolli web.
\subsection{Continuous Integration}
Pratica che consiste nell'automatizzazione dell'integrazione del codice in progetti software che coinvolgono
molteplici programmatori. È considetata una "best practice" e permette di mantenere il codice sempre aggiornato e 
aderente agli standard del progetto.
\newpage
\section{D}
\section{E}
\subsection{EMT}
si veda voce Sistema EMT.
\newpage
\section{F}
\subsection{Flask}
Micro Framework scritto in Python che permette lo sviluppo e la visualizzazione di semplici applicazioni web.
\newpage
\section{G}
\subsection{Github Action}
Sistema di Continuous Integration interno ed offerto dalla piattaforma Github.
\subsection{Google Sheet}
Software utilizzato per la creazione e utilizzo di fogli di calcolo. Parte della suite offerta da Google, compatibile con fogli di stile Excel.
\newpage
\section{H}
\section{I}
\subsection{Issue Tracking System}
Software che permette di controllare e facilitare il processo di sviluppo attraverso la gestione di attività.
\newpage
\section{J}
\section{K}
\section{L}
\section{M}
\subsection{MQTT}
Message Queue Telemetry Transport o abbreviato MQTT è un protocollo di messaggistica progettato per le situazioni in cui è richiesto
un basso impatto e dove la banda è limitata.
\newpage
\section{N}
\section{O}
\section{P}
\subsection{Piattaforma Riunioni}
Applicazione utilizzata per riunioni online, comunicazione tramite audio e video, permette inoltre la condivisione schermo.
\subsection{Presenza}
si veda la voce check-in. 
\subsection{Product Baseline}
Fase nella quale il prodotto non è ancora pronto per il rilascio ma funziona correttamente.
\subsection{Proof of Concept}
Realizzazione incompleta di un progetto al fine di dimostrarne la fattibilità.
\subsection{PyCharm}
IDE sviluppato da JetBrains per la codifica di codice Python.
\subsection{Python}
Linguaggio di programmazione di alto livello. Supporta diversi tipi di paradigmi di programmazione, come programmazione
orientata agli oggetti e programmazione funzionale.
\newpage
\section{Q}
\section{R}
\subsection{Redmine}
Redmine è un software per la pianificazione di progetti e per il tracciamento delle segnalazioni di bug tramite interfaccia web.
\newpage
\section{S}
\subsection{Sistema EMT}
Applicativo aziendale interno ad Imola Informatica, sviluppato per effettuare operazioni di Check-in, tracciamento attività svolte e prenotazione delle postazioni in sede.
\subsection{StarUML}
Software utilizzato per la creazione di diagrammi UML. I diagrammi UML vengono utilizzati nel campo dell'Ingegneria del Software per fornire una rappresentazione standardizzata di un sistema.
\subsection{Stakeholder}
Qualsiasi soggetto direttamente o indirettamente coinvolto in un progetto o in un'attività inerente al progetto.
\newpage
\section{T}
\subsection{Technology Baseline}
Avanzamento continuo delle tecnologie che faranno parte del prodotto finale, le cui modifiche saranno incrementali.
\subsection{Telegram}
Applicazione di messaggistica istantanea basata su Cloud, disponibile in versione Desktop e Mobile.
\subsection{Ticket}
Report creato da un dipendente aziendale per comunicare un problema o un bug.
\subsection{Token}  
"Permesso" consegnato all'utente dopo l'autenticazione che gli consente di accedere a dei servizi riservati.
\newpage
\section{U}
\section{V}
\subsection{Version Control System}
Software che permette di gestire e tracciare i cambiamenti apportati ad un insieme di file.
\subsection{Visual Studio Code}
Editor di codice che permette l'integrazione con molteplici linguaggi e tipologie di file.
\newpage
\section{W}
\section{X}
\section{Y}
\section{Z}
\subsection{Zoom}
Programma di videoconferenza sviluppato da Zoom Video Communications.
\subsection{Zoom Meetings}
vedere voce Zoom.

