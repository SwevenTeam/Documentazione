\subsection{Formazione}
\subsubsection{Scopo}
Il processo di formazione è un processo atto a fornire e mantenere le competenze per il personale. Quello che ci si attende da questo processo è assicurare la qualità del lavoro tramite le competenze dei membri del gruppo. 
In particolar modo si cerca di garantire le competenze di tutti i membri del gruppo per quanto riguarda:
\begin{itemize}
	\item le tecnologie per la stesura della documentazione.
	\item gli strumenti da utilizzare per l'organizzazione.
	\item i linguaggi di programmazione e gli ambienti di sviluppo.
\end{itemize}
\subsubsection{Aspettative}
Per la formazione, ogni membro del gruppo è tenuto a provvedere alla propria, studiando le tecnologie necessarie. Tutti i membri del gruppo devono cercare di rimanere allineati sulle competenze necessarie ai vari ambiti del progetto. E dunque importante che i membri più esperti in determiti ambiti, condividano le loro competenze con chi ha qualche mancanza.

\subsubsection{Attività}
Le attività di questo processo prevedono la formazione individuale, la condivisione delle competenze e se necessario la formazione di gruppo. E necessario reperire le giuste fonti per la formazione dando priorità al materiale fornito dal proponente e dai docenti. Le competenze per le quali è richiesta la formazione e le relative fonti sono di seguito elencate :
\begin{itemize}
	\item Organizzazione
		\begin{itemize}		
			\item \textbf{GitHub} : \href{https://docs.github.com/en}{GitHub documentation}
		\end{itemize}
	\item Documentazione
		\begin{itemize}
			\item \textbf{Latex} : \href{https://www.latex-project.org/help/}{latex documentation}
		\end{itemize}
	\item Sviluppo
		\begin{itemize}
			\item \textbf{Python} : \href{https://www.html.it/guide/guida-python/}{python guide html.it}
			\item \textbf{API Rest}, \href{https://www.redhat.com/it/topics/api/what-is-a-rest-api}{Articolo API Rest RedHat}
			\item \textbf{Chatterbot}, \href{https://chatterbot.readthedocs.io/en/stable/}{chatterbot documentation}
		\end{itemize}
\end{itemize}