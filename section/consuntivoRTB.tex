In questa sezione si consuntivano le ore e il reale costo delle varie fasi di lavoro elencate nel 
capitolo 4 e preventivate nel capitolo 5. \newline
Per ogni fase verranno scritti i seguenti punti, evidenziando le differenze rispetto al preventivo:
\begin{enumerate}
    \item Consuntivo orario
    \item Consuntivo economico
    \item Considerazioni
\end{enumerate}
In particolare nelle considerazioni si motiva l'eventuale scostamento notevole rispetto al preventivo
e di conseguenza potrebbe essere necessario rimaneggiare il preventivo delle fasi successive.
Il bilancio della tra consuntivo e preventivo può essere:
\begin{itemize}
    \item \textbf{Negativo: } spesa minore rispetto al preventivo
    \item \textbf{Uguale: } nessuna differenza di spesa tra preventivo e consuntivo
    \item \textbf{Positivo: } spesa maggiore rispetto al preventivo
\end{itemize}
Per bilancio complessivo si intende sommare le differenze dei bilanci precedenti 
fino al bilancio del momento.

\subsection{RTB}
\subsubsection{Baseline documentale}
\paragraph{Consuntivo orario}
\begin{center}
	\renewcommand{\arraystretch}{1.8} %aumento ampiezza righe
	\begin{tabular}{ |m{8em}|c|c|c|c|c|c|c| }
	\hline
	\textbf{Membro} & \textbf{Re} & \textbf{Am} &  \textbf{An} &  \textbf{Pt} &  \textbf{Pg} &  \textbf{Ve} &  \textbf{Totale}\\
    \hline
    Irene Benetazzo   & 4 (+1) & -      & 0 (-3) & - & - & -     & \textbf{4} (-2) \\
    \hline
    Tommaso Berlaffa  & -      & 3 (-2) & -      & - & - & 1      & \textbf{4} (-2) \\
    \hline
    Mattia Episcopo   & -      & 3 (-3) & 2 (+2) & - & - & -      & \textbf{5} (-1) \\
    \hline
    Pietro Macrì      & -      & 4 (-1) & -      & - & - & 1 (-1) & \textbf{5} (-2) \\
    \hline
    Qi Fan Andrea Pan & -      & 3 (-1) & 2 (-1) & - & - & -      & \textbf{5} (-2) \\
    \hline
    Matteo Pillon     & -      & 4 (-3) & -      & - & - & -      & \textbf{4} (-3) \\
    \hline
    Samuele Rizzato   & -      & 3 (-1) & 2 (-1) & - & - & -      & \textbf{5} (-2) \\
    \hline
    \textbf{Totale ore} & \textbf{4} (+1) & \textbf{20} (-11) &  \textbf{6} (-3) &  \textbf{0} &  \textbf{0} &  \textbf{2} (-1) &  \textbf{32} (-14)\\
    \hline
	\end{tabular}
\end{center}
\begin{figure}[H]
    \centering\includegraphics{images/consuntivo/RTB-documentale-ore.png}
\end{figure}

\paragraph{Consuntivo economico}
\begin{center}
	\renewcommand{\arraystretch}{1.8} %aumento ampiezza righe
	\begin{tabular}{ |m{6em}|c|c|c|c|c|c|c| }
	\hline
	\textbf{Ruolo} & \textbf{Re} & \textbf{Am} &  \textbf{An} &  \textbf{Pt} &  \textbf{Pg} &  \textbf{Ve} &  \textbf{Totale}\\
    \hline
    Totale ore & 4 & 20 & 6 & 0 & 0 & 2 & \textbf{32}\\
    \hline
    Costo \euro/h & 30\euro/h & 20\euro/h & 25\euro/h & 25\euro/h & 15\euro/h & 15\euro/h & \\
    \hline
    \textbf{Totale costo} & \textbf{120\euro} & \textbf{400\euro} &  \textbf{150\euro} & \textbf{0\euro} &  \textbf{0\euro} &  \textbf{30\euro} &  \textbf{700\euro} \\
    & (+30\euro) & (-120\euro) & (-75\euro) &  &  & (-15\euro) & (-280\euro) \\
    \hline
	\end{tabular}

    \begin{figure}[H]
       \centering\includegraphics{images/consuntivo/RTB-documentale-costo.png}
    \end{figure}
\end{center}

\paragraph{Considerazioni}
Erano state preventivate più ore (e giorni) di lavoro per gli amministratori, dato il cospicuo 
numero di pagine dei documenti e si pensava fosse necessario più studio anche per gli analisti.
Quindi questa fase si chiude 7 giorni prima del previsto cioè il 26 Aprile, con un risparmio di 280\euro. 
\begin{center}
	\renewcommand{\arraystretch}{1.8} %aumento ampiezza righe
	\begin{tabular}{ | l |c|c| }
    \hline
    & \textbf{Ore} & \textbf{Costo} \\
	\hline
    \textbf{Consuntivo} & 32 & 700\euro \\
    \hline
    \textbf{Preventivo} & 46 & 980\euro \\
    \hline
    \textbf{Bilancio fase} & -14 & -280\euro \\
    \hline
    \textbf{Bilancio complessivo} & \textbf{-14} & \textbf{-280\euro} \\
    \hline
    \end{tabular}
\end{center}

Di conseguenza si decide di partire subito con la fase successiva: Baseline dei Requisiti di cui inizialmente 
è stata prevista una sola settimana di lavoro, ma dato la grande mole di casi d'uso da identificare e 
analizzare nel progetto probabilmente verrà usato parte del risparmio di questa fase.

\subsubsection{Baseline dei requisiti}

\subsubsection{Baseline delle tecnologie}