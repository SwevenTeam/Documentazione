\subsection{Scopo del documento}
Il seguente documento è necessario per organizzare la suddivisione dei lavori all'interno del gruppo e la conseguente realizzazione del progetto. Per ogni attività verranno dunque definiti i seguenti attributi: 
\begin{itemize}
    \item Rischi connessi allo svolgimento dell'attività
    \item Attribuzione di un ruolo ad ogni membro del team per consentirne lo svolgimento
    \item Preventivo risorse necessarie per portarla a termine
    \item Tempo e risorse effettivamente impiegate per la realizzazione
    \item Analisi generale dell'attività svolta
\end{itemize}
La definizione di tali attributi permette di organizzare il lavoro in maniera efficiente in modo tale da consentire al gruppo di lavorare in parallelo. 

\subsection{Scopo del capitolato}
Lo scopo di tale progetto è quello di sviluppare un Chatbot che interfacciandosi con software aziendali spesso complessi e dispersivi, semplifichi i compiti che i dipendenti devono svolgere. In particolare vengono individuate le seguenti operazioni: 
\begin{itemize}
    \item Tracciamento della presenza in sede (\textbf{EMT}\textsubscript{G})
    \item Rendiconto attività svolte quotidianamente (\textbf{EMT}\textsubscript{G})
    \item Apertura del cancello aziendale (\textbf{MQTT}\textsubscript{G})
    \item Creazione di una riunione in un servizio esterno
    \item Servizio di ricerca documentale (\textbf{CMIS}\textsubscript{G})
    \item Creazione e tracciamento di bug (\textbf{Redmine}\textsubscript{G})
\end{itemize}

\subsection{Glossario}
Per assicurare la massima fruibilità e leggibilità del documento, il team SWEven ha deciso di creare un documento denominato \textit{Glossario} il cui scopo sarà quello di contenere le definizioni dei termini ambigui o specifici del progetto. Sarà possibile riconoscere i termini presenti al suo interno in quanto terminanti con la lettera \textit{G} posta come pedice della parola stessa. 
\subsection{Riferimenti}

\subsubsection{Normativi}
\begin{itemize}
    \item Norme di progetto 1.0.0
\end{itemize}

\subsubsection{Informativi}
\begin{itemize}
    \item Analisi dei requisiti 1.0.0
    \item  \href{https://www.math.unipd.it/~tullio/IS-1/2021/Progetto/C1.pdf}{\color{blue} Capitolato di appalto C1 - BOT4ME}
    \item Software Engineering - Ian Sommerville: 10th Edition
    \begin{itemize}
        \item Capitolo 22 - Project Management
        \item Capitolo 23 - Project Planning
    \end{itemize}
    \item \href{https://www.math.unipd.it/~tullio/IS-1/2021/Dispense/T05.pdf}{\color{blue} Slide del corso - Ciclo di vita del software}
    \item \href{https://www.math.unipd.it/~tullio/IS-1/2021/Dispense/T06.pdf}{\color{blue} Slide del corso - Gestione di progetto}
\end{itemize}

\subsection{Programma revisioni}
Durante lo svolgimento del progetto sono previste 3 diverse revisioni, il cui scopo è quello di verificare il corretto avanzamento del lavoro e la validità di quanto prodotto fino ad allora. Il gruppo prevede di effettuare tali revisioni secondo lo schema di seguito riportato: 
\begin{itemize}
    \item \textbf{Requirement and Tecnology Baseline}: settimana dal    2022-05-02 al 2022-05-09
    \item \textbf{Product Baseline}: in corso di definizione.
    \item \textbf{Customer Acceptance}: in corso di definizione.
\end{itemize}