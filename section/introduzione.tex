\subsection{Scopo del documento}
Il seguente documento è necessario per organizzare la suddivisione dei lavori all'interno del gruppo e la conseguente realizzazione del progetto. Per ogni attività verranno dunque definiti i seguenti attributi: 
\begin{itemize}
    \item Rischi connessi allo svolgimento dell'attività
    \item Attribuzione di un ruolo ad ogni membro del team per consentirne lo svolgimento
    \item Preventivo risorse necessarie per portarla a termine
    \item Tempo e risorse effettivamente impiegate per la realizzazione
    \item Analisi generale dell'attività svolta
\end{itemize}
La definizione di tali attributi permette di organizzare il lavoro in maniera efficiente in modo tale da consentire al gruppo di lavorare in parallelo. 

\subsection{Scopo del capitolato}
Lo scopo di tale progetto è quello di sviluppare un Chatbot, che interfacciandosi con software aziendali, spesso complessi e dispersivi semplifichi i compiti che i dipendenti devono svolgere. In particolare vengono individuate le seguenti operazioni: 
\begin{itemize}
    \item Tracciamento della presenza in sede (\textbf{EMT}\textsubscript{G})
    \item Rendiconto attività svolte quotidianamente (\textbf{EMT}\textsubscript{G})
    \item Apertura del cancello aziendale (\textbf{MQTT}\textsubscript{G})
    \item Creazione di una riunione in un servizio esterno
    \item Servizio di ricerca documentale (\textbf{CMIS}\textsubscript{G})
    \item Creazione e tracciamento di bug (\textbf{Redmine}\textsubscript{G})
\end{itemize}

\subsection{Glossario}
Per assicurare la massima fruibilità e leggibilità del documento, il team SWEven ha deciso di creare un documento denominato \textit{Glossario} il cui scopo sarà quello di contenere le definizioni dei termini ambigui o specifici del progetto. Sarà possibile riconoscere i termini presenti al suo interno in quanto terminanti con la lettera \textit{G} posta come pedice della parola stessa. 
\subsection{Riferimenti}

\subsubsection{Normativi}
\begin{itemize}
    \item IEEE 830-1998 Specifica dei requisiti software
    \item Norme di progetto 1.0.0
    \item Verbale esterno 2022-03-18
    \item Verbale esterno 2022-04-15
\end{itemize}

\subsubsection{Informativi}
\begin{itemize}
    \item \href{https://www.math.unipd.it/~tullio/IS-1/2021/Progetto/C1.pdf}{\color{blue} Capitolato di appalto C1 - BOT4ME}
    \item \href{https://www.math.unipd.it/~tullio/IS-1/2021/Dispense/T07.pdf}{\color{blue} Slide del corso - Analisi dei requisiti}
    \item \href{https://www.math.unipd.it/~rcardin/swea/2022/Diagrammi%20Use%20Case.pdf}{\color{blue} Slide del corso - Diagrammi dei casi d'uso}
\end{itemize}
\newpage