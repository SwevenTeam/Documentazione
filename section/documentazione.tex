\subsection{Documentazione}
\subsubsection{Scopo}
Lo scopo di questa sezione è di riportare tutte le regole, template e decisioni 
per la stesura di tutti i documenti del gruppo Sweven. \newline
Tutti i documenti verranno scritti in \LaTeX.

\subsubsection{Ciclo di vita dei documenti}
Il documento viene innanzitutto pianificato, cioè ci si chiede perchè è necessario il documento
e si pensa al suo contento. Gli amministratori nel ruolo di redattori, creano e scrivono 
il documento, poi i verificatori lo controllano e il responsabile approva il documento 
finito e pronto ad essere ufficialmente pubblicato. \newline
Se i verificatori trovano errori superficiali e/o di ortografia correggono immediatamente,
mentre se l'errore è più profondo o è necessario riscrivere delle parti allora si
incarica nuovamente il redattore di sistemare e poi il verificatore controllerà nuovamente.

\subsubsection{Versionamento dei documenti}
I documenti sono redatti in maniera incrementale e in più momenti, quindi per una migliore 
gestione si tengono versionati.
\newline Versione: x.y.z
\begin{description}
    \item \textbf{x} è il valore più grande e indica quale versione è stata ufficialmente pubblicata.
            Viene aggiornato solo dal responsabile quando l'approvazione del documento ha esito positivo
            e si riportano a zero sia y che z;
    \item \textbf{y} è il valore che indica lo stato di verifica globale del documento. Viene aggiornato dal verificatore 
            quando la verifica globale del documento ha esito positivo, e si riporta a zero z;
    \item \textbf{z} è il valore che indica lo stato di stesura e relativa verifica di singole parti del documento. 
            Viene aggiornato dal verificatore quando verifica positivamente quella sezione scritta dal redattore.
\end{description}
A fine progetto è necessario che tutti i documenti saranno alla versione x.0.0.

\subsubsection{Suddivisione documenti interni ed esterni}
I documenti si suddividono in interni ed esterni in base allo scopo del documento e ai destinatari.
Nei documenti interni i destinatari sono i membri del gruppo Sweven, nei documenti esterni anche i 
professori committenti e l'azienda proponente. \newline
\textbf{Interni:}
\begin{itemize}
    \item \textbf{Verbali} di riunioni interne tra i componenti del gruppo Sweven
    \item \textbf{Glossario} nei destinatari si è deciso di mettere tutti per conoscenza
    \item \textbf{Norme di Progetto} nei destinatari si è deciso di mettere tutti per conoscenza
\end{itemize}

\textbf{Esterni:}
\begin{itemize}
    \item \textbf{Verbali} di riunioni con il gruppo Sweven e l'azienda proponente
    \item \textbf{Studio di Fattibilità}
    \item \textbf{Candidatura}
    \item \textbf{Analisi dei Requisiti}
    \item \textbf{Piano di Progetto}
    \item \textbf{Piano di Qualifica}
\end{itemize}

\subsubsection{Documenti e template}
La prima sottosezione illustra il template comune a tutti i documenti.\\
Nelle sottosezioni successive, una per ogni documento, è scritto lo scopo 
del documento ed eventuali particolarità di template.
\paragraph{Template comune a tutti i documenti} \hfill \break
Tutti i documenti vengono suddivisi in varie sezioni e ad ognuna corrisponde 
un file, in modo tale da rendere più agevole l'aggiornamento e la revisione dei file, 
oltre a garantire la possibilità di lavorare in contemporanea in maniera 
asincrona allo stesso file (è sufficiente lavorare in sezioni diverse).

\begin{enumerate}
    \item \textbf{configuration}:
            costituisce il file principale del documento, contiene tutti i comandi, 
            i pacchetti necessari e riporta le regole generali del documento come 
            i margini, lo stile della pagina, l'intestazione, la numerazione. 
            Inoltre contiene il link a tutti gli altri file che costituiscono le 
            varie parti del documento.

    \item \textbf{frontespizio}:
            il file rappresenta la prima pagina del documento, e quindi in alto si
            è lasciato un notevole spazio bianco, il contenuto è tutto centrato e 
            inizia con il nome del documento, poi l'immagine del logo con sotto 
            nome ed email del gruppo Sweven Team. Poi c'è una tabella centrata 
            con due colonne (visibile solo riga superiore e bordo centrale delle colonne), 
            in cui vengono riporte le varie informazioni, che sono state settate all'inizio 
            del file configuration:
            \begin{itemize}
                \item Versione
                \item Uso
                \item Destinatari
                \item Stato
                \item Redattori
                \item Verificatori
                \item Approvatori
            \end{itemize}
            L'elenco dei destinatari, redattori, verificatori e approvatori 
            può essere più di una persona e affinchè i nomi siano scritti uno sotto 
            l'altro, nel comando si può scrivere : 
            \begin{center}
            	Nome\textbackslash\textbackslash 
            	\& Nome\textbackslash\textbackslash \& Nome\textbackslash\textbackslash.
            \end{center}.
            \newline Dopo altro spazio si trova la sintesi del documento, questa frase
            ha lo scopo di rappresentare molto sinteticamente il contenuto del documento 
            così da permettere al lettore di capire dal frontespizio se è di suo interesse
            o meno.
    
    \item \textbf{diario delle modifiche}:
            il diario dellle modifiche è costituito da una tabella con 5 colonne e tutti i 
            bordi anche delle righe visibili. Le 3 colonne Versione, Data, Ruolo (ultima colonna) 
            sono state impostate center mentre per le altre due Descrizione e Autore è 
            rispettivamente stata data la dimensione di 12em e 7em, inoltre il testo è allineato 
            a sinistra. 
            Viene aumentata l'ampiezza righe ad 1.8 così da non avere tutte le righe attaccate. \newline
            La nuova riga la si aggiunge sempre ad inizio tabella così da ottenere che la prima 
            riga comunica quale è stata l'ultima modifica al documento e la versione qui riportata 
            deve corrispondere a quella scritta nel frontespizio. \newline
            La data viene scritta in formato americano aaaa-mm-gg
            Nella colonna autore non si suddividono in sillabe i nomi o cognomi, se capita usare
            Nome \textbackslash newline Cognome.


            Dopo il diario delle modifiche, nel file configuration c'è il comando di creare 
            l'indice del documento.

    \item \textbf{contenuto}: 
            le pagine successive contengono il contenuto vero e proprio del documento seguendo 
            l'indice (possono essere anche più file). In tutte le pagine diverse dal frontespizio è prevista un'intestazione in 
            grigio in cui a sinistra c'è il nome del gruppo e a destra si riporta il nome del documento.
            Mentre nel piè di pagina viene riportato il numero della pagina rispetto alle pagine totali.
\end{enumerate}

\paragraph{Verbali} \hfill \linebreak
Rispetto a quanto scritto nella sezione 3.1.5, sono inoltre presenti altri due file template
per la stesura dei verbali: "informazioni" e "conclusioni-decisioni". \\
L'ordine delle varie parti sarà il seguente:
\begin{enumerate}
        \item \textbf{configuration}
        \item \textbf{frontespizio}
        \item \textbf{diario delle modifiche}
        \item \textbf{informazioni}:
                 questa pagina contiene le informazioni della riunione e l'ordine del giorno previsto.
                Le informazioni prevedono data, ora, luogo, lista partecipanti ed eventuali assenti.
        \item \textbf{svolgimento}:
                l'equivalente del "contenuto" degli altri documenti, sarà un unico file e si 
                svilupperanno i punti scritti nell'ordine del giorno.
        \item \textbf{conclusioni-decisioni}:
                la sezione conclusioni riassume in maniera sintetica quanto detto durante la riunione, 
                riassume ciò che è da fare nel breve futuro e se già stabilita si indica la data 
                della prossima riunione. \newline
                La tabella del tracciamento delle decisioni, costituita da due colonne, serve per 
                riportare in maniera schematica le decisioni prese e assegnare loro un codice 
                VI\_aaaa-mm-gg oppure VE\_aaaa-mm-gg in base a verbale interno o esterno, così 
                se necessario questo codice può essere usato per riferirsi alla decisione presa 
                e leggendo il verbale ne troverà la spiegazione.
\end{enumerate}
Il verbale è un documento che viene scritto, verificato e approvato una sola volta, 
in quanto nel tempo non si aggiorna il verbale di una vecchia riunione.
Inoltre il redattore non verrà affiancato dal verificatore durante la scrittura ma avverrà solo al termine. 
Quindi all'esito positivo della verifica si userà direttamente il codice 0.1.0 e poi 1.0.0 con l'approvazione.


\paragraph{Glossario}  \hfill \linebreak
Il glossario è un documento che ha lo scopo di spiegare il significato di alcuni termini usati 
all'interno degli altri documenti, così da facilitare la comprensione di essi. \newline
Il glossario non prevede la sezione di introduzione, si è deciso che si userà l'ordine alfabetico 
così da facilitarne la ricerca, nell'indice compariranno tutte le parole presenti nel documento. 
Ogni lettera dell'alfabeto costituirà una nuova sezione e ogni parola avrà la sua sottosezione 
con scritto il significato. \newline
Quando nei documenti si usa una parola e si vuole rimandare al glossario per la spiegazione 
avrà la G al pedice, ciò si inserisce usando il comando \textbackslash glossario. 
Per praticità e averne lo stile già impostato, questo comando è stato definito nel file 
configuration con la scrittura in corsivo del termine e l'aggiunta del pedice.

\paragraph{Norme di Progetto}  \hfill \break
Le norme di progetto, cioè questo documento, hanno lo scopo di redigere e riportare tutte le norme 
e le decisioni del gruppo Sweven. Il documento sarà composto da tre grandi sezioni: \\
processi primari, processi di supporto e processi organizzativi.

\paragraph{Studio di Fattibilità} \hfill \break
Lo studio di fattibilità è un documento scritto durante la scelta del capitolato per l'appalto con 
lo scopo di analizzare ogni capitolato rimasto scrivendo pregi, difetti e le criticità rilevate dai 
membri del gruppo.

\paragraph{Candidatura} \hfill \break
La candidatura è il documento ufficiale mediante il quale il gruppo si candida alla gara d'appalto 
per l'assegnazione del capitolato, dichiarando anche il preventivo dei costi e della data di consegna.

\paragraph{Analisi dei Requisiti}  \hfill \break
L'analisi dei requisiti è un documento scritto durante la fase di analisi del progetto scrivendo in 
dettaglio i casi d'uso e poi poter mappare i requisiti del prodotto.

\paragraph{Piano di Progetto}  \hfill \break
Il piano di progetto è un documento in cui il gruppo dichiara come verrà gestito lo sviluppo del progetto,
scrivendo in dettaglio la pianificazione con relativo preventivo e infine viene riportato il consuntivo effettivo.

\paragraph{Piano di Qualifica} \hfill \break
Il piano di qualifica è un documento in cui il gruppo dichiara come analizza e verifica che i documenti e 
il prodotto siano di qualità ponendosi sia soglie minime che soglie desiderabili.

\subsubsection{Convenzioni tipografiche} 
In questa parte si riportano le varie convenzioni per i documenti decise all'interno del gruppo.
\paragraph{Nome del file}  \hfill \break
I documenti pdf vanno nominati con il nome del documento iniziando con la lettera maiuscola 
e ogni nuova parola inizia con la lettera maiuscola (comprese le proposizioni). Per i verbali in 
aggiunta va scritta anche la data all'americana usando i trattini e un trattino stacca la data 
dal nome; esempi: NormeDiProgetto, VerbaleInterno-2022-03-31. \newline
Il gruppo dà importanza ai nomi dei file che vengono ufficialmente pubblicati, mentre per gli altri documenti viene data più libertà al creatore del file, invitando comunque 
all'utilizzo dell'underscore (trattino basso) o delle maiuscole per separare diverse parole.

\paragraph{Stile del testo}
\begin{itemize}
        \item \textbf{Normale 12 pt} in generale tutto il testo dei documenti
        \item \textbf{Grassetto} tutti i titoli che si differenziano per la grandezza automatica di \LaTeX
        \item \textbf{Corsivo} i riferimenti ad altri documenti, alle parole del glossario
        \item \textbf{Elenchi} gli elementi dell'elenco vanno in grassetto solo se è prevista una descrizione. \\
                                Nel caso di elenchi annidati, il grassetto si usa solo nell'elenco principale.
\end{itemize}

\paragraph{Scrittura della data e dell'ora} \hfill \break
La data va sempre scritta nel formato americano aaaa-mm-gg. \\
Per l'ora si usa il formato a 24 ore hh:mm.

\paragraph{Sigle}
\begin{itemize}
        \item \textbf{Nomi dei documenti}
                \begin{itemize}
                        \item VI: Verbale interno
                        \item VE: Verbale esterno
                        \item G: Glossario
                        \item NdP: Norme di progetto
                        \item AdR: Analisi dei requisiti
                        \item PdP: Piano di progetto
                        \item PdQ: Piano di qualifica
                        \item MU: Manuale utente
                \end{itemize}
        \item \textbf{Revisioni di avanzamento}
                \begin{itemize}
                        \item RTB: Requirements and Technology Baseline
                        \item PB: Product Baseline
                        \item CA: Customer Acceptance
                \end{itemize}
        \item \textbf{Ruoli}
                \begin{itemize}
                        \item Re: Responsabile
                        \item Am: Amministratore
                        \item An: Analista
                        \item Pt: Progettista
                        \item Pg: Programmatore
                        \item Ve: Verificatore
                \end{itemize}
\end{itemize}

\subsubsection{Metriche}
\textbf{M13IG}:
    \begin{itemize}
      \item Nome: Indice di Gulpease;
      \item Descrizione: numero intero positivo rappresentante la leggibilità di un testo italiano. Il valore è compreso tra 0 e 100:
           \begin{itemize}
                \item < 80 il testo è difficile da leggere per chi ha la licenza elementare;
                \item < 60 il testo è difficile da leggere per chi ha la licenza media;
                \item < 40 il testo è difficile da leggere per chi ha la licenza superiore.
           \end{itemize}
      \item Scopo: rendere il documento leggibile;
      \item Formula:
      \begin{center}
        $ \textit{89+}\frac{\textit{300*(\#frasi) - 10*(\#parole)}}{\textit{\#lettere}}$
      \end{center}
\end{itemize}