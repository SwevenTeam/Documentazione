\section{Svolgimento}
\subsection{Discussione metriche}
In seguito ad una fase di studio, avvenuta nelle settimane precedenti, nella prima parte della riunione il gruppo si confronta sulle metriche trovate, in modo da inserire all'interno dei \newline documenti quelle più utili ad un avanzamento corretto del lavoro da svolgere. 
Durante una fase iniziale di verifica dei documenti stilati in questo periodo, ci si è inoltre accorti dell'utilizzo di una nomenclatura per le metriche diversa, in alcuni punti, rispetto a quanto esplicitato all'interno delle Norme di Progetto. Il gruppo prende atto della svista e si appresta a risolvere l'errore prima della fine di questa fase.

\subsection{Fine RTB baseline requisiti}
Con l'incontro di oggi il gruppo ha fatto il punto sul lavoro svolto fino ad ora, la cui fine sarà sancita dalla verifica dei documenti redatti in tale periodo. Secondo quando deciso nelle riunioni precedenti, i verificatori procederrano con la validazione dei documenti in seguito alle ultime modifiche apportate durante il colloquio odierno.

\subsection{Inizio RTB baseline tecnologie}
In seguito alla verifica dei documenti redatti nella fase precedente, dal giorno Mercoledì 11 Maggio avrà inizio la nuova fase: RTB Baseline tecnologie, avente una durata di due settimane. 
Lo scopo di questa fase è quello di proseguire e finire i documenti necessari allo sviluppo del progetto, ma soprattutto di realizzare un PoC\textsubscript{G} che mostri una prima versione dell'applicativo. Grazie anche alle informazioni fornite dall'azienda Imola, nei vari colloqui fatti, il gruppo è orientato sulla realizzazione di un PoC\textsubscript{G}, suddiviso in lato Client\textsubscript{G} e lato Server\textsubscript{G}, il cui compito sarà quello di mostrare un funzionamento primitivo di quello che sarà il chatbot finale. L'ideale è che l'utente, tramite un'interfaccia web, possa mandare un messaggio al chatbot, il quale fornirà una banale risposta. 
Di seguito vengono mostrati come sono stati suddivisi i compiti inizialmente: 
\begin{itemize}
    \item Il ruolo di Responsabile, per la gestione dell'avanzamento di questa fase, è stato affidato a Qi Fan Andrea Pan.
    \item Mattia Episcopo e Matteo Pillon si occuperanno della realizzazione della parte Server\textsubscript{G} relativa al PoC\textsubscript{G}.
    \item Pietro Macrì e Tommaso Berlaffa si occuperanno invece dello sviluppo del lato Client\textsubscript{G}.
    \item Per questa prima suddivisione Irene Benetazzo e Samuele Rizzato si occupperanno della sistemazione finale dei documenti redatti, successivamente Irene si unirà al team lato Client\textsubscript{G} e Samuele al team lato Server\textsubscript{G}. 
\end{itemize}
\newpage