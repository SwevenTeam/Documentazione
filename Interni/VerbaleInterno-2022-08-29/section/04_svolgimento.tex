\section{Svolgimento}
\subsection{Dimostrazione del prodotto Bot4Me}
Matteo illustra come si è completata l'integrazione Client e Server, gestendo anche il salvataggio del login se l'utente non fa il logout prima di chiudere il browser o aggiornare la pagina. \newline
Il gruppo approva all'unanimità il prodotto finora realizzato.
\subsection{Documento Specifica Architetturale}
Irene chiede delucidazione su alcuni dubbi e conferme di quanto scritto nel documento SA. \newline
Andrea fa vedere il diagramma \glossario{UML} della classi unico per server e client, essendo il server statless, mentre i diagrammi di sequenza sono distinti tra client e server.
Viene quindi lasciato il compito ad Irene di ultimare SA inserendo anche le immagini grafiche.
\subsection{Conclusione incremento 4}
Dato che il chatbot realizzato è funzionante e contiene tutte le funzionalità obbligatorie previste oltre alla facoltativa di apertura cancello, si decide di terminare l'incremento 4 chiedendo la revisione Product Baseline. \newline
Ultimi task da completare prima di richiedere la revisione PB:
\begin{itemize}
    \item deployment finale dell'applicazione;
    \item completamento dei documenti Specifica Architetturale e Manuale Utente assegnati ad Irene Benetazzo;
    \item verifica e approvazione di tutti i documenti;
    \item scrittura e invio lettera di presentazione per la revisione Product Baseline assegnato al responsabile Matteo Pillon;
\end{itemize} 
