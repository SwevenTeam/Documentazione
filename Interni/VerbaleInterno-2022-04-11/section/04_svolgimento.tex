\section{Svolgimento}
\subsection{Discussione sui documenti del progetto}
A seguito dell'aggiudicazione dell'appalto C1 il gruppo Sweven si è riunito per discutere sui passi successivi da compiere.
Durante l'incontro è risaltato lo scarso livello di informazione di tutti i componenti riguardo i documenti:
\begin{itemize}
    \item Piano di Progetto.
    \item Analisi dei Requisiti.
    \item Piano di Qualifica.
\end{itemize}
Ogni membro, di conseguenza, si è assunto il compito di trovare informazioni sui documenti per il progetto ed in particolare
sull'AdR e sul PdP, in quanto si pensa siano questi ultimi quelli che permetterebbero di comprendere e sviluppare i compiti successivi.

\subsection{Richiesta di una riunione con Imola Informatica}
Il gruppo, oltre a studiare i documenti, vede la necessità di un incontro con l'azienda Imola, poiché
permetterebbe di fissare le idee sugli obiettivi, le tecnologie richieste dal prodotto e anche per ricavare
informazioni utili all'\emph{Analisi dei Requisiti} e al \emph{Piano di Progetto}.

\subsection{Apertura dell'account GitHub ufficiale del gruppo}
Dopo la candidatura il gruppo ha deciso di creare l'account GitHub ufficiale in cui verranno caricati i lavori del team
fino alla fine del progetto. Tutto ciò che è stato creato prima dell'apertura dell'account è stato trasferito
in un repository chiamato \emph{Candidatura} (sempre collegato al profilo GitHub ufficiale del gruppo)
da cui è possibile visionare i documenti e la cronologia delle attività svolte in precedenza. 
Per le nuove attività, invece, verranno creati ulteriori repository.