\section{Svolgimento}
\subsection{Discussione interazione Client-Server}
	Inizia la riunione con la presentazione del lavoro fatto da Matteo e Pietro, dimostrando il corretto funzionamento dell'interazione fra Client e Server, il corretto funzionamento della conversazione e dell'inserimento vocale.
	Il bottone del login attualmente esegue solo il compito di inserire la parora chiave al chatbot. Pietro propone di disporre la funzionalità di salvataggio dell'APIkey; dopo una discussione il gruppo approva questa idea e decide di chiedere consiglio anche all'azienda Imola su come implementare ciò.
	
	\subsection{Piccole modifiche del codice}
	Mattia propone delle modifiche per migliorare la qualità del codice. Il primo cambiamento consiste nell'eliminare lo stato autenticazione e trasferire il compito di controllo identità per ogni singolo adapter; quindi ciascun adapter, in base alla sua neccessità, verifica se l'utente è loggato.
	Propone inoltre l'aggiunta di due controlli per la robustezza del codice: il primo è quello di controllare le parole con input simili e il secondo il refactory dei nomi delle classi per rendere uniforme il codice; il gruppo apprezza molto questa idea e decide di lavorarci nei futuri incrementi.
	
	\subsection{Riunione con Imola}
	Il gruppo è pronto a chiedere una riunione con l'azienda Imola per presentare il lavoro effettuato fino a questo punto; vengono inoltre segnalate due domande da chiedere all'azienda:
	\begin{itemize}
	\item Opinione sul funzionamento del bottone di login;
	\item API per la riunione esterna.
	\end{itemize}
	Il gruppo decide di chiedere la riunione all'azienda per la settimana successiva.


