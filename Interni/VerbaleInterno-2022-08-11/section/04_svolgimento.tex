\section{Svolgimento}
\subsection{Discussione interazione Client Server}
	Inizia la riunione con la discussione del lavoro fatto da Matteo e Pietro, il corretto funzionamento della iterazione fra Client e Server, il corretto funzionamento della conversazione e del inserimento vocale.
	Pietro chiede l'opinione del gruppo per il bottone del login, attualmente il bottone login fa solo il compito di inserire la parora chiave login al chatbot, chiede se è utile disporre la funzionalità del salvataggio dell'apikey, dopo vari discussioni il gruppo pensa che sia molto utile, viene chiesto l'opinione all'azienda Imola in seguito.
	
	\subsection{Piccole modifiche del codice}
	Mattia propone delle modifiche per migliorare la qualità del codice, il promo elemento è quello di eliminare lo stato autenticazione e di trasferrire il compito di cotrollare identità per ogni singolo adapter, quindi ogni singolo adapter in base alla sua neccessità verifica se utente è logato.
	Proppone inoltre due aggiunte di controlli per la robustazza del codice, il primo è quello di controllare le parole con input simili, e il secondo il refactory dei nomi delle classi, per rendere uniforme il codice, il gruppo apprezza molto queste idea e decide di lavorare dei futuri incrementi.
	
	\subsection{Riunione con Imola}
	Il gruppo sento proto a chiedere una riunione con l'azienda Imola per presentarli il lavoro effettuato fino a questo punto, innoltre vengono segnalate due domande per chiedere l'oppinione dell'azienda:
	\begin{itemize}
	\item Opinione sul funzionamento del bottone di login
	\item Api per la riunione esterna
	\end{itemize}
	Il gruppo decide di chiedere la riunione per la prossima settimana.


