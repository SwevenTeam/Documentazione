\section{Svolgimento}
\subsection{Discussione sui Compiti Eseguiti}
	Come prima interazione della riunione, viene ricordata la necessita di studio individuale dello \textbf{standard ISO/IEC 12207}. \\
	In seguito viene discusso il lavoro che i componenti del gruppo stanno attualmente eseguendo, ovvero la scrittura dei vari documenti per la fase di Revisione dei Requisiti. \\
	Da questa interazione tra i membri del gruppo, risulta che la stesura della documentazione procede in linea con quanto concordato. Sono tuttavia emersi dei dubbi, tra cui : 
	\begin{itemize}
		\item il capitolo 3.3 del documento \textbf{Piano di Progetto}, assegnato a \textit{Matteo Pillon}, viene temporaneamente rimandato a data da destinarsi, causa mancanza di alcune informazioni necessarie alla scrittura del suddetto capitolo;
		\item si decide di cambiare la struttura del documento \textbf{Piano di Progetto}, redatto da \textit{Irene Benetazzo}, da un insieme di elenchi puntati ad una struttura a predicati completi;
		\item viene affidato il compito di scegliere i codici identificativi per i Casi d'Uso e i Requisiti del capitolo 2 del documento \textbf{Norme di Progetto} a \textit{Tommaso Berlaffa}, utilizzati nel documento \textbf{Analisi dei Requisiti}.
	\end{itemize}
	Poiché la stesura di questi documenti è stata completata in minor tempo rispetto a quello che il gruppo si aspettava, si è deciso di accorciare questa fase da 2 settimane ad 1 settimana. \\
	Viene inoltre discusso l'implementazione dei diagrammi di Gantt, da modificare poiché la stesura ha richiesto minor tempo di quanto il gruppo si aspettava, e la modalità di Preventivazione degli Orari. Da questa discussione è emersa la necessità di creare un file \textbf{Google Sheet} tramite il quale aggiornare dinamicamente gli orari per il preventivo.
	
\subsection{Assegnazione dei Ruoli di Responsabile}
	Risolti questi dubbi, vengono assegnati a diversi membri del gruppo i ruoli di responsabile di alcuni documenti : 
	\begin{itemize}
		\item per il documento \textbf{Norme di Progetto}, viene assegnato \textit{}Pietro Macrì;
		
		\item per il documento \textbf{Piano di Progetto}, viene assegnato \textit{}Mattia Episcopo;
		
		\item per il documento \textbf{Piano di Qualifica}, viene assegnato \textit{}Qi Fan Andrea Pan. 
		
	\end{itemize}

\subsection{Assegnazione dei Capitoli e Riunioni Future}
	Assegnati i ruoli di responsabile, il gruppo inizia a discutere sul prossimo documento da completare, ovvero l'Analisi dei Requisiti. Si è deciso per questa divisione del documento :
	\begin{itemize}
		\item \textit{Irene Benetazzo} si occuperà dei Capitoli 1 e 2, ovvero rispettivamente \textbf{Introduzione} e \textbf{Descrizione del Prodotto};
		\item \textit{Matteo Pillon}, \textit{Mattia Episcopo} e \textit{Tommaso Berlaffa} si divideranno il Capitolo 3, contenente i \textbf{Casi d'Uso};
		\item \textit{Qi Fan Andrea Pan} si occuperà del Capitolo 4, ovvero \textbf{Requisiti};
		\item \textit{Pietro Macrì} si occuperà del Capitolo 5, ovvero \textbf{Tracciamento dei requisiti}.
	\end{itemize}
	Per la verifica del documento, vengono assegnati rispettivamente \textit{Irene Benetazzo} per la verifica del Capitolo 3 e \textit{Matteo Pillon} per la verifica dei restanti Capitoli. \\

