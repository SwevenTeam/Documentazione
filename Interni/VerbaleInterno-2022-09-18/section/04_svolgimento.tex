\section{Svolgimento}

\subsection{Lettura esito PB e confronto }
Il gruppo si ritrova per discutere dell'esito dei colloqui avvenuti rispettivamente con i professori Cardin e Vardanega riguardo il progetto Bot4Me. \\
La discussione è incentrata principalmente su ciò che non è stato considerato corretto e quindi dovrà essere migliorato. I punti principali sono stati: 
\begin{itemize}
  \item interpretazione dei verbali come strutture narrative e non come momento di autovalutazione e definizione di cosa fare successivamente;
  \item struttura interna dei processi nelle Norme di Progetto non conforme tra di loro;
  \item Piano di Progetto inefficace poiché visto solamente come un resconto contabile e scarso utilizzo dell'attualizzazione dei rischi;
  \item dissociazione tra specifiche di Test e metriche di qualità nel Piano di Qualifica; 
  \item scatto di versione ad ogni azione sul prodotto versionato.
\end{itemize}

\subsection{Correzione errori PB}
Il gruppo cerca quindi delle soluzioni ai vari problemi rilevati in sede di valutazione. Come soluzioni, è stato deciso rispettivamente di:
\begin{itemize}
  \item assicurarsi che i verbali contengano sempre le riflessioni sullo stato attuale del progetto e di ciò che è stato fatto nel periodo e sul cosa andare a fare in seguito. 
  Queste discussioni avvenivano anche precedentemente nel gruppo durante riunioni interne ma non sempre queste venivano registrate nei verbali;
  \item viene corretta la struttura delle Norme di Progetto, al fine di rispettare la struttura canonica dei processi;
  \item correzione e aggiornamento del Piano di Progetto.
  \item correzione e aggiornamento del Piano di Qualifica.
\end{itemize}
Infine, riguardo allo scatto di versione per ogni commit, il gruppo pensava di aver risolto questo problema introducendo il meccaniscmo di doppio issue <stesura> e <verifica>, tale per cui ogni
avanzamento di versione doveva essere verificato prima di essere tale. Questa però non è stata una soluzione considerata corretta, poiché era comunque possibile far avanzare un documento di versione anche se 
quest'ultimo non era ancora stato verificato. \\
I membri del gruppo presteranno quindi particolare attenzione a non commettere più questo tipo di errore, motivo per cui la versione di ogni documento verrà modificata solamente da chi effettuerà una verifica. 


\subsection{Suddivisione compiti per la prossima settimana}
Infine, dopo la lettura del documento, il gruppo decide di suddividere i compiti, tra correzione dei documenti e validazione finale, in previsione del collaudo con Imola Informatica e della revisione CA:
\begin{itemize}
  \item \textit{Matteo} e \textit{Pietro} si occuperanno degli ultimi Test riguardanti la parte Client \\ dell'applicativo;
  \item \textit{Tommaso}, \textit{Samuele} e \textit{Andrea} si occuperanno dell'aggiornamento e correzione del Piano di Qualifica e degli ultimi Test riguardanti la parte Server;
  \item \textit{Irene} e \textit{Mattia} si occuperà della correzione di alcuni documenti, tra cui le Norme di Progetto e il Piano di Progetto;
\end{itemize}
Inoltre, \textit{Samuele} aggiornerà l'Analisi dei Requisiti.
