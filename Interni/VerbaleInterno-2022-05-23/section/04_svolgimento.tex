\section{Svolgimento}
\subsection{Sviluppo del \glossario{PoC}}
La riunione inizia con Mattia e Andrea che presentano il lavoro svolto durante la settimana per lo sviluppo del \glossario{PoC} lato server e
mostrano di aver integrato in modi diversi l'uso del \glossario{access token} per le API di Imola. Mattia ha implementato
una classe specifica per la funzionalità di richiesta della sede mentre Andrea ha implementato una classe per gestire le API di Imola.
Dopo una piccola discussione si è deciso che Mattia e Andrea devono accordarsi per l'implementazione di quella parte. Successivamente
Tommaso e Pietro fanno notare come il programma sviluppato da Matteo la settimana scorsa sia già abbastanza per il lato client dell'applicazione,
di conseguenza non proseguiranno con la codifica di quest'ultimo. Il gruppo decide, inoltre, di implementare la funzionalità di presenza in sede
e di affidarne la codifica a Pietro e Matteo. Infine il gruppo si chiede se nel \glossario{PoC} l'applicazione possa essere utilizzata 
in locale o debba essere caricata su delle piattaforme esterne, in tal caso Mattia propone di utilizzare Heroku.

\subsection{\glossario{GitHub Actions} per controllare lo stile del codice}
Tommaso propone di utilizzare le \glossario{GitHub Actions} per controllare lo stile del codice, in quanto permettono non solo
di trovare ma anche di correggere eventuali violazioni nello stile di codifica. Il gruppo approva la proposta e Tommaso
si impegna a implementarle.

\subsection{Richiesta di un incontro con Imola Informatica}
Il gruppo decide di richiedere un incontro con l'azienda per mostrare il lavoro svolto, capire se si sta procedendo
nella direzione giusta e per chiarire dubbi sul \glossario{PoC}.

\subsection{Discussione sui documenti}
Irene Benetazzo e Samuele Rizzato decidono di diversi gli ultimi aggiornamenti e controlli dei documenti e
di verificare a vicenda le modifiche attuate.