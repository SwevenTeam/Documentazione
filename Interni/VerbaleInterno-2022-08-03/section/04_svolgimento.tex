\section{Svolgimento}
\subsection{Discussione sviluppo lato Server}
\textit{Tommaso} espone ai membri presenti del team la soluzione che ha adottato per realizzare il requisito di consultivazione delle ore. In linea con il modello delineato da \textit{Andrea} nelle settimane precedenti, \textit{Tommaso} ha concentrato lo sviluppo sulla gestione degli \glossario{Stati}, in particolare essendo un requisito particolarmente complesso che richiede di ricevere svariate informazioni da parte dell'utente per poter registrare l'attività svolta, sono stati creati una serie di sottostati a partire da quello principale. \newline
Il team ha apprezzato molto questo tipo di soluzione, in quanto una gestione di questo tipo, permette all'utente di poter modificare singolarmente le voci inserite durante il processo di registrazione dell'attività svolta. Allo stesso tempo il team si riserva nelle settimane successive la possibilità di utilizzare un approccio alternativo che permetta una gestione più semplice ed efficiente di requisiti complessi, che richiedono molti \glossario{Stati} come quello realizzato da \textit{Tommaso}.

\subsection{Chiusura incremento}
In seguito alla presentazione fatta da \textit{Tommaso}, chiariti alcuni dubbi tra i presenti, il team decide di procedere con la chiusura dell'incremento precedente e la conseguente apertura di quello successivo, il cui responsabile sarà: \textit{Pietro Macrì}. \newline
I partecipanti alla riunione decidono quindi di procedere con la suddivisione dei task per il periodo successivo, in particolare vengono così suddivisi:
  \begin{itemize}
    \item \textit{Tommaso} si occuperà di completare i controlli e i test della funzionalità svolta nell'incremento precedente.
    \item \textit{Matteo} e \textit{Pietro} avendo quasi terminato il lato Client dell'applicativo si incaricano di realizzare il requisito obbligatorio di autenticazione mediante \glossario{API-KEY} e di effettuare l'integrazione fra lato Client e lato Server, in prospettiva di una riunione futura con Imola.
    \item \textit{Mattia} e \textit{Samuele} si divono lo sviluppo dei requisiti di apertura del cancello aziendale e creazione di riunioni su una piattaforma esterna. 
    \item \textit{Irene} si occuperà dell'inizio della stesura dei documenti di \glossario{Specifica Architetturale} e \glossario{Manuale Utente}
    \item \textit{Andrea}, dato il buon lavoro svolto nella realizzazione della specifica architetturale, si occuperà dello studio per la ricerca di un'eventuale  soluzione più efficiente per la gestione degli \glossario{Stati}.
  \end{itemize}
\newpage
