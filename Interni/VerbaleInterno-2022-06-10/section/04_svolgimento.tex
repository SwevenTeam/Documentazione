\section{Svolgimento}
\subsection{Modificare il documento \emph{AnalisiDeiRequisiti}}
Questa mattina, per mail, è arrivato il responso del professore Cardin Riccardo indicava dei punti da sistemare nel documento \emph{AnalisiDeiRequisiti}.
Il gruppo si confronta e decide di:
    \begin{itemize}
        \item inserire la descrizione testuale dei UC1.1.1 e UC1.1.2 riportandola anche negli altri;
        \item aggiungere un requisito desiderabile sull'uso della geolocalizzazione nel UC5;
        \item aggiungere l'attore secondario nel UC6;
        \item modificare UC8.3 e UC8.4 chiedendo all'azienda Imola maggiori informazioni sulle priorità e status dei \glossario{ticket};
        \item aggiungere i diagrammi nei casi d'uso mancanti;
        \item modificare la descrizione del RO-V-1 aggiungendo la versione minima del sistema operativo che l'applicazione dovrà supportare;
        \item spostare RO-V-2 nella sezione dei requisiti funzionali.
    \end{itemize}
Queste modifiche vengono assegnate ai membri che hanno redatto le relative sezioni cioè Tommaso Berlaffa, Mattia Episcopo, Matteo Pillon, Qi Fan Andrea Pan. 
\subsection{Altre piccole correzioni per tutti i documenti}
Inoltre il professore fa notare che mancano le didascalie alle immagini e nel nome del documento deve comparire anche la versione, queste integrazioni verranno riportate in tutti i documenti da Irene Benetazzo e Samuele Rizzato.
Si è deciso che la versione verrà aggiunta con un trattino alla fine del nome, quindi diventerà NomeDocumento-v1.0.0.
Per le ore di lavoro da rendicontare di questa fase extra si è deciso che peseranno in negativo sul consuntivo, avendo già ripianificato tutte le ore precedentemente avanzate.