\section{Svolgimento}
\subsection{Discussione temi trattati con Imola Informatica}
Al termine dell'incontro con Imola Informatica, il team si è riunito per discutere dei temi che si sono trattati nella riunione. Visto che gli argomenti toccati sono stati molti, tra i quali API Rest, Python e i vari servizi sul quale appoggiarsi, il gruppo decide che serve del tempo per approfondire tutti questi argomenti e per dare un feedback all'azienda. Per questo il team si prenderà qualche giorno nel quale ogni componente approfondirà i temi trattati nella riunione con Imola Informatica, per questa fase di studio individuale e approfondimenti vari si è dato come termine il 2022-04-19.
\subsection{Discussione sul metodo di lavoro del gruppo}
Il gruppo si è confrontato sulla metodologia di lavoro da utilizzare per questo lavoro. Dopo aver discusso sulle varie opzioni, il gruppo ha ritenuto opportuno utilizzare da qui in avanti il framework SCRUM per la gestione del progetto. La scelta è ricaduta su questa metodologia di lavoro principalmente per due ragioni, alcuni l'avevano già provata e ritengono sia ottima per questo progetto, altri non l'hanno mai utilizzata, ma essendo una di quelle ritenute migliori, sono volenterosi di provarla.

\subsection{Pianificazione della Macrofase RTB}
Al termine dello studio singolo per i vari approfondimenti il gruppo ha stabilito che entrerà nella prima macrofase del progetto che terminerà con la revisione RTB a fine Maggio. 
La macrofase verrà suddivisa in tre fasi e la prima fase va dal 2022-04-19 al 2022-05-02.
La riunione era particolarmente incentrata all'organizzazione di questa prima fase, per tempistiche, documenti da sviluppare e ruoli da tenere per i componenti del gruppo. Dopo aver valutato tutte le possibili alternative, il gruppo è giunto alla seguente organizzazione. \newline
In questa fase si inizierà la stesura dei seguenti moduli del documento \textbf{Piano di Progetto}:
\begin{itemize}
	\item Introduzione
	\item Analisi dei Rischi
	\item Modello di Sviluppo
	\item Pianificazione Macrofase RTB
	\item Preventivo fase Baseline documentale
\end{itemize}
Nello stesso tempo si continuerà la stesura del documento \textbf{Norme di Progetto}, per seguenti sezioni:
\begin{itemize}
	\item Processi Primari
	\item Processi di Supporto
	\item Processi Organizzativi
\end{itemize}
Inoltre si inizierà la stesura del documento \textbf{Piano di Qualifica}:
\begin{itemize}
	\item Introduzione
	\item Qualità di Processo
	\item Qualità di Prodotto
\end{itemize}

\subsection{Organizzazione della fase Baseline documentale}
Il gruppo a questo punto ha discusso dei vari ruoli da ricoprire in questa fase, ponendo attenzione a suddividere equamente il lavoro in base alle esigenze e alle volontà di tutti i componenti.
Il lavoro sui documenti è stato diviso come segue:
\begin{center}
\renewcommand{\arraystretch}{1.8}
\begin{tabular}{ |c|c|c| }
\hline
\textbf{Documento} & \textbf{Sezione} & \textbf{Componenti}\\
\hline
Norme Di Progetto & Processi Primari & Tommaso Berlaffa e Matteo Pillon\\
\hline
Norme Di Progetto & Processi di Supporto & Pietro Macrì e Qi Fan Andrea Pan\\
\hline
Norme Di Progetto & Processi Organizzativi & Samuele Rizzato e Mattia Episcopo\\
\hline
Piano Di Progetto & Introduzione & Matteo Pillon\\
\hline
Piano Di Progetto & Analisi dei Rischi & Qi Fan Andrea Pan\\
\hline
Piano Di Progetto & Modello di Sviluppo & Matteo Pillon\\
\hline
Piano Di Progetto & Pianificazione Macrofase RTB & Irene Benetazzo\\
\hline
Piano Di Progetto & Preventivo fase Baseline documentale & Irene Benetazzo\\
\hline
Piano Di Qualifica & Introduzione & Pietro Macrì\\
\hline
Piano Di Qualifica & Qualità di Processo & Samuele Rizzato\\
\hline
Piano Di Qualifica & Qualità di Prodotto & Irene Benetazzo\\
\hline
\end{tabular}
\end{center}
Verso la fine Pietro Macrì e Tommaso Berlaffa verificheranno il materiale prodotto in questa prima fase. \newline
Ne consegue che i ruoli che i componenti del gruppo svolgeranno in questa fase di baseline documentale, saranno i seguenti:
\begin{center}
\renewcommand{\arraystretch}{1.8}
\begin{tabular}{ |c|c|c| }
\hline
\textbf{Componente} & \textbf{Ruoli}\\
\hline
Irene Benetazzo & Responsabile e Analista\\
\hline
Tommaso Berlaffa & Amministratore e Verificatore \\
\hline
Mattia Episcopo & Amministratore\\
\hline
Pietro Macrì & Amministratore e Verificatore\\
\hline
Qi Fan Andrea Pan & Amministratore e Analista\\
\hline
Matteo Pillon & Amministratore \\
\hline
Samuele Rizzato & Amministratore e Analista\\
\hline
\end{tabular}
\end{center}
\newpage
