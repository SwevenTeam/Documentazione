\section{Svolgimento}
\subsection{Discussione circa lo sviluppo della versione web}
In questa fase, il gruppo ha valutato i pro e i contro dell'eventuale sviluppo di una versione web del chatbot. \newline
Si è giunti alla conclusione che il tempo necessario per creare il client per il bot è un buon investimento, in quanto permette di visualizzare con più facilità il comportamento di Bot4Me.
Dunque il gruppo ha deciso di continuare con lo sviluppo del lato client, che rimane affidato agli studenti Tommaso Berlaffa e Pietro Macrì.

\subsection{Divisione ruoli per compiti mancanti}
Il gruppo ha analizzato con maggiore attenzione le suddivisioni interne ai sottogruppi creati precedentemente. \newline
Dopo un'attenta discussione, il gruppo conclude che ogni sottogruppo si organizza per conto proprio per lo svolgimento del proprio compito.

\subsection{Decisioni circa il preventivo dello sviluppo}
Il gruppo ha analizzato quale processo di sviluppo fosse l'ideale per il raggiungimento dell'obiettivo. \newline
Al termine del ragionamento, si è scelto di optare per una divisione del lavoro in due macrocomponenti -
funzioni obbligatorie e funzioni facoltative - senza tuttavia scendere nel dettaglio di come tali macrocomponenti
siano suddivisi a loro volta. Tale decisione viene lasciata ad un'analisi più approfondita da parte di Irene
Benetazzo, la quale si è offerta volontariamente per questo compito.

\newpage