\section{Discussione valutazione PoC}
Sono stati analizzati i punti di criticità individuati dal Professor Vardanega durante una lettura dettagliata della 
valutazione da lui inviata al gruppo. Sono state assegnate ad alcuni membri del gruppo le modifiche da effettuare per 
migliorare i documenti. Inoltre è stata individuata la funzionalità ``Kanban'' di GitHub per gestire ottimamente le 
coppie $<$azione, verifica$>$ richieste senza incappare nella necessità di creare due differenti issues.

\section{Organizzazione prossimi compiti}
Sono stati analizzati i prossimi compiti necessari per lo sviluppo del prodotto durante la fase Product Baseline in cui 
il gruppo è appena entrato. Il primo incremento della prima sottofase (Requisiti Obbligatori) viene nuovamente stimato in 
termini di tempo in due settimane; la fine di tale incremento è quindi posta al giorno 13 luglio. 
Sono stati valutati i primi passi da compiere nella creazione dei due nuovi documenti necessari, ovvero Manuale Utente e 
Specifica Architetturale. Il primo è stato deciso di analizzarlo dopo un incontro prossimo, ancora da fissare, con 
l'azienda Imola Informatica. Per quanto riguarda il secondo, invece, è stata assegnata la creazione di una bozza ai 
membri del gruppo non partecipanti al miglioramento dei documenti prima accennato; tali membri dovranno anche studiare 
la piattaforma Heroku per essere in grado in futuro di effettuare al meglio il deployment dell'app. A queste ultime 
operazioni è stata posta la scadenza nel giorno 4 luglio, al termine del quale verrà fissato l'incontro con l'azienda.

\newpage