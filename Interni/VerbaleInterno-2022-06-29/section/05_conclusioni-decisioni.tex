\section{Conclusioni}
Il gruppo è riuscito a organizzarsi nell'individuazione dei prossimi obiettivi da raggiungere.
\newline
``Kanban'' è risultato un elemento potenzialmente comodo per la gestione delle microverifiche.
\newline
Sono risultati necessari uno studio più approfondito di Heroku e un miglioramento dei documenti creati. Il gruppo decide 
anche di iniziare a mettere le basi per lo sviluppo della Specifica Architetturale.
\newline
La prossima riunione interna viene fissata al giorno 4 luglio; la prossima riunione esterna, invece, viene proposta, ma non 
fissata in una data precisa. L'azienda è stata prontamente avvisata dopo la riunione dell'esito della prima fase e della 
mancanza di data per l'incontro, che comunque avverrà, salvo imprevisti, la settimana successiva.
\newpage

\section*{Tracciamento delle decisioni}
	\renewcommand{\arraystretch}{1.8} %aumento ampiezza righe
	\begin{tabular}{ |c|l| }
		\hline
		\textbf{Codice} & \textbf{Descrizione} \\
		\hline
		VI\_2022-06-29.1 & Assegnato ``Kanban'' per la gestione delle microverifiche.\\ %VI o VE in base al verbale interno o esterno
		\hline
		VI\_2022-06-13.2 & Ad ogni membro del gruppo assegnato uno di questi tre compiti: studio \\ & di Heroku, 
		creazione bozza Specifica Architetturale, miglioramento \\ & documento esistente.\\ %VI o VE in base al verbale interno o esterno
		\hline
		VI\_2022-06-13.3 & Riunione interna fissata al 2022/07/04, riunione esterna ancora da \\ & fissare. 
		Azienda informata della situazione in toto.\\ %VI o VE in base al verbale interno o esterno
		\hline
	\end{tabular}