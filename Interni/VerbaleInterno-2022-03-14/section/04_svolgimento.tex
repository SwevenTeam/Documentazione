\section{Svolgimento}
	\subsection{Decisioni riguardo l'identità del gruppo}
	Dopo una breve discussione dove i membri del gruppo hanno espresso idee e preferenze è stato scelto come nome del gruppo Sweven Team, come proposto da Mattia Episcopo. In seguito si sono valutate varie proposte per il logo del gruppo, e dopo averle valutate con attenzione si è scelto quello proposto da Matteo Pillon. Come richiesto dal regolamento del progetto didattico, il gruppo ha poi attivato la propria email: swe7.team@gmail.com che verrà usata per tutte le comunicazioni. Infine il gruppo ha scelto di tenere il giorno lunedì e l'ora 18:00 come incontro con cadenza settimanale per le prossime riunioni.
	\subsection{Discussione e scelta del capitolato}
	Sono stati valutati insieme i capitolati a disposizione. Dopo una discussione riguardo la fattibilità dei capitolati, il gruppo si è orientato verso la scelta del capitolato C1 e, proprio per approfondire questa scelta, richiederà un incontro con l'azienda proponente. Si è quindi tenuta una breve sessione in cui i membri del gruppo hanno raggruppato idee e domande per l'incontro con l'azienda proponente.
	\subsection{Organizzazione di comunicazione e condivisione materiale all'interno del gruppo}
	Il gruppo ha deciso all'unanimità di utilizzare GitHub per la condivisione e il versionamento dei documenti che riguardano il progetto e l'organizzazione interna; quindi è stata creata la repository SWEven con due branch: main, in cui verrà caricato il materiale approvato, e developers, per un uso interno del gruppo; ciò per agevolare lo scambio e la modifica, mantenendo il versionamento. Inoltre per la stesura dei documenti del gruppo si è deciso di utilizzare \LaTeX .
	\subsection{Informazioni da richiedere sul capitolato scelto}
	Chiarimento sui vari protocolli da utilizzare per le funzionalità richieste dall'applicativo.
	Linguaggi di programmazione da utilizzare per lo sviluppo dell'applicativo, in particolare per la realizzazione della parte mobile.


