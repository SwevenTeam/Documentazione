\section{Svolgimento}
\subsection{Discussione UML Server}
Dopo una breve introduzione alla riunione, 
i membri del gruppo si confrontano sul tema principale della riunione, ovvero la discussione del diagramma UML della parte server.
Il diagramma, creato da alcuni membri del gruppo, viene esposto a tutti da \textit{Andrea}, il quale spiega nel dettaglio la progettazione e il funzionamento della parte server dell'applicativo. 
In seguito, il diagramma viene discusso e capito da tutti i componenti del gruppo, apportando alcune leggere variazioni. 
Questa nuova versione viene poi approvata all'unanimità da tutto il gruppo.
\subsection{Organizzazione prossimi compiti}
In primis viene deciso di creare una nuova repository che contenga solo il codice, sia della parte Client che della parte Server. 
Dopo di ché, vengono suddivisi i compiti per questa fase tra i membri del gruppo come segue :
\begin{itemize}
  \item \textit{Andrea} e \textit{Mattia} lavorano alla parte Server per fornire una prima versione funzionanante di quello che è stato progettato.
  \item \textit{Irene}, \textit{Tommaso} e \textit{Samuele} proseguono da questo punto in poi implementando le funzionalità di registrazione presenza in sede e consuntivazione.
  \item \textit{Pietro} e \textit{Matteo} continuano lo studio e la prima implementazione della parte Client.
\end{itemize}   


