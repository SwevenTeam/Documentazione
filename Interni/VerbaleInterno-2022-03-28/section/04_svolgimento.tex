\section{Svolgimento}
\subsection{Discussione Template Aggiornato}
 Come primo argomento delle riunione viene proposta una nuova versione di template, creata da \textbf{Irene Benetazzo} da utilizzare per la stesura della documentazione.
 \\ Questa è una versione aggiornata della precedente, divisa in due versioni: una ideata per i verbali e l'altra pensata per documentazione generica da adattare successivamente.
 \\La nuova versione contiene tutti i pacchetti utili e inoltre supporta una divisione delle sezioni del documento in più file, aumentando così la flessibilità e dando la possibilità di stesura di diversi capitoli di un singolo documento in contemporanea.
	
	
\subsection{Valutazione Compiti Svolti Precedentemente}
 Viene fatta dal gruppo una valutazione generale del lavoro svolto nella settimana precedente, ovvero lo studio di fattibilità dei capitolati presentati e la stesura delle motivazioni alla base della scelta del capitolato.
 \\ Il gruppo si trova soddisfatto di come sono stati svolti questi lavori. Viene inoltre aggiunta una nuova motivazione inerente alla scelta del capitolato.
 \\ Vengono quindi assegnati i ruoli di verificatore ed approvatore per il documento di Studio di Fattibilità, rispettivamente a \textbf{Irene Benetazzo} come approvatore e \textbf{Qi Fan Andrea Pan} come responsabile.


\subsection{Inizio Lavoro Candidatura}
 Il gruppo inizia il lavoro per la stesura dei documenti necessari alla candidatura. 
 \\In primo luogo, viene discusso il calcolo delle ore per ruolo in base al costo, tenendo conto dei dati forniti precedentemente.
 \\In seguito,viene scelta una data per la consegna finale del progetto. La data scelta dal gruppo è il 30 Settembre.

\subsection{Divisione Compiti}
Come ultimo argomento della riunione, vengono assegnati i ruoli da svolgere in settimana ai vari componenti del gruppo:
 \begin{itemize}
 	\item stesura del glossario, contenente tutte le terminologie necessarie con la propria descrizione. Questo compito è stato affidato a \textbf{Mattia Episcopo} 
 	\item norme di Progetto su processi Organizzativi, affidato a \textbf{Samuele Rizzato} 
 	\item creazione PDF per la candidatura del gruppo all'appalto, scritto da \textbf{Irene Benetazzo}. Su questo PDF si baseranno le slide di presentazione della candidatura del gruppo, create da \textbf{Matteo Pillon} 
 	\item riordinamento dei vari file sulla piattaforma github, per avere una gestione migliore e ottimizzata. Compito preso in carico da \textbf{Pietro Macrì}
 \end{itemize}
In particolare, il glossario creato sarà una versione iniziale che verrà aggiornata costantemente nel corso del progetto.



