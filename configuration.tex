\newcommand{\docNome}{Analisi dei Requisiti}                          % NOME DEL DOCUMENTO
\newcommand{\docVersione}{1.0.0}                 % INSERIRE VERSIONE IN FORMATO x.y.z
\newcommand{\docStatus}{Approvato}          % AGGIORNARE SOLO QUANDO APPROVATO
\newcommand{\docUso}{Esterno}                           % INTERNO O ESTERNO
\newcommand{\docDestinatari}{
      Gruppo Sweven Team \\ %aggiungere altri con & Nome\\
      & Prof. Tullio Vardanega \\
      & Prof. Riccardo Cardin \\
      & Azienda Imola Informatica\\
} 
\newcommand{\docNomeTeam}{Sweven Team}
\newcommand{\docRedattori}{
      Irene Benetazzo \\ %aggiungere altri con & Nome\\
      & Matteo Pillon \\
      & Mattia Episcopo \\
      & Tommaso Berlaffa \\
      & Pietro Macrì\\
      & Pan Qi Fan Andrea\\
}
\newcommand{\docVerificatori}{
      Irene Benetazzo \\
      & Matteo Pillon 
}
\newcommand{\docApprovazione}{Samuele Rizzato}
\newcommand{\glossario}[1]{\textit{#1}\textsubscript{\textit{G}}}
\documentclass[12pt, a4paper,table]{article}
\usepackage[utf8]{inputenc}
\usepackage{lastpage}
\usepackage{hyperref}
\usepackage{fancyhdr}
\usepackage{fancyvrb}
\usepackage{geometry}
\usepackage{xcolor}
\usepackage{array}
\usepackage{graphicx}
\usepackage{float}
\usepackage{charter}
\usepackage{eurosym}
\usepackage{pdflscape}
\hypersetup{
pdfborder = {0 0 0}
}
\geometry{a4paper,top=3cm,bottom=3cm,left=2cm,right=2cm}
\title{\textsc{\docNome}}
\author{}
\date{}
\definecolor{footer-gray}{HTML}{808080}
\pagestyle{fancy}
\fancyhf{}
\rhead{\textcolor{footer-gray}{\docNome} }
\lhead{\textcolor{footer-gray}{Sweven Team}}
\fancyfoot{}
\cfoot{\textcolor{footer-gray}{Pagina \thepage  \hspace{1pt} di \pageref*{LastPage}} }
\setcounter{tocdepth}{5}	%aggiunge paragrafi e sottoparagrafi all'indice
\setcounter{secnumdepth}{5}	%aggiunge numero indicizazzione a paragrafi e sottoparagrafi
\renewcommand*\contentsname{Indice}

\begin{document}
\maketitle
	\vspace{-2em}
	\begin{center}
	\includegraphics[scale=0.50]{images/logo.jpg} \\
	\vspace{2em}
	\huge \textsc{\docNomeTeam}\\
	\normalsize \href{mailto:swe7.team@gmail.com}{swe7.team@gmail.com}\\
	\vspace{1em}
	\begin{tabular}{r|l}
		\multicolumn{2}{c}{ \textsc{Informazioni sul documento} } \\
		\hline
		\textbf{Versione}     & \docVersione\\
		\textbf{Uso}          & \docUso\\
        \textbf{Destinatari}  & \docDestinatari\\
		\textbf{Stato}        & \docStatus\\
		\textbf{Redattori}    & \docRedattori\\
		\textbf{Verificatori} & \docVerificatori\\
		\textbf{Approvatori} & \docApprovazione\\
	\end{tabular}
	\end{center}
    \begin{center}
        \LARGE{\textbf{Sintesi}} 
    \end{center}
    \normalsize{Glossario dei termini utilizzati nei documenti.}
	\thispagestyle{empty}   
	\newpage
\section*{Diario delle modifiche}
	\begin{center}
	\renewcommand{\arraystretch}{1.8} %aumento ampiezza righe
	\begin{longtable}{ |c|c|p{8em}|c|m{5em}|m{6em}| }
	\hline
	\textbf{Versione} & \textbf{Data} & \textbf{Descrizione} &  \textbf{Ruolo} &  \textbf{Autore} & \textbf{Verificatore}\\ %Aggiungere le nuove righe sopra la prima
	\hline % Se il nome non ci sta, metterlo a mano con aggiunta di \newline (esempio: Nome \newline Cognome)
	0.1.4 & 2022-05-24 & Aggiornamento \$2.2.6 & Amministratore & Tommaso \newline Berlaffa & \\
  \hline
  0.1.3 & 2022-05-13 & Aggiunto \$4.1.6, aggiornato \$2.2.4 & Amministratore & Samuele \newline Rizzato & Irene Benetazzo\\
  \hline
	0.1.2 & 2022-05-07 & Aggiornamento \$2 & Amministratore & Tommaso \newline Berlaffa & Matteo \newline Pillon\\
  \hline
	0.1.1 & 2022-05-02 & Aggiornamento \$1 & Amministratore & Irene \newline Benetazzo & Matteo \newline Pillon\\
	\hline
	0.1.0 & 2022-04-28 & Verifica dell'intero Documento & Verificatore & Tommaso Berlaffa, \newline Pietro Macrì & \\
	\hline
	0.0.9 & 2022-04-28 & Scrittura \newline \$3.5, \$3.6 & Amministratore & Pan Qi Fan \newline Andrea & Tommaso \newline Berlaffa\\
	\hline
	0.0.8 & 2022-04-25 & Scrittura \newline \$3.2, \$3.3, \$3.4 & Amministratore& Pietro \newline Macrì & Tommaso \newline Berlaffa\\
	\hline
	0.0.7 & 2022-04-24 & Scrittura \newline \$2.1, \$2.2 & Amministratore & Tommaso \newline Berlaffa & Pietro \newline Macrì\\
	\hline
	0.0.6 & 2022-04-22 & Modifica \newline \$4.3, \$4.4 & Amministratore & Mattia \newline Episcopo & Pietro \newline Macrì\\
	\hline
	0.0.5 & 2022-04-21 & Continuazione \$4.1, scrittura \$4.2 e \$4.3 & Amministratore & Samuele \newline Rizzato & Pietro \newline Macrì\\
	\hline
	0.0.4 & 2022-04-03 & Scrittura \$4.1 & Amministratore & Samuele \newline Rizzato & Pietro \newline Macrì\\
	\hline
	0.0.3 & 2022-04-02 & Continuazione \$3.1 & Amministratore & Irene \newline Benetazzo & Tommaso \newline Berlaffa \\
	\hline
	0.0.2 & 2022-03-27 & Scrittura \$3.1 & Amministratore & Irene \newline Benetazzo & Tommaso \newline Berlaffa \\
	\hline
	0.0.1 & 2022-03-26 & Scrittura \$1.1 & Amministratore & Irene \newline Benetazzo & Pietro \newline Macrì \\
	\hline
    & 2022-03-26 & Creazione documento & Amministratore & Irene \newline Benetazzo & \\
	\hline
	\end{longtable}
	\end{center}
	\newpage
\tableofcontents
\newpage
\section{Introduzione}
\subsection{Scopo del Documento}
Il seguente documento è necessario per organizzare la suddivisione dei lavori all'interno del gruppo e la conseguente realizzazione del progetto. Per ogni attività verranno dunque definiti i seguenti attributi: 
\begin{itemize}
    \item rischi connessi allo svolgimento dell'attività;
    \item attribuzione di un ruolo ad ogni membro del team per consentirne lo svolgimento;
    \item preventivo delle risorse necessarie per portare a termina l'attività;
    \item tempo e risorse effettivamente impiegate per la realizzazione dell'attività;
    \item analisi generale dell'attività svolta.
\end{itemize}
La definizione di tali attributi permette di organizzare il lavoro in maniera efficiente, in modo tale da consentire al gruppo di lavorare in parallelo. 

\subsection{Scopo del Capitolato}
Lo scopo di tale progetto è quello di sviluppare un Chatbot che si interfacci con software aziendali, semplificando i compiti che i dipendenti devono svolgere. 
In particolare vengono individuate le seguenti operazioni: 
\begin{itemize}
    \item tracciamento della presenza in sede (\textbf{EMT}\textsubscript{G});
    \item rendiconto attività svolte quotidianamente (\textbf{EMT}\textsubscript{G});
    \item apertura del cancello aziendale (\textbf{MQTT}\textsubscript{G});
    \item creazione di una riunione in un servizio esterno;
    \item servizio di ricerca documentale (\textbf{CMIS}\textsubscript{G});
    \item creazione e tracciamento di bug (\textbf{Redmine}\textsubscript{G}).
\end{itemize}

\subsection{Glossario}
Per assicurare la massima fruibilità e leggibilità del documento, il team SWEven ha deciso di creare un documento denominato \textit{Glossario} il cui scopo sarà quello di contenere le definizioni dei termini ambigui o particolarmente specifici del progetto. Sarà possibile riconoscere i termini presenti al suo interno in quanto terminanti con la lettera \textit{G} posta come pedice della parola stessa. 
\subsection{Riferimenti}

\subsubsection{Normativi}
\begin{itemize}
    \item Norme di progetto
\end{itemize}

\subsubsection{Informativi}
\begin{itemize}
    \item Analisi dei requisiti;
    \item  \href{https://www.math.unipd.it/~tullio/IS-1/2021/Progetto/C1.pdf}{\color{blue} Capitolato di appalto C1 - BOT4ME};
    \item Software Engineering - Ian Sommerville: 10th Edition
    \begin{itemize}
        \item Capitolo 22 - Project Management;
        \item Capitolo 23 - Project Planning;
    \end{itemize}
    \item \href{https://www.math.unipd.it/~tullio/IS-1/2021/Dispense/T05.pdf}{\color{blue} Slide del corso - Ciclo di vita del software};
    \item \href{https://www.math.unipd.it/~tullio/IS-1/2021/Dispense/T06.pdf}{\color{blue} Slide del corso - Gestione di progetto};
\end{itemize}

\subsection{Programma Revisioni}
Durante lo svolgimento del progetto sono previste 3 diverse revisioni, il cui scopo è quello di verificare il corretto avanzamento del lavoro e la validità di quanto prodotto fino ad allora. 
Il gruppo prevede di effettuare tali revisioni secondo lo schema di seguito riportato: 
\begin{itemize}
    \item \textbf{Requirement and Tecnology Baseline}: settimana dal 2022-05-30 al 2022-06-03
    \item \textbf{Product Baseline}: settimana dal 2022-08-01 al 2022-08-05
    \item \textbf{Customer Acceptance}: in corso di definizione.
\end{itemize}
\newpage

\newpage
\section{Descrizione del prodotto}
\subsection{Obbiettivi del prodotto}
L'obbiettivo del \glossario{chatbot} è quello di aiutare i dipendenti dell'azienda. Essi potranno fornire dei comandi testuali o vocali in input al  \glossario{chatbot}, guidandoli e semplificando l'esecuzione di alcuni compiti, come: aprire il cancello, tracciare la presenza, 
consuntivare le ore di lavoro, ricercare documenti, programmare una riunione, segnalare e 
tracciare i bug. \newline
Il \glossario{chatbot} potrà essere utilizzato solo dai dipendenti dell'azienda \textit{Imola Informatica}, cioè utenti che dispongono una
mail con dominio \textbf{@imolainformatica.it}

\subsection{Struttura}
Il prodotto avrà come componenti principali:
\begin{itemize}
    \item \textbf{Interfaccia autenticazione:} l'autenticazione avverrà mediante un server esterno che 
                genererà un \glossario{token} di accesso
    \item \textbf{Interfaccia chatbot:} interfaccia messaggistica dell'applicazione in cui l'utente dialoga 
                con il bot tramite input testuale o vocale, resterà traccia del flusso dei messaggi.
    \item \textbf{Web server:} l'intermediario tra le richieste dell'utente e i servizi aziendali, interpreta 
                i messaggi scritti dall'utente e se possibile esegue subito l'azione richiesta altrimenti 
                richiede all'utente altre informazioni più specifiche.
\end{itemize}

\subsection{Vincoli}
Per poter utilizzare il \glossario{chatbot} è necessario un dispositivo (smartphone, tablet o computer) che 
abbia tastiera o microfono e una connessione internet attiva.

\subsection{Attori}
A seguito dell'analisi del capitolato e dagli incontri con Imola Informatica, nel sistema sono 
presenti solo attori primari:
\begin{itemize}
    \item \textbf{Utente non autorizzato:} utente che non è ancora stato identificato come dipendente aziendale e quindi non ha accesso ai servizi offerti dal \glossario{chatbot}, ma solamente al sistema di autenticazione.
    \item \textbf{Utente autorizzato:} l'utente è stato riconosciuto come dipendente aziendale, quindi 
                ha accesso a tutte le funzionalità del \glossario{chatbot}. \newline
                L'utente autorizzato nel \glossario{chatbot} è una generalizzazione di:
                \begin{itemize}
                \item \textbf{utente autorizzato e ha eseguito l'accesso alla piattaforma di riunione}
                \item \textbf{utente autorizzato e non ha eseguito l'accesso alla piattaforma di riunione}
                \end{itemize}
                cioè se è stato effettuato il login anche nella piattaforma (Zoom, Meet, Teams) in 
                cui si programma la riunione.
\end{itemize}

\begin{figure}[h]
    \centering
    \includegraphics[scale=1]{images/Attori.png} 
    \caption{Rappresentazione grafica degli attori coinvolti all'interno dei casi d'uso}
\end{figure}
\newpage
\newpage
\section{Casi d'uso}
\subsection{UC1 - Gestione autenticazione}
\begin{itemize}
    \item \textbf{Identificativo}: UC1
    \item \textbf{Nome}: Gestione autenticazione
    \item \textbf{Descrizione grafica}:
\end{itemize}

\begin{center}
    \includegraphics[scale=0.50]{images/UC1.png} 
\end{center}

 \begin{itemize}
    \item \textbf{Attori}
 \begin{itemize} 
    \item \textit{Primari}: utente non autorizzato
    \item \textit{Secondari}: non presenti
 \end{itemize}
 \item \textbf{Precondizione}: l'utente non dispone di un token\textsubscript{G} di accesso per poter effettuare l'autenticazione con il chatbot.
 \item \textbf{Postcondizione}:il chatbot risponde alla richiesta dell'utente fornendo un link attraverso il quale l'utente può ottenere il token\textsubscript{G} di accesso ai servizi messi a disposizione dal chatbot.
 \item \textbf{Scenario principale}: l'utente deve effettuare il login al sistema, fa una richiesta scritta (\textbf{UC1.1.1}) oppure vocale (\textbf{UC1.1.2}) alle quali il bot risponde fornendo un link per effettuare l'autenticazione (\textbf{UC1.1}). Se l'utente prova ad autentificarsi con un token\textsubscript{G} di accesso non valido viene visualizzata una schermata di errore. (\textbf{UC1.2})
\end{itemize}
\newpage

\subsubsection{UC1.1 - Token di accesso autenticazione}
\begin{itemize}
    \item \textbf{Identificativo}: UC1.1
    \item \textbf{Nome}: token\textsubscript{G} di accesso autenticazione
    \item \textbf{Descrizione grafica}: (approfondita in UC1)
    \item \textbf{Attori}
 \begin{itemize} 
    \item \textit{Primari}: utente non autorizzato
    \item \textit{Secondari}: non presenti
 \end{itemize}
 \item \textbf{Precondizione}: l'utente ha a disposizione un token\textsubscript{G} di accesso.
 \item \textbf{Postcondizione}: l'utente fornisce il token\textsubscript{G} al chatbot.
 \item \textbf{Scenario principale}: se il token\textsubscript{G} fornito è valido allora l'utente avrà effettuato il login correttamente e potrà usufruire dei sistemi messi a disposizione. Se il token\textsubscript{G} non risulta valido, verrà mostrato un messaggio di errore ad esso relativo (\textbf{UC1.2})
\end{itemize}

\subsubsection{UC1.2 - Token non valido}
\begin{itemize}
    \item \textbf{Identificativo}: UC1.2
    \item \textbf{Nome}: token\textsubscript{G} non valido
    \item \textbf{Descrizione grafica}: (approfondita in UC1)
    \item \textbf{Attori}
 \begin{itemize} 
    \item \textit{Primari}: utente non autorizzato 
    \item \textit{Secondari}: non presenti
 \end{itemize}
 \item \textbf{Precondizione}: l'utente ha a disposizione un token\textsubscript{G} non valido.
 \item \textbf{Postcondizione}: chatbot nega l'accesso ai servizi, viene comunicato l'errore all'utente e riproposto il sistema di login.
 \item \textbf{Scenario principale}: il token\textsubscript{G} inserito dall'utente risulta essere non valido per l'accesso, viene mostrato un messaggio di errore e il chatbot propone all'utente di rieffettuare la procedura di login.
\end{itemize}
\newpage
\newpage
\subsection{UC2 - Visualizzazione Errore Interpretazione}
\begin{itemize}
    \item \textbf{Identificativo}: UC2
    \item \textbf{Nome}: Visualizzazione errore interpretazione
    \item \textbf{Descrizione grafica}:
\end{itemize}

\begin{figure}[h]
    \includegraphics[scale=0.50]{images/UC2.png} 
    \caption{Diagramma UML del caso d'uso UC2 - Visualizzazione errore interpretazione}
\end{figure}

\begin{itemize}
    \item \textbf{Attori}:
    \begin{itemize} 
        \item \textit{Primari}: utente autorizzato
        \item \textit{Secondari}: non presenti
    \end{itemize}
 \item \textbf{Precondizione}: l'utente (precedentemente autenticato) sta scambiando una serie di messaggi con il chatbot (UC2.1.1) o (UC2.1.2) con lo scopo di eseguire una determinata azione.
 \item \textbf{Postcondizione}: il chatbot non è stato in grado di comprendere la richiesta dell'utente restituendo un messaggio di errore seguito da un insieme di comandi che è invece in grado di riconoscere.   
 \item \textbf{Scenario principale}: 
    \begin{itemize}
        \item Utente manda un messaggio al chatbot al fine di compire l'azione richiesta.
        \item Chatbot non è in grado di interpetare ciò che gli è stato chiesto.
        \item Chatbot restituisce un messaggio di errore e suggerisce alcuni comandi che l'utente può utilizzare.
    \end{itemize}
\end{itemize}
\newpage

\subsubsection{UC2.1 - Visualizzazione errore e suggerimento comandi}
\begin{itemize}
    \item \textbf{Identificativo}: UC2.1
    \item \textbf{Nome}: Visualizzazione errore e suggerimento comandi
    \item \textbf{Descrizione grafica}: (approfondita in UC2)
    \item \textbf{Attori}:
    \begin{itemize} 
        \item \textit{Primari}: utente autorizzato
        \item \textit{Secondari}: non presenti
    \end{itemize}
        \item \textbf{Precondizione}: il chatbot non è stato in grado di interpretare l'input fornito testualmente (UC2.1.1) oppure vocalmente (UC2.1.2), dall'utente (precedentemente) autenticato; verificandosi quindi una condizione di errore (UC2.2).
        \item \textbf{Postcondizione}: il chatbot mostra un messaggio di errore all'interno della chat e fornisce all'utente una serie di comandi che è in grado di interpretare in maniera corretta. 
     \item \textbf{Scenario principale}: 
        \begin{itemize}
            \item L'utente ha richiesto di eseguire un'operazione
            \item Chatbot non è in grado di interpetare ciò che gli è stato chiesto.
            \item Chatbot restituisce un messaggio di errore e suggerisce alcuni comandi che l'utente può utilizzare.
        \end{itemize}
\end{itemize}

\paragraph{UC2.1.1 - Richiesta testuale}
\begin{itemize}
    \item \textbf{Identificativo}: UC2.1.1
    \item \textbf{Nome}: Richiesta testuale
    \item \textbf{Descrizione grafica}: (approfondita in UC2)
    \item \textbf{Attori}:
    \begin{itemize} 
        \item \textit{Primari}: utente autorizzato
        \item \textit{Secondari}: non presenti
    \end{itemize}
        \item \textbf{Precondizione}: l'utente si è autenticato al sistema e vuole comunicare un messaggio al chatbot, in maniera testuale. 
        \item \textbf{Postcondizione}: l'utente fornisce un input testuale al chatbot. 
     \item \textbf{Scenario principale}: 
        \begin{itemize}
            \item L'utente ha effettuato l'accesso al sistema 
            \item L'utente fornisce un input testuale al chatbot 
        \end{itemize}
\end{itemize}

\paragraph{UC2.1.2 - Richiesta vocale}
\begin{itemize}
    \item \textbf{Identificativo}: UC2.1.2
    \item \textbf{Nome}: Richiesta vocale
    \item \textbf{Descrizione grafica}: (approfondita in UC2)
    \item \textbf{Attori}:
    \begin{itemize} 
        \item \textit{Primari}: utente autorizzato
        \item \textit{Secondari}: non presenti
    \end{itemize}
        \item \textbf{Precondizione}: l'utente si è autenticato al sistema e vuole comunicare un messaggio al chatbot, in maniera vocale. 
        \item \textbf{Postcondizione}: l'utente fornisce un input vocale al chatbot. 
     \item \textbf{Scenario principale}: 
        \begin{itemize}
            \item L'utente ha effettuato l'accesso al sistema 
            \item L'utente fornisce un input vocale al chatbot 
        \end{itemize}
\end{itemize}

\subsubsection{UC2.2 - Errore Interpretazione}
\begin{itemize}
    \item \textbf{Identificativo}: UC2.2
    \item \textbf{Nome}: errore interpretazione
    \item \textbf{Descrizione grafica}: (approfondita in UC2)
    \item \textbf{Attori}:
    \begin{itemize} 
        \item \textit{Primari}: utente autorizzato
        \item \textit{Secondari}: non presenti
    \end{itemize}
        \item \textbf{Precondizione}: l'utente ha fornito un input al chatbot, testuale (UC2.1.1) oppure vocale (UC2.1.2). 
        \item \textbf{Postcondizione}: il chatbot non è stato in grado di interpretare la richiesta dell'utente, vericando una condizione di errore. 
     \item \textbf{Scenario principale}: 
        \begin{itemize}
            \item Il chatbot ha ricevuto un input dall'utente che non è stato in grado di interpretare, causando un errore. 
        \end{itemize}
\end{itemize}


\newpage
\newpage

\newpage
\section{Requisiti}
Seguendo le Norme di Progetto ogni requisito è identificato da il suo codice, la sua descrizione e la fonte di provenienza.
\subsection{Requisiti Funzionali}
\begin{center}
\renewcommand{\arraystretch}{1.8} %aumento ampiezza righe
\begin{tabular}{ | m{8em} | m{18em} | m{12em} | }
\hline
Codice&Descrizione&Fonte\\
\hline
RO-F-1 & Il \glossario{ChatBot} deve poter riconoscere input testuale & Capitolato, UC1.1.1, UC2.1.1, UC3.1.1, UC4.1.1, UC5.1.1, UC6.1.1, UC6.2.1, UC6.3.1, UC6.4.1, UC7.1.1, UC7.2.1, UC8.1.1, UC8.2.1,  UC8.3.1, UC8.4.1\\
\hline
RO-F-2&Il \glossario{ChatBot} deve poter riconoscere input vocale&Capitolato, UC1.1.2, UC2.1.2, UC3.1.2, UC4.1.2, UC 5.1.2, UC6.1.2, UC6.2.2, UC6.3.2, UC6.4.2, UC7.1.2, UC7.2.2, UC8.1.2, UC8.2.2,  UC8.3.2, UC8.4.2\\
\hline
RO-F-3&L’utente deve poter autenticarsi tramite un \glossario{token}&UC1\\
\hline
RO-F-4&Il programma deve poter riconoscere un \glossario{token} non valido&UC1\\
\hline
RO-F-5&Il programma deve poter far visualizzare un errore se il \glossario{token} non è valido&UC1.2\\
\hline
RO-F-6&L’utente non autenticato deve poter richiedere un \glossario{token} per autenticarsi&UC1.1\\
\hline
RD-F-7&Se l’utente ha più \glossario{token} a disposizione, può decidere quale usare.&Verbale Esterno\\
\hline
RO-F-8&Il \glossario{ChatBot} deve essere in grado di riconoscere i comandi non validi.&UC2\\
\hline
RO-F-9&Il \glossario{ChatBot} deve restituire un messaggio di errore con suggerimenti dopo un messaggio non valido &UC2 \\
\hline
\end{tabular}
\newpage
\begin{tabular}{ | m{8em} | m{18em} | m{12em} | }
\hline
RO-F-10&L’utente deve poter effettuare la registrazione della propria presenza &UC3 \\
\hline
RO-F-11&L’utente deve poter inserire la sede per la registrazione della presenza &UC3.1 \\
\hline
RO-F-12&Il \glossario{ChatBot} deve notificare l’utente se la sede per la registrazione non è valida &UC3.2 \\
\hline
RO-F-13&L’utente deve poter inserire un'\glossario{attività} all’interno del \glossario{sistema EMT} &UC4 \\
\hline
RO-F-14&L’utente deve poter inserire il tipo dell’\glossario{attività} &UC4.1 \\
\hline
RO-F-15&L’utente deve poter inserire le ore da consuntivare nell’\glossario{attività} &UC4.2 \\
\hline  
RO-F-16&L’utente deve poter inserire il progetto da consuntivare nell’\glossario{attività} &UC4.3 \\
\hline
RO-F-17&L’utente deve poter inserire il luogo in cui ha svolto l’\glossario{attività} &UC4.4 \\
\hline
RO-F-18&Il \glossario{ChatBot} deve notificare l'utente se l'\glossario{attività} è in un formato non valido &UC4.5 \\
\hline
RO-F-19&Il \glossario{ChatBot} deve notificare l'utente se le ore sono in un formato non valido &UC4.6 \\
\hline
RO-F-20&Il \glossario{ChatBot} deve notificare l'utente se il nome del progetto di un'\glossario{attività} è in un formato non valido &UC4.7 \\
\hline
RO-F-21&Il \glossario{ChatBot} deve notificare l'utente se il luogo relativo all’\glossario{attività} è in un formato non valido &UC4.8 \\
\hline
RD-F-22&L’utente deve poter aprire il cancello di un sede tramite \glossario{ChatBot} &UC5 \\
\hline
RD-F-23&Il \glossario{ChatBot} deve notificare l'utente se la sede indicata non è valida &UC5.2 \\
\hline
RD-F-24&L’utente deve poter creare, tramite il \glossario{ChatBot}, una riunione su un'applicazione esterna &UC6 \\
\hline
\end{tabular}
\newpage
\begin{tabular}{ | m{8em} | m{18em} | m{12em} | }
\hline
RD-F-25&L’utente deve poter inserire il nome della piattaforma esterna per la videoconferenza &UC6.1 \\
\hline
RD-F-26&L’utente deve poter inserire la data per la riunione &UC6.2 \\
\hline
RD-F-27&L’utente deve poter inserire l’ora per la riunione &UC6.3 \\
\hline
RD-F-28&L’utente deve poter inserire i partecipanti alla riunione &UC6.4 \\
\hline
RD-F-29&Il \glossario{ChatBot} deve notificare l’utente se la piattaforma inserita non è valida o non è supportata &UC6.5 \\
\hline
RD-F-30&Il \glossario{ChatBot} deve notificare l’utente se la data inserita non è valida o è indisponibile &UC6.6 \\
\hline
RD-F-31&Il \glossario{ChatBot} deve notificare l’utente se l’ora inserita non è valida o è indisponibile &UC6.7 \\
\hline
RD-F-32&Il \glossario{ChatBot} deve notificare l’utente se i partecipanti inseriti non sono corretti, chiedendo il reinserimento &UC6.8 \\
\hline
RD-F-33&L’utente deve poter chiedere di cercare un documento &UC7 \\
\hline
RD-F-34&L’utente deve poter inserire il nome del progetto per la ricerca del documento &UC7.1 \\
\hline
RD-F-35&L’utente deve poter inserire il nome del documento in cui vuole fare la ricerca &UC7.2 \\
\hline
RD-F-36&Il \glossario{ChatBot} deve notificare l’utente se l’inserimento del progetto non è corretto, chiedendo il reinserimento &UC7.3\\
\hline
RD-F-37&Il \glossario{ChatBot} deve notificare l’utente se il nome del documento inserito non è valido, chiedendo il reinserimento &UC7.4 \\
\hline
RD-F-38&L’utente deve poter inserire un \glossario{ticket} &UC8 \\
\hline
\end{tabular}
\newpage
\begin{tabular}{ | m{8em} | m{18em} | m{12em} | }
\hline
RD-F-39&L’utente deve poter fornire l’oggetto per la creazione del \glossario{ticket} &UC8.1 \\
\hline
RD-F-40&L’utente deve poter inserire la descrizione per la creazione del \glossario{ticket} dopo aver fornito l’oggetto &UC8.2 \\
\hline
RD-F-41&L’utente deve poter comunicare lo status del \glossario{ticket} dopo aver inserito la descrizione &UC8.3 \\
\hline
RD-F-42&L’utente deve poter inserire la priorità del \glossario{ticket} dopo aver comunicato lo status &UC8.4 \\
\hline
RD-F-43&Il \glossario{ChatBot} deve notificare l’utente se l’oggetto non è stato inserito in maniera idonea per la creazione del \glossario{ticket}  &UC8.5 \\
\hline
RD-F-44&Il \glossario{ChatBot} deve notificare l’utente se lo status del \glossario{ticket} non è stato inserito correttamente  &UC8.6 \\
\hline
RD-F-45&Il \glossario{ChatBot} deve notificare l’utente se la priorità del \glossario{ticket} non è stato inserita in un formato valido &UC8.7 \\
\hline
RO-F-46&Ogni qualvolta un operazione è avvenuta con successo, l’utente deve essere sempre notificato &UC9 \\
\hline
RO-F-47&Ogni qualvolta che non è possibile eseguire una richiesta, l’utente deve essere notificato &UC10 \\
\hline
RO-F-48&L’utente deve poter annullare un operazione in corso &UC11 \\
\hline
RO-F-49&Il \glossario{ChatBot} deve notificare l’utente se l’annullamento è avvenuto con successo &UC11 \\
\hline
RO-F-50&Ogni qualvolta l’utente tenta di effettuare un’operazione non consentita, il \glossario{ChatBot} deve fermare e notificare l'utente.&UC12 \\
\hline
RO-F-51&L’utente deve poter riconoscere il proprio stato di \glossario{check-in}/Check-out tramite messaggio &UC13 \\
\hline
\end{tabular}
\newpage

\begin{tabular}{ | m{8em} | m{18em} | m{12em} | }
\hline
RO-F-52&Il \glossario{ChatBot} deve restituire lo stato di \glossario{check-in}/Check-out dopo la richiesta del utente &UC13\\
\hline
RO-F-53&L’utente deve poter visualizzare le proprie ore consuntivate &UC14 \\
\hline
RO-F-54&L’utente deve poter visualizzare le proprie ore da consuntivare rimanenti &UC15 \\
\hline
RD-F-55&L’utente deve poter visualizzare correttamente i documenti trovati dopo la ricerca &UC16 \\
\hline
RD-F-56&L’utente deve poter visualizzare correttamente tutte le sue riunioni del giorno & UC17\\
\hline
RO-F-57&L’utente deve poter visualizzare le proprie impostazioni &UC18 \\
\hline
RD-F-58&L’utente deve poter richiedere il link per l’autenticazione alla piattaforma di riunione esterna &UC19\\
\hline
RD-F-59&Dopo aver ricevuto il token il programma deve essere in grado di inserirlo dentro piattaforma esterna &UC19.1 \\
\hline
RO-F-60&Il \glossario{ChatBot} deve notificare l'utente se la sede fornita è valida ma non è stata trovata nel sistema.&UC20 \\
\hline
RO-F-61&Il \glossario{ChatBot} deve riconoscere anche i linguaggi naturali dei comandi disponibili &Capitolato \\
\hline
RO-F-62&Il \glossario{ChatBot} deve notificare l'utente se il nome di un progetto è valido ma non è stato trovato nel sistema. &UC21 \\
\hline
\end{tabular}
\end{center}
\newpage

\subsection{Requisiti di qualità}
\begin{center}
\renewcommand{\arraystretch}{1.8} %aumento ampiezza righe
\begin{tabular}{ | m{8em} | m{18em} | m{12em} | }
\hline
Codice&Descrizione&Fonte\\
\hline
RO-Q-1&Almeno l'80\% delle funzioni del programma deve essere testato e correlato dal report del test&Capitlato\\
\hline
RO-Q-2&Redarre documenti sulle scelte implementative e progettuali e relative motivazioni&Capitolato\\
\hline
RO-Q-3&Redarre documenti su problemi aperti e soluzioni possibili&Capitolato\\
\hline
RO-Q-4&Rispettare il piano di qualifica&Verbale interno\\
\hline
\end{tabular}
\end{center}

\subsection{Requisiti di vincolo}
\begin{center}
\renewcommand{\arraystretch}{1.8} %aumento ampiezza righe
\begin{tabular}{ | m{8em} | m{18em} | m{12em} | }
\hline
Codice&Descrizione&Fonte\\
\hline
RO-V-1&Sviluppare un applicazione mobile (IOS o Android) che permette di utilizzare tutte le funzioni descritte dai requisiti funzionali&Capitolato\\
\hline
RF-V-2&Cifrare tutte le comunicazioni fra App e Server per garantire la validità delle informazioni&Capitolato\\
\hline
\end{tabular}
\end{center}

\newpage
\section{Tracciamento dei requisiti}
Vengono qui raccolte e riassunte tutte le combinazioni requisito - fonte per motivi di praticità.
\subsection{Tracciamento Filtrato per Fonte}
\begin{center}
\renewcommand{\arraystretch}{1.8} %aumento ampiezza righe
\begin{tabular}{ |m{8em}|m{13em}| }
    \hline
    \textbf{Fonte} & \textbf{Requisito} \\
    \hline
    UC1         &   RO-F-3 \\
    \hline
    UC1.1       &   RO-F-6 \\
    \hline
    UC1.1.1     &   RO-F-1 \\
    \hline
    UC1.2       &   RO-F-5, RO-F-4 \\
    \hline
    UC2         &   RO-F-8 \\
    \hline
    UC2.1       &   RO-F-9 \\
    \hline
    UC2.1.1     &   RO-F-1 \\
    \hline
    UC2.1.2     &   RO-F-2 \\
    \hline
    UC2.2       &   RO-F-10 \\
    \hline
    UC3         &   RO-F-11 \\
    \hline
    UC3.1       &   RO-F-12 \\
    \hline
    UC3.1.1     &   RO-F-1 \\
    \hline
    UC3.1.2     &   RO-F-2 \\
    \hline
    UC3.2       &   RO-F-13 \\
    \hline
    UC3.2.1     &   RO-F-1 \\
    \hline
    UC3.2.2     &   RO-F-2 \\
    \hline
    UC3.3       &   RO-F-14 \\
    \hline
    UC3.3.1     &   RO-F-1 \\
    \hline
    UC3.3.2     &   RO-F-2 \\
    \hline
    UC3.4       &   RO-F-15 \\
    \hline
    UC3.4.1     &   RO-F-1 \\
    \hline
    \end{tabular}
    \newpage
    \begin{tabular}{ |m{8em}|m{13em}| }
    \hline
    UC3.4.2     &   RO-F-2 \\
    \hline
    UC3.5       &   RO-F-16 \\
    \hline
    UC3.6       &   RO-F-17 \\
    \hline
    UC3.7       &   RO-F-18 \\
    \hline
    UC3.8       &   RO-F-19 \\
    \hline
    UC4         &   RD-F-20 \\
    \hline
    UC4.1       &   RO-F-21 \\
    \hline
    UC4.1.1     &   RO-F-1 \\
    \hline
    UC4.1.2     &   RO-F-2 \\
    \hline
    UC4.2       &   RD-F-22 \\
    \hline
    UC5         &   RD-F-23 \\
    \hline
    UC5.1       &   RD-F-24 \\
    \hline
    UC5.1.1     &   RO-F-1 \\
    \hline
    UC5.1.2     &   RO-F-2 \\
    \hline
    UC5.2       &   RD-F-25 \\
    \hline
    UC5.2.1     &   RO-F-1 \\
    \hline
    UC5.2.2     &   RO-F-2 \\
    \hline
    UC5.3       &   RD-F-26 \\
    \hline
    UC5.3.1     &   RO-F-1 \\
    \hline
    UC5.3.2     &   RO-F-2 \\
    \hline
    UC5.4       &   RD-F-27 \\
    \hline
    UC5.4.1     &   RO-F-1 \\
    \hline
    UC5.4.2     &   RO-F-2 \\
    \hline
    UC5.5       &   RD-F-28 \\
    \hline
    UC5.6       &   RD-F-29 \\
    \hline
    UC5.7       &   RD-F-30 \\
    \hline
    \end{tabular}
    \newpage
    \begin{tabular}{ |m{8em}|m{13em}| }
    \hline
    UC5.8       &   RD-F-31 \\
    \hline
    UC6         &   RD-F-32 \\
    \hline
    UC6.1       &   RD-F-33 \\
    \hline
    UC6.1.1     &   RO-F-1 \\
    \hline
    UC6.1.2     &   RO-F-2 \\
    \hline
    UC6.2       &   RD-F-34 \\
    \hline
    UC6.2.1     &   RO-F-1 \\
    \hline
    UC6.2.2     &   RO-F-2 \\
    \hline
    UC6.3       &   RD-F-35 \\
    \hline
    UC6.4       &   RD-F-36 \\
    \hline
    UC7         &   RD-F-37 \\
    \hline
    UC7.1       &   RD-F-38 \\
    \hline
    UC7.1.1     &   RO-F-1 \\
    \hline
    UC7.1.2     &   RO-F-2 \\
    \hline
    UC7.2       &   RD-F-39 \\
    \hline
    UC7.2.1     &   RO-F-1 \\
    \hline
    UC7.2.2     &   RO-F-2 \\
    \hline
    UC7.3       &   RD-F-40 \\
    \hline
    UC7.3.1     &   RO-F-1 \\
    \hline
    UC7.3.2     &   RO-F-2 \\
    \hline
    UC7.4       &   RD-F-41 \\
    \hline
    UC7.5       &   RD-F-42 \\
    \hline
    UC8         &   RD-F-43 \\
    \hline
    UC8.1       &   RD-F-43 \\
    \hline
    UC8.1.1     &   RO-F-1 \\
    \hline
    UC8.1.2     &   RO-F-2 \\
    \hline
    \end{tabular}
    \newpage
    \begin{tabular}{ |m{8em}|m{13em}| }
    \hline
    UC8.2       &   RD-F-44 \\
    \hline
    UC9         &   RD-F-45 \\
    \hline
    UC9.1       &   RD-F-45 \\
    \hline
    UC9.1.1     &   RO-F-1 \\
    \hline
    UC9.1.2     &   RO-F-2 \\
    \hline
    UC9.2       &   RD-F-46 \\
    \hline
    UC10        &   RD-F-47 \\
    \hline
    UC10.1      &   RD-F-47 \\
    \hline
    UC10.1.1    &   RO-F-1 \\
    \hline
    UC10.1.2    &   RO-F-2 \\
    \hline
    UC10.2      &   RD-F-48 \\
    \hline
    UC11        &   RD-F-49 \\
    \hline
    UC11.1      &   RD-F-49 \\
    \hline
    UC11.1.1    &   RO-F-1 \\
    \hline
    UC11.1.2    &   RO-F-2 \\
    \hline
    UC11.2      &   RD-F-50 \\
    \hline
    UC12        &   RD-F-51 \\
    \hline
    UC12.1      &   RD-F-51 \\
    \hline
    UC12.1.1    &   RO-F-1 \\
    \hline
    UC12.1.2    &   RO-F-2 \\
    \hline
    UC12.2      &   RD-F-52 \\
    \hline
    UC13        &   RD-F-53 \\
    \hline
    UC14        &   RD-F-54 \\
    \hline
    UC14.1      &   RD-F-54 \\
    \hline
    UC14.2      &   RD-F-55 \\
    \hline
    \end{tabular}
    \newpage
    \begin{tabular}{ |m{8em}|m{13em}| }
    \hline
    UC14.3      &   RD-F-56 \\
    \hline
    UC14.4      &   RD-F-57 \\
    \hline
    Interno     &   RO-Q-4 \\
    \hline
    Esterno     &   RD-F-7 \\
    \hline
    Capitolato  &   RO-F-1, RO-F-2, RD-F-58, RO-Q-1, RO-Q-2, RO-Q-3, RO-V-1 \\
    \hline
\end{tabular}
\end{center}
\subsection{Tracciamento Filtrato per Requisito}
\begin{center}
\renewcommand{\arraystretch}{1.8} %aumento ampiezza righe
\begin{tabular}{ |m{8em}|m{13em}| }
    \hline
    \textbf{Retuisito} & \textbf{Fonte} \\
    \hline
    RO-F-1  &   Capitolato, UC1.1.1, UC2.1.1, UC3.1.1, UC3.2.1, UC3.3.1, UC3.4.1, UC4.1.1, UC5.1.1, UC5.2.1, UC5.3.1, UC5.4.1, UC6.1.1, UC6.2.1, UC7.1.1, UC7.2.1, UC7.3.1, UC8.1.1, UC9.1.1, UC10.1.1, UC11.1.1, UC12.1.1 \\
    \hline
    RO-F-2  &   Capitolato, UC2.1.2, UC3.1.2, UC3.2.2, UC3.3.2, UC3.4.2, UC4.1.2, UC5.1.2, UC5.2.2, UC5.3.2, UC5.4.2, UC6.1.2, UC6.2.2, UC7.1.2, UC7.2.2, UC7.3.2, UC8.1.2, UC9.1.2, UC10.1.2, UC11.1.2, UC12.1.2 \\
    \hline
    RO-F-3  &   UC1 \\
    \hline
    RO-F-4  &   UC1.2 \\
    \hline
    RO-F-5  &   UC1.2 \\
    \hline
    RO-F-6  &   UC1.1 \\
    \hline
    RD-F-7  &   Esterno \\
    \hline
    RO-F-8  &   UC2 \\
    \hline
    RO-F-9  &   UC2.1 \\
    \hline
    RO-F-10  &  UC2.2 \\
    \hline
    RO-F-11  &  UC3 \\
    \hline
    RO-F-12  &  UC3.1 \\
    \hline
    RO-F-13  &  UC3.2 \\
    \hline
    RO-F-14  &  UC3.3 \\
    \hline
    RO-F-15  &  UC3.4 \\
    \hline
    \end{tabular}
    \newpage
    \begin{tabular}{ |m{8em}|m{13em}| }
    \hline
    RO-F-16  &  UC3.5 \\
    \hline
    RO-F-17  &  UC3.6 \\
    \hline
    RO-F-18  &  UC3.7 \\
    \hline
    RO-F-19  &  UC3.8 \\
    \hline
    RD-F-20  &  UC4 \\
    \hline
    RO-F-21  &  UC4.1 \\
    \hline
    RD-F-22  &  UC4.2 \\
    \hline
    RD-F-23  &  UC5 \\
    \hline
    RD-F-24  &  UC5.1 \\
    \hline
    RD-F-25  &  UC5.2 \\
    \hline
    RD-F-26  &  UC5.3 \\
    \hline
    RD-F-27  &  UC5.4 \\
    \hline
    RD-F-28  &  UC5.5 \\
    \hline
    RD-F-29  &  UC5.6 \\
    \hline
    RD-F-30  &  UC5.7 \\
    \hline
    RD-F-31  &  UC5.8 \\
    \hline
    RD-F-32  &  UC6 \\
    \hline
    RD-F-33  &  UC6.1 \\
    \hline
    RD-F-34  &  UC6.2 \\
    \hline
    RD-F-35  &  UC6.3 \\
    \hline
    RD-F-36  &  UC6.4 \\
    \hline
    RD-F-37  &  UC7 \\
    \hline
    RD-F-38  &  UC7.1 \\
    \hline
    RD-F-39  &  UC7.2 \\
    \hline
    RD-F-40  &  UC7.3 \\
    \hline
    \end{tabular}
    \newpage
    \begin{tabular}{ |m{8em}|m{13em}| }
    \hline
    RD-F-41  &  UC7.4 \\
    \hline
    RD-F-42  &  UC7.5 \\
    \hline
    RD-F-43  &  UC8, UC8.1 \\
    \hline
    RD-F-44  &  UC8.2 \\
    \hline
    RD-F-45  &  UC9, UC9.1 \\
    \hline
    RD-F-46  &  UC9.2 \\
    \hline
    RD-F-47  &  UC10, UC10.1 \\
    \hline
    RD-F-48  &  UC10.2 \\
    \hline
    RD-F-49  &  UC11, UC11.1 \\
    \hline
    RD-F-50  &  UC11.2 \\
    \hline
    RD-F-51  &  UC12, UC12.1 \\
    \hline
    RD-F-52  &  UC12.2 \\
    \hline
    RD-F-53  &  UC13 \\
    \hline
    RD-F-54  &  UC14, UC14.1 \\
    \hline
    RD-F-55  &  UC14.2 \\
    \hline
    RD-F-56  &  UC14.3 \\
    \hline
    RD-F-57  &  UC14.4 \\
    \hline
    RD-F-58  &  Capitolato \\
    \hline
    RO-Q-1  &  Capitolato \\
    \hline
    RO-Q-2  &  Capitolato \\
    \hline
    RO-Q-3  &  Capitolato \\
    \hline
    RO-Q-4  &  Interno \\
    \hline
    RO-V-1  &  Capitolato \\
    \hline
\end{tabular}
\end{center}
\newpage

\end{document}